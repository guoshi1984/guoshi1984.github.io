\documentclass[a4paper]{article}
\usepackage{amssymb, amsmath}
\usepackage{graphicx}
\begin{document}
\section{Josephson Junction}
basic IV relationship
\begin{align*}
	I_s = I_c sin(\phi)
	\dot \phi = 2eV/\hbar
\end{align*}
where $\phi = \phi_1 - \phi_2$ is the phase difference between the two superconducting electronodes.\\
The energy in a Josephson junction\\
\begin{align*}
        E = E_J(1 - cos \phi)
\end{align*}
subsection{Quantizing Electrical Circuits}
An electrical circuit can be described as a number of nodes connected through circuit elements. As generalized coordinates for such a circuit it is often convenient to use the node fluxes
\begin{align*}
        \Phi_n(t) = \int_{-\infty}^{t} V_n(t^{'})dt^'
\end{align*}
Where $V_n$ denotes the node voltage at node n. The corresponding generalized momenta will usually, but not every time, be the node charges
\begin{align*}
        Q_n(t) = \int_{-\infty}^t I_n(t^{'}) dt^'
\end{align*}
Where $I_n$ denotes the node current. If there is a loop l in the circuit, the voltage drop around that loop should be zero, which implies
\begin{align*}
        \sum_{b\hspace{2mm} around \hspace{2mm} l} \Phi_b = \Phi_{ext}
\end{align*}
And the external magnetic flux is constrained by the quantization condition $\Phi_{ext} = m \Phi_0$, where $m\in Z$ and $\Phi_0 = \frac{h}{2e}$ is the flux quantum.\\
The generalized coordinates obey the canonical commutation relation
\begin{align*}
        [\Phi, Q] = i \hbar
\end{align*}
Similarly, we can write down the energy for capacitor
\begin{align*}
        \frac{CV^2}{2} = \frac{C( \dot{\Phi}_1- \dot{\Phi}_2)}{2}
\end{align*}
\begin{align*}
        \frac{LI^2}{2} = \frac{( \Phi_1- \Phi_2)}{2L}
\end{align*}

Node flux\\
\begin{align*}
	\Phi_n(t) = \int^t_{\-infty} V_n(t^{'}) dt^{'}
\end{align*}
Node charge\\
\begin{align*}
	Q_n(t) = \int^t_{\-infty} I_n(t^{'}) dt^{'}
\end{align*}
\end{document}
Kirchhoff's law
\begin{align*}
	\sum_{loop} \Phi_b = \Phi_{ext}
\end{align*}
the $\Phi_{ext}$ is constrained by the quantization condition
\begin{align*}
	\Phi_{ext} = m \Phi_0
\end{align*}
Energy in capacitor
\begin{align*}
	\frac{CV^2}{2} = \frac{C(\dot \Phi_2 - \dot \Phi_1)}{2}
\end{align*}
Energy in inductor
\begin{align*}
	\frac{LI^2}{2} = \frac{(\Phi_1 - \Phi_2)^2}{2L}
\end{align*}
Energy in J jucntion
\begin{align*}
	\int^t_{-\infty} I(\tau)V(\tau)d\tau = E_J(1-cos\phi)
\end{align*}

Hamiltonian in Josephson junction
\begin{align*}
        H = E_{co} n^2 - E_J cos \phi - \hbar/(2e)I\phi
\end{align*}
Where $E_{co} = (2e)^2/(2C)$ is the charging energy.
Hamiltonian for the charge qubit
\begin{align*}
	H_{CPB} = 4 E_C(n-n_g)^2-E_J cos\phi
\end{align*}
Hamiltonian for the phase qubit
\begin{align*}
	H_{CPB} = \frac{2\pi}{\Phi_0} \frac{p^2}{2C_J} -\frac{\Phi_0}{2\pi}I_b \phi-E_J cos\phi
\end{align*}
