\documentclass[a4paper]{article}
\title{Modern Theory of Conduction and Polarization}
\usepackage{amssymb, amsmath}
\usepackage{graphicx}
\newcommand{\BZ}{\text{BZ}} % Brillouin Zone
\newcommand{\occ}{\text{occ}} % occupied bands
\newcommand{\kvec}{\mathbf{k}}
\newcommand{\rvec}{\mathbf{r}}
\newcommand{\vvec}{\mathbf{v}}
\newcommand{\Evec}{\mathbf{E}}
\newcommand{\Bvec}{\mathbf{B}}
\begin{document}
\section*{Abstract}
This note traces the development of electrical conduction theory from early classical models to the modern quantum geometric formulation. We begin with the Drude model (1900) and Sommerfeld free electron model (1927), highlight their successes and limitations, then progress to the full quantum treatment incorporating band theory and Berry phase effects.
\section{Classical Foundations: The Drude Model (1900)}
\subsection{Basic Assumptions}
The Drude model, proposed by Paul Drude in 1900, treats electrons in a metal as a classical gas of charged particles. Key assumptions:
\begin{itemize}
    \item Electrons move freely between collisions with stationary ions.
    \item Collisions are instantaneous and randomize electron velocity.
    \item Electron-ion collisions occur with a characteristic \textbf{relaxation time} $\tau$.
    \item Between collisions, electrons accelerate under external fields: $m\dot{\vvec} = -e\Evec$.
\end{itemize}

\subsection{DC Conductivity}
Under a constant electric field $\Evec$, the average drift velocity reaches a steady state:
\begin{align}
\vvec_d = -\frac{e\tau}{m}\Evec
\end{align}
The current density $\mathbf{j} = -ne\vvec_d$ gives \textbf{Ohm's law}:
\begin{align}
\mathbf{j} = \sigma \Evec, \quad \sigma = \frac{ne^2\tau}{m}
\end{align}
where $n$ is the electron density.

\subsection{AC Response and Limitations}
For time-dependent fields $\Evec(t) = \Evec_0 e^{-i\omega t}$:
\begin{align}
\sigma(\omega) = \frac{ne^2\tau}{m(1-i\omega\tau)}
\end{align}
This predicts the \textbf{Drude peak} at $\omega=0$ and a plasma frequency at $\omega_p = \sqrt{ne^2/m\epsilon_0}$.

\textbf{Limitations of the Drude model:}
\begin{enumerate}
    \item \textbf{Wrong temperature dependence}: Predicts $\sigma \propto 1/\sqrt{T}$ (from kinetic theory) but metals show $\sigma \propto 1/T$.
    \item \textbf{Wrong heat capacity}: Classical equipartition gives $C_v = \frac{3}{2}nk_B$, but electron contribution is $\sim 100\times$ smaller.
    \item \textbf{Ignores quantum statistics}: Treats electrons as classical Maxwell-Boltzmann particles.
    \item \textbf{No explanation for insulators}: Cannot explain why some materials don't conduct at all.
\end{enumerate}
\section{Sommerfeld Free Electron Model (1927)}

\subsection{Quantum Mechanical Revision}
Arnold Sommerfeld incorporated quantum mechanics and Fermi-Dirac statistics:
\begin{itemize}
    \item Electrons are quantum particles obeying Pauli exclusion principle.
    \item Energy states: $E(\kvec) = \frac{\hbar^2 k^2}{2m}$ for free electrons.
    \item At $T=0$, electrons fill states up to the \textbf{Fermi energy} $E_F = \frac{\hbar^2 k_F^2}{2m}$.
    \item Fermi wavevector: $k_F = (3\pi^2 n)^{1/3}$.
    \item Fermi velocity: $v_F = \frac{\hbar k_F}{m} \sim 10^6$ m/s.
\end{itemize}

\subsection{Improved Predictions}
\begin{align}
\text{Heat capacity: } & C_e = \gamma T, \quad \gamma = \frac{\pi^2}{2} \frac{nk_B^2}{E_F} \\
\text{DC conductivity: } & \sigma = \frac{ne^2\tau}{m^*} \quad \text{(same form as Drude)} \\
\text{But now: } & \tau \sim \ell/v_F, \quad \ell = \text{mean free path}
\end{align}
This resolves the heat capacity discrepancy and gives better temperature dependence.

\subsection{The Boltzmann Equation Approach}
For a distribution function $f(\rvec, \kvec, t)$, the semi-classical Boltzmann equation is:
\begin{align}
\frac{\partial f}{\partial t} + \vvec \cdot \nabla_{\rvec} f - \frac{e}{\hbar}(\Evec + \vvec \times \Bvec) \cdot \nabla_{\kvec} f = \left(\frac{\partial f}{\partial t}\right)_{\text{coll}}
\end{align}
In relaxation time approximation:
\begin{align}
\left(\frac{\partial f}{\partial t}\right)_{\text{coll}} = -\frac{f - f_0}{\tau}
\end{align}
For DC conductivity, the solution gives:
\begin{align}
f(\kvec) = f_0(\kvec) - \frac{e\tau}{\hbar} \Evec \cdot \nabla_{\kvec} f_0(\kvec)
\end{align}
and current:
\begin{align}
\mathbf{j} = -e \int \frac{d^3k}{(2\pi)^3} \vvec(\kvec) f(\kvec)
\end{align}

\subsection{Remaining Limitations of Sommerfeld Model}
\begin{enumerate}
    \item \textbf{Ignores periodic lattice potential}: Cannot explain:
    \begin{itemize}
        \item Why some materials are insulators
        \item Effective mass $m^* \neq m$
        \item Existence of holes
    \end{itemize}
    \item \textbf{No explanation for Hall coefficient sign}: Free electron model predicts $R_H = -1/(ne)$, but some materials show positive Hall coefficient.
    \item \textbf{Cannot explain anomalous Hall effect}: Observed in ferromagnets without external magnetic field.
    \item \textbf{No geometric/topological effects}: Misses Berry phase contributions.
\end{enumerate}
\section{Band Theory Foundations}

The failures of the free electron models motivated the development of band theory, which properly accounts for the periodic crystal potential. Two complementary approaches emerge: the \textbf{nearly-free electron model} (weak potential limit) and the \textbf{tight-binding model} (strong potential limit).

\subsection{Nearly-Free Electron Model}

For electrons in a weak periodic potential $V(\rvec) = V(\rvec + \mathbf{R})$, we treat the potential as a perturbation to free electrons.

\subsubsection{Free Electron Basis}
For a one-dimensional system of length $L$ (with periodic boundary conditions), the unperturbed Hamiltonian is:
\[
H_0 = -\frac{\hbar^2}{2m}\frac{d^2}{dx^2}
\]
The eigenstates are plane waves:
\[
\psi_k^{(0)}(x) = \frac{1}{\sqrt{L}} e^{ikx}, \quad E^{(0)}(k) = \frac{\hbar^2 k^2}{2m}
\]
where $k = \frac{2\pi n}{L}$ with integer $n$.

\subsubsection{Weak Periodic Potential}
We introduce a weak periodic potential with lattice constant $a$:
\[
V(x) = V(x + a)
\]
For simplicity, we consider only the dominant Fourier component:
\[
V(x) = 2V_0 \cos\left(\frac{2\pi}{a}x\right) = V_0\left(e^{iGx} + e^{-iGx}\right)
\]
where $G = \frac{2\pi}{a}$ is the smallest reciprocal lattice vector. We assume $V_0$ is small and real.

\subsubsection{Degeneracy at Brillouin Zone Boundary}

{\bf Free Electron Degeneracy Condition}\\
Two free electron states $|k\rangle$ and $|k'\rangle$ are degenerate when:
\[
E^{(0)}(k) = E^{(0)}(k') \implies k' = \pm k
\]
However, in a periodic potential, states differing by a reciprocal lattice vector are physically equivalent. The important degeneracy occurs when:
\[
k' = k - G
\]
with:
\[
E^{(0)}(k) = E^{(0)}(k - G)
\]

\noindent{\bf Solution of Degeneracy Condition}\\
Solving:
\[
\frac{\hbar^2 k^2}{2m} = \frac{\hbar^2 (k-G)^2}{2m} \implies k^2 = (k-G)^2
\]
This gives:
\[
k = \frac{G}{2} = \frac{\pi}{a}
\]
This is precisely the \textbf{boundary of the first Brillouin zone}.

\subsubsection{Degenerate Perturbation Theory at $k = \pi/a$}

{\bf Basis States}\\
We consider the two nearly degenerate states at $k \approx \pi/a$:
\[
|k\rangle = \frac{1}{\sqrt{L}} e^{ikx}, \quad |k-G\rangle = \frac{1}{\sqrt{L}} e^{i(k-G)x}
\]

\noindent{\bf Perturbation Matrix Elements}\\
{\bf a. Diagonal Elements}\\
Assuming the average potential is zero (or absorbed into energy zero):
\[
\langle k|V|k\rangle = \frac{1}{L}\int_0^L V(x) dx = 0
\]
Similarly:
\[
\langle k-G|V|k-G\rangle = 0
\]
Thus:
\[
\langle k|H|k\rangle = E^{(0)}(k), \quad \langle k-G|H|k-G\rangle = E^{(0)}(k-G)
\]

\noindent{\bf b. Off-Diagonal Elements (Crucial!)}\\
\begin{align*}
\langle k|V|k-G\rangle &= \frac{1}{L}\int_0^L e^{-ikx} V_0\left(e^{iGx} + e^{-iGx}\right) e^{i(k-G)x} dx \\
&= \frac{V_0}{L}\int_0^L \left[e^{0} + e^{-2iGx}\right] dx \\
&= V_0 + 0 = V_0
\end{align*}
The second term integrates to zero over a period because $e^{-2iGx}$ oscillates with period $a/2$.

Similarly:
\[
\langle k-G|V|k\rangle = V_0
\]

\subsubsection{SECULAR Equation}
The perturbation matrix in the $\{|k\rangle, |k-G\rangle\}$ basis is:
\[
\mathbf{H} = \begin{pmatrix}
E^{(0)}(k) & V_0 \\
V_0 & E^{(0)}(k-G)
\end{pmatrix}
\]

The eigenvalues satisfy:
\[
\det\begin{pmatrix}
E^{(0)}(k) - E & V_0 \\
V_0 & E^{(0)}(k-G) - E
\end{pmatrix} = 0
\]
which gives:
\[
\left[E^{(0)}(k) - E\right]\left[E^{(0)}(k-G) - E\right] - V_0^2 = 0
\]

\subsubsection{Energy Gap Opening at Zone Boundary}

At $k = \pi/a$, we have:
\[
E^{(0)}(k) = E^{(0)}(k-G) \equiv E_0 = \frac{\hbar^2}{2m}\left(\frac{\pi}{a}\right)^2
\]
The secular equation simplifies to:
\[
(E_0 - E)^2 - V_0^2 = 0 \implies E_{\pm} = E_0 \pm |V_0|
\]

Thus, the two degenerate free electron levels split into two distinct energy levels separated by an \textbf{energy gap}:
\[
\boxed{E_g = 2|V_0|}
\]

Solving the quadratic equation generally:
\[
E_{\pm}(k) = \frac{1}{2}\left[E^{(0)}(k) + E^{(0)}(k-G) \pm \sqrt{\left(E^{(0)}(k) - E^{(0)}(k-G)\right)^2 + 4V_0^2}\right]
\]

%\subsubsection{Near the Zone Boundary}
Let $q = k - \pi/a$ be a small deviation from the zone boundary. Expanding to first order:
\[
E^{(0)}(k) \approx E_0 + \frac{\hbar^2\pi q}{ma}, \quad E^{(0)}(k-G) \approx E_0 - \frac{\hbar^2\pi q}{ma}
\]
Substituting:
\[
E_{\pm}(q) \approx E_0 \pm \sqrt{\left(\frac{\hbar^2\pi q}{ma}\right)^2 + V_0^2}
\]

\subsubsection{Wavefunctions and Physical Interpretation}

{\bf Eigenstates at Zone Boundary}\\
At $k = \pi/a$, the eigenstates corresponding to $E_{\pm}$ are:
\begin{align*}
\psi_+(x) &\propto \cos\left(\frac{\pi x}{a}\right) \quad \text{(lower energy)} \\
\psi_-(x) &\propto \sin\left(\frac{\pi x}{a}\right) \quad \text{(higher energy)}
\end{align*}

\noindent{\bf Charge Density Distribution}\\
\begin{itemize}
\item \textbf{Lower band ($\psi_+$)}: Charge density $\propto \cos^2(\pi x/a)$ peaks at $x = 0, a, 2a, \dots$ (ion positions). Electrons sit in potential minima $\Rightarrow$ lower energy.
\item \textbf{Upper band ($\psi_-$)}: Charge density $\propto \sin^2(\pi x/a)$ peaks at $x = a/2, 3a/2, \dots$ (between ions). Electrons sit in potential maxima $\Rightarrow$ higher energy.
\end{itemize}

\noindent{\bf Group Velocity}
The group velocity is:
\[
v_g = \frac{1}{\hbar}\frac{dE}{dk}
\]
At the zone boundary ($k = \pi/a$), $\frac{dE}{dk} = 0$, so $v_g = 0$. This reflects complete Bragg reflection: right- and left-going waves have equal amplitude, forming a standing wave with no net propagation.

\subsubsection{Connection to Bragg Condition}

The standard Bragg condition for normal incidence is:
\[
2a = n\lambda
\]
For $n=1$, $\lambda = 2a$, giving wavevector:
\[
k = \frac{2\pi}{\lambda} = \frac{\pi}{a}
\]
This is exactly the Brillouin zone boundary. The quantum mechanical condition $2k \cdot G = G^2$ reduces in 1D to $k = \pi/a$.

\subsubsection{Summary}

\begin{itemize}
\item Starting from free electrons $E^{(0)}(k) = \frac{\hbar^2 k^2}{2m}$, a weak periodic potential $V(x)$ couples states differing by reciprocal lattice vectors.
\item At Brillouin zone boundaries ($k = \pi/a$), free electron states $|k\rangle$ and $|k-G\rangle$ become degenerate.
\item The periodic potential lifts this degeneracy, opening an energy gap $E_g = 2|V_0|$.
\item The resulting band structure has two branches $E_{\pm}(k)$ with a forbidden energy range.
\item Wavefunctions at zone boundaries become standing waves due to Bragg reflection.
\end{itemize}

This derivation in the \textbf{nearly-free electron model} demonstrates how continuous free electron parabolas evolve into the band structure of crystals, providing the foundation for understanding metals, semiconductors, and insulators.


\subsection{Tight-Binding Model}

For strongly bound electrons, we start from localized atomic orbitals and consider hopping between neighbors.

\subsubsection{Simple 1D Chain}
For a chain of identical atoms with spacing $a$, each with an atomic orbital $|\phi_n\rangle$ centered at site $n$:
\begin{align}
|\psi_k\rangle = \frac{1}{\sqrt{N}} \sum_n e^{ikna} |\phi_n\rangle
\end{align}
The energy dispersion is:
\begin{align}
E(k) = \epsilon_0 - 2t \cos(ka)
\end{align}
where:
\begin{itemize}
    \item $\epsilon_0 = \langle \phi_n|H|\phi_n\rangle$: on-site energy
    \item $t = -\langle \phi_n|H|\phi_{n+1}\rangle > 0$: hopping integral
\end{itemize}
The bandwidth is $4t$, reflecting the degree of electron delocalization.

\subsubsection{Physical Interpretation}
\begin{itemize}
    \item At $k=0$: All phases align constructively $\rightarrow$ bonding state (lowest energy)
    \item At $k=\pi/a$: Alternating phases $\rightarrow$ antibonding state (highest energy)
    \item For a half-filled band (1 electron per atom): $E_F$ lies within the band $\rightarrow$ metal
    \item For a filled band (2 electrons per atom): $E_F$ in the gap $\rightarrow$ insulator
\end{itemize}
\subsection{General Band Structure Properties}

\subsubsection{Number of States per Band}
For a crystal with $N$ primitive cells:
\begin{align}
\text{Number of } k\text{-states in a band} = N
\text{Electron capacity of a band} = 2N \quad \text{(including spin)}
\end{align}
\subsubsection{Effective Mass}
Near band extrema, the dispersion can be approximated as:
\begin{align}
E(\kvec) \approx E_0 + \frac{\hbar^2}{2}\sum_{ij} \left(m^{*-1}\right)_{ij} k_i k_j
\end{align}
where the effective mass tensor is:
\begin{align}
\left(m^{*-1}\right)_{ij} = \frac{1}{\hbar^2} \frac{\partial^2 E}{\partial k_i \partial k_j}
\end{align}
Unlike free electrons ($m^* = m$), in crystals $m^*$ can be:
\begin{itemize}
    \item Positive: electron-like carriers
    \item Negative: hole-like carriers (near valence band maximum)
    \item Anisotropic: different along different crystal directions
\end{itemize}

\subsubsection{Group Velocity and Current}
The group velocity of a Bloch electron is:
\begin{align}
\boxed{\vvec_n(\kvec) = \frac{1}{\hbar} \nabla_{\kvec} E_n(\kvec)}
\end{align}
This fundamental relation explains why partially filled bands conduct (available states at $E_F$) while filled bands don't (velocity contributions cancel by symmetry).

\subsubsection{Direct vs. Indirect Gaps}
\begin{itemize}
    \item \textbf{Direct gap}: Valence band maximum and conduction band minimum at same $\kvec$ (e.g., GaAs)
    \item \textbf{Indirect gap}: Extrema at different $\kvec$ (e.g., Si, Ge)
\end{itemize}
This distinction is crucial for optical absorption and emission processes.

\subsection{From Band Theory to Modern Transport}

The success of band theory in explaining metals, semiconductors, and insulators was revolutionary. However, several phenomena remained unexplained:
\begin{enumerate}
    \item \textbf{Anomalous Hall effect} in ferromagnets
    \item \textbf{Integer quantum Hall effect} (discovered 1980)
    \item \textbf{Quantization of polarization} in insulators
    \item \textbf{Topological insulators} (discovered 2005s)
\end{enumerate}

These required going beyond the simple $\vvec = \frac{1}{\hbar}\nabla_{\kvec}E$ picture to include the \textbf{geometric properties} of Bloch wavefunctions, leading to the Berry phase formalism discussed next.

\section{Transition to Modern Theory}
The limitations above motivated the development of \textbf{band theory} (Bloch, 1928) and later the incorporation of \textbf{geometric phases} (Berry, 1984). The key insights:
\begin{enumerate}
    \item \textbf{Crystalline potential matters}: $H = \frac{p^2}{2m} + V(\rvec)$ with $V(\rvec+\mathbf{R}) = V(\rvec)$.
    \item \textbf{Bloch theorem}: Wavefunctions $\psi_{n\kvec}(\rvec) = e^{i\kvec\cdot\rvec}u_{n\kvec}(\rvec)$.
    \item \textbf{Velocity is modified}: $\vvec_n(\kvec) = \frac{1}{\hbar}\nabla_{\kvec}E_n(\kvec) + \text{geometric terms}$.
    \item \textbf{Geometry matters}: Berry curvature $\Omega_n(\kvec)$ affects electron dynamics.
\end{enumerate}

The following sections develop the full quantum mechanical theory that overcomes these limitations.
\section{Current derivation in quantum approach}
In quantum mechanics, the average position is defined as
\begin{align*}
<\rvec(t)> = <\Psi(t)|\hat\rvec|\Psi(t)>
\end{align*}
The average velocity is its time derivative 
\begin{align*}
<\dot \rvec(t)> = \frac{d}{dt}<\Psi(t)|\hat \rvec|\Psi(t)>
\end{align*}
We first need to derive $\Psi(t)$.\\
\subsection{Wavefunction under adiabatic approximation }
Based on the adiabatic theorem and Bloch's theorem,  the wave function can be written as
\begin{align*}
|\Psi_n(t)\rangle = e^{i\theta_n(t)} | u_n(\rvec(t)) \rangle
\end{align*}
Its time derivative is
\begin{align*}
\frac{d}{dt}|\Psi_n(t)\rangle = i \dot \theta_n e^{i\theta_n(t)}|u_n(\rvec(t)) \rangle+ e^{i \theta_n} \frac{d}{dt}|u_n(\rvec(t)) \rangle
\end{align*}
Substituting $|\Psi_n(t)$ into Schrodinger equation:
\begin{align*}
i \hbar \frac{d}{dt}|\Psi_n(t) \rangle = H(\rvec(t)) |\Psi_n(t) \rangle
\end{align*}
yields:
\begin{align*}
i\hbar[i\dot \theta_n |u_n(\rvec(t)) \rangle + \frac{d}{dt}|u_n(\rvec(t)) \rangle]e^{i \theta_n(t)} = H( \rvec(t)) e^{i\theta_n}|u_n(\rvec(t)) \rangle
\end{align*}
Canceling the phase factor $e^{i \theta_n(t)}$ on both sides:
\begin{align*}
-\hbar\dot \theta_n |u_n(\rvec(t)) \rangle + i \hbar \frac{d}{dt}|u_n(\rvec(t)) \rangle = E_n(\rvec(t)) |u_n(\rvec(t))\rangle
\end{align*}
Multiplying $\langle u_n|$ on the left:
\begin{align*}
-\hbar \dot \theta_n + i \hbar \langle u_n(\rvec(t))|\frac{d}{dt}|u_n(\rvec(t)) \rangle = E_n(\rvec(t))
\end{align*}
Thus
\begin{align*}
\dot \theta_n = i \langle u_n|\frac{d}{dt}|u_n \rangle - \frac{1}{\hbar} E_n(\rvec(t))
\end{align*}
Integrating over time gives
\begin{align*}
\theta_n(t) = - \frac{1}{\hbar} \int_0^t E_n(t')dt'+ i \int_0^t \langle u_n(\rvec(t'))|\frac{d}{dt'}u_n(\rvec(t')) \rangle dt'
\end{align*}
Define the Berry phase
\begin{align*}
\gamma_n(t) = i\int_0^t \langle u_n(\rvec(t'))|\frac{d}{dt'}u_n(\rvec(t'))\rangle dt'
\end{align*}
The final result of adiabatic solution is:
\begin{align*}
\boxed{|\Psi_n(t)\rangle = e^{i\gamma_n(t)} \exp\left(-\frac{i}{\hbar}\int_0^t E_n(t') dt'\right) |u_n(\rvec(t))\rangle}
\end{align*}
The state acquires both a dynamical phase from the energy eigenvalue and a geometric Berry phase from the parametric evolution
\subsection{Expectation value of velocity}
The velocity expectation value is:
\begin{align*}
\langle \dot r(t) \rangle & = \frac{d}{dt}<\Psi(t)|\hat r|\Psi(t)>\\
                 & = <\frac{d}{dt}\Psi(t)|\hat r| \Psi(t)> + <\Psi(t)|\frac{d}{dt}\hat r|\Psi(t)> +  <\Psi(t)|\hat r| \frac{d}{dt}\Psi(t)>
\end{align*}
\subsubsection{Second term $<\Psi(t)|\frac{d}{dt}\hat r|\Psi(t)>$ }
1)Based on Heisenburg equation
\begin{align*}
\frac{d}{dt} \hat r = \frac{i}{\hbar}[H, r] = \frac{1}{i\hbar}[\hat r, \hat H]
\end{align*}
Thus:
\begin{align*}
<\Psi(t)|\frac{d}{dt}\hat r|\Psi(t)> = \frac{1}{i\hbar}<\Psi(t)|[\hat r,\hat H]|\Psi(t)>
\end{align*}
2)Schrodinger equation with periodic Hamiltonian.
Consider a Bloch eigenstate
\begin{align*}
|\Psi_{n\kvec}> = e^{i\kvec r} |u_{n\kvec}> 
\end{align*}
Then Schrodinger equation gives:
\begin{align*}
\hat H(k) e^{i\kvec \cdot \rvec} |u_{n \kvec}> = \epsilon_{n \kvec} e^{i\kvec \cdot \rvec}|u_{n\kvec}>
\end{align*}
Multiplying $e^{-ikr}$ on the left hand side,
\begin{align*}
e^{-i\kvec \cdot \rvec}\hat H(\kvec) e^{i\kvec \cdot \rvec} |u_{n\kvec}\rangle = e^{-i\kvec \cdot \rvec}\epsilon_{n \kvec} e^{i \kvec \cdot \rvec}|u_{n \kvec}> = \epsilon_{n\kvec} |u_{n\kvec}>
\end{align*}
Define the cell-periodic Hamiltonian
\begin{align*}
\hat H(\kvec) \equiv e^{-i \kvec \cdot \rvec} \hat H e^{i \kvec \cdot \rvec}
\end{align*}
then we have
\begin{align*}
\hat H(\kvec) |u_{n\kvec}> = \epsilon_n(\kvec) |u_{n\kvec}>
\end{align*}
3)link $[r, \hat H]$ to $\partial_k H(k)$,\\  
Differentiating $\hat H(k)$:
\begin{align*}
\nabla_{\kvec} H(\kvec) &= \nabla_{\kvec} \left[ e^{-i\kvec\cdot\rvec} H e^{i\kvec\cdot\rvec} \right] \\
&= -i e^{-i\kvec\cdot\rvec} [\rvec, H] e^{i\kvec\cdot\rvec}
\end{align*}
Therefore:
\begin{align*}
e^{-i\kvec\cdot\rvec} [\rvec, H] e^{i\kvec\cdot\rvec} = i \nabla_{\kvec} H(\kvec)
\end{align*}
4)Evaluating the expectation value of $ [r, \hat H] $:
\begin{align*}
\langle \Psi_{n\kvec} | [\hat{r}, H] | \Psi_{n\kvec} \rangle 
&= \langle u_{n\kvec} | e^{-i\kvec\cdot\rvec} [\hat{r}, H] e^{i\kvec\cdot\rvec} | u_{n\kvec} \rangle \\
&= \langle u_{n\kvec} | i \nabla_{\kvec} H(\kvec) | u_{n\kvec} \rangle
\end{align*}
According to Hellmann–Feynman theorem
\begin{align*}
\langle u_{n\kvec}| i \nabla_{\kvec} \hat H(k)|u_{n\kvec}> = i \nabla_{\kvec} \epsilon(\kvec)
\end{align*}
Combining 1) through 4)
\begin{align*}
\boxed{\langle \Psi(t) | \partial_t \hat{r} | \Psi(t) \rangle = \frac{1}{\hbar} \nabla_{\kvec} \epsilon_n(\kvec)}
\end{align*}
This is the  \textbf{Bloch velocity} term. It shows the current contribution coming directly from band structure, and it allows us to explain the conductivity of metal, which will be explained in later sections.\\
\subsubsection{First and Third terms: Geometric Contribution}
The 3rd term is the complex conjugate of the 1st term. \
{(1)Derivative of wavefunction\\}
The time-dependent wavefunction is the Bloch form with a Berry phase.
\begin{align*}
|\Psi> = e^{i\theta(t)}e^{ik(t)r}|u(k(t))>
\end{align*}  
\begin{align*}
 | \partial_t\Psi> & =[i\dot \theta(t)  + (i\dot k(t)r)] e^{i\theta}e^{ik(t)r}|u(k(t))>+ e^{i\theta}e^{ik(t)r} \dot k(t)|\partial_ku(k(t))> \\
<\partial_t \Psi | & = <u(k(t))|e^{-ik(t)r}e^{-i\theta}[-i\dot \theta(t) + (-ik(t)r)] +<\partial_k u(k(t))|e^{-ik(t)r}e^{-i\theta}\dot k(t) 
\end{align*}
So
\begin{align*}
<\Psi|r|\partial_t\Psi> &= <u(k(t))|e^{-ik(t)r}e^{-i\theta(t)} i r \dot \theta(t) e^{i\theta(t)} e^{ik(t)r}|u(k(t))>  \\
                                &+ <u(k(t))|e^{-ik(t)r}e^{-i\theta(t)} i r k(t)r e^{i\theta(t)} e^{ik(t)r}|u(k(t))>\\
                                &+ <u(k(t))|e^{-ik(t)r}e^{-i\theta(t)} r \dot k(t) e^{i\theta(t)} e^{ik(t)r}|\partial_k u(k(t))>\\
<\partial_t \Psi|r|\Psi> & =  <u(k(t))|e^{-ik(t)r}e^{-i\theta(t)}(-ir  \dot \theta(t)) e^{i\theta(t)} e^{ik(t)r}|u(k(t))> \\
                                 &  + <u(k(t))|e^{-ik(t)r}e^{-i\theta(t)} (-i r k(t)r) e^{i\theta(t)} e^{ik(t)r}|u(k(t))> \\
                                 & + <\partial_k u(k(t))|e^{-ik(t)r}e^{-i\theta(t)} r \dot k(t) e^{i\theta(t)} e^{ik(t)r}|u(k(t))>\\
<\Psi|r|\partial_t\Psi> + <\partial_t \Psi|r|\Psi>
& =<u(k(t))|e^{-ik(t)r}e^{-i\theta(t)}( i r \dot \theta(t) -ir  \dot \theta(t)) e^{i\theta(t)} e^{ik(t)r}|u(k(t))>\\
& +  <u(k(t))|e^{-ik(t)r}e^{-i\theta(t)} (i r k(t)r -i r k(t)r)  e^{i\theta(t)} e^{ik(t)r}|u(k(t))>\\
& + <u(k(t))|e^{-ik(t)r}e^{-i\theta(t)} r \dot k(t) e^{i\theta(t)} e^{ik(t)r}|\partial_k u(k(t))>\\
& + <\partial_k u(k(t))|e^{-ik(t)r}e^{-i\theta(t)} r \dot k(t) e^{i\theta(t)} e^{ik(t)r}|u(k(t))> \\
& = <u(k(t))|e^{-ik(t)r}e^{-i\theta(t)} r \dot k(t) e^{i\theta(t)} e^{ik(t)r}|\partial_k u(k(t))>\\
 & + <\partial_k u(k(t))|e^{-ik(t)r}e^{-i\theta(t)} r \dot k(t) e^{i\theta(t)} e^{ik(t)r}|u(k(t))>\\
\end{align*}
Define $S_{geom}$
\begin{align*}
S_{geom} = <u(k(t))|e^{-ik(t)r}e^{-i\theta(t)} r \dot k(t) e^{i\theta(t)} e^{ik(t)r}|\partial_k u(k(t))> + c.c
\end{align*}
Based on an identity
\begin{align*}
<u_k|e^{-ikr} \hat r e^{ikr}| u_k> = i \partial_k + <u_k| i\partial_k u_k>
%<u_k​%|%e^{−ikr} %\hat r e^{ikr}|u_k​>%=i\partial k​+<u_k​∣i\partial_k ​u_k​>
\end{align*}
The component of $S_{geom,i}$ is
\begin{align*}
S_{geom, \alpha} = \dot k_{\beta} [\partial_{k_{\beta}}(i<u|\partial_{k_{\alpha}}u>) - \partial_k{k_\alpha}(i<\partial_{k_{\beta}}u|u> )]
= \dot k_{\beta} (\partial_{k_{\beta}} A_{\alpha} - \partial_{k_{\alpha}}A_{\beta})
\end{align*}
Define Berry connection:
\begin{align*}
A_{\alpha}(k) = i<u|\partial_{k_{\alpha}}u>
\end{align*}
The corresponding Berry curvature is
\begin{align*}
\Omega_{\gamma} = (\nabla_k \times A)_{\gamma}
\end{align*}
\begin{align*}
S_{geom} = - \dot k \times \Omega(k)
\end{align*}
When the electric field E is present
\begin{align*}
\hbar \dot k = -e E
\end{align*}
The expectation value of the velocity
\begin{align*}
<v> = \frac{1}{\hbar}\nabla_k \epsilon_n(k) + \frac{e}{\hbar} E \times \Omega_n(k)
       = v_{Bloch} + v_{anomalous}
\end{align*}
So the contribution of electron velocity comes from two terms: one is Bloch velocity $v_{Bloch} =  \frac{1}{\hbar}\nabla_k \epsilon_n(k)$ which can explan the conductivity of metal, the other one is anomalous velocity $\frac{e}{\hbar} E \times \Omega_n(k)$, which explains the conductivity for Hall effect, topological insulators, etc.\\
\subsection{Current expression}
\begin{align*}
j = -e \sum_n \int_{BZ}\frac{d^3k}{(2\pi)^3} f({\bf k}) v_n({\bf k})
\end{align*}
where $f({\bf k})$ is Fermi-Dirac distribution and summation over n means summation over all occupied bands.
We can decompose the current into two terms,\\
Bloch current:\\
\begin{align*}
j_{Bloch} = -e \sum_n \int_{BZ}\frac{d^3k}{(2\pi)^3} f({\bf k}) v_{Bloch}({\bf k})
\end{align*}
and anomalous current:\\
\begin{align*}
j_{anomalous} = -e \sum_n \int_{BZ}\frac{d^3k}{(2\pi)^3} f({\bf k}) v_{Bloch}({\bf k})
\end{align*}
\subsubsection{Bloch Current}
1) Electric field is not present. We have symmetrical engery band $E_n({\bf k}) = E_n(-{\bf k})$, so $\nabla_k E_n({\bf k})$ is odd function. $\nabla_{\bf k} E_n(-{\bf k}) = - \nabla_{\bf k} E_n({\bf k})$. Also $f({\bf k})$ is symmetrical with respect to ${\bf k}$, $f({\bf k}) = f(E({\bf k}))=f(e({\bf -k}))$. So $ f({\bf k}) v_{Bloch}({\bf k})$ is an odd function with respect to k. As a consequence the integral over the Brillion zone is zero
\begin{align*}
j_{Bloch} = -e \sum_n \int_{BZ}\frac{d^3k}{(2\pi)^3} f({\bf k}) v_{Bloch}({\bf k})  = 0
\end{align*}
So the net current without eletric field.\\
2) Electric field is present\\
a) If all the bands are fully occupied, the distribution function does not change when electric field is applied because all the states are occupied. Therefore the net current is still zero.\\
b) When some bands are partially filled, the distribution function becomes
\begin{align*}
f({\bf k}) = f_0({\bf k}) + \delta f({\bf k})
\end{align*}
Where $f_0({\bf k})$ is the distribution function without electric field.
Based on Boltzman relaxation time approximation
\begin{align*}
\delta f({\bf k}) = \frac{e \tau}{\hbar} {\hat E} \cdot \nabla_k f_0(k)
\end{align*}
\begin{align*}
j_{Bloch} & = -e \sum_n \int_{BZ}\frac{d^3k}{(2\pi)^3} \delta f({\bf k}) v_{Bloch}(k) \\
& = e^2 \tau \sum_n \int_{BZ}\frac{d^3k}{(2\pi)^3}  (E \cdot \nabla_k f_0) v_n(k) \\
& =e^2 \tau \sum_n \int_{BZ}\frac{d^3k}{(2\pi)^3}  (E \cdot \frac{\partial f}{\partial \epsilon} \frac{\partial \epsilon}{\partial k} ) v_n(k)
\end{align*} 
Define 
\begin{align*}
\sigma_{ij} = e^2 \tau \sum_n \int \frac{d^3k}{(2\pi)^3} v_i(k) v_j(k) (-\frac{d f_0}{d \epsilon})
\end{align*}
We have
\begin{align*}
{\bf j} = \sigma {\bf E}
\end{align*}
\subsubsection{Anomalous Current}
\begin{align*}
v_{anomalous} = - \frac{e}{\hbar} E \times \Omega_n{\bf k}
\end{align*}
Imagine we apply an electric field in x direction, and we would like to know the current in y
\begin{align*}
j_y = -e \sum_n \int \frac{d^2k}{(2\pi)^2}f_n(\bf k)[-\frac{e}{\hbar}(E \times \Omega_n({\bf k}))_y]
\end{align*}
For 2D system, $E = (E_x, E_y)$, and $\Omega$ is in z direction. So
\begin{align*}
(E \times \Omega_n)_y = E_z \Omega_n^x - E_x \Omega_n^z
\end{align*}
Because $E_z = 0$, and $\Omega_n^x=0$
\begin{align*}
(E \times \Omega_n)_y = - E_x \Omega_n^z({\bf k})
\end{align*}
Plugging this into the formula of $j_y$
\begin{align*}
j_y = [-\frac{e^2}{\hbar}\sum_n \int_{BZ} \frac{d^2 k}{(2\pi)^2}f_n({\bf k})\Omega_n({\bf k})]E_x
\end{align*}
The conductivity is
\begin{align*}
\sigma_{xy} =-\frac{e^2}{\hbar}\sum_n \int_{BZ} \frac{d^2k}{(2\pi)^2}f_n({\bf k})\Omega_n(k_x, k_y)
\end{align*}
In the insulator, all valence bands are occupied, $f_n({\bf k}) = 1$.
\begin{align*}
\sigma_{xy} =-\frac{e^2}{\hbar}\sum_n \int_{BZ} \frac{d^2k}{(2\pi)^2} \Omega_n(k_x, k_y)
\end{align*}
Define Chern number for the nth band
\begin{align*}
C_n = \frac{1}{2\pi} \int_{BZ} d^2 k \Omega_n({\bf k})
\end{align*}
For all the occupied band, the total Chern number is
\begin{align*}
C = \sum_n C_n
\end{align*}
Plugging in C into $\sigma_{xy}$
\begin{align*}
\sigma_{xy} = - \frac{e^2}{\hbar} \sum_n (2 \pi C_n) \frac{1}{(2\pi)^2} = -\frac{e^2}{2\pi\hbar}\sum_n C_n = -\frac{e^2}{2\pi\hbar} C = -\frac{e^2}{h}C
\end{align*}
\subsection{The Chern number is an integer}
The Brillouin zone (BZ) is a torus $ T^2$, which has no boundary. To apply Stokes' theorem, we divide it into two overlapping regions where the Bloch wavefunctions \( |u_{n\mathbf{k}}\rangle \) can be defined with smooth, single-valued gauges.
\begin{itemize}
    \item Let \( R_I \) be the region \( 0 \leq k_y \leq \pi \).
    \item Let \( R_{II} \) be the region \( \pi \leq k_y \leq 2\pi \).
\end{itemize}

These regions share the boundaries:
\begin{itemize}
    \item \( C_1 \): the line \( k_y = \pi \) (traversed from \( k_x = 0 \) to \( 2\pi \)).
    \item \( C_2 \): the line \( k_y = 0 \) and \( k_y = 2\pi \), identified as the same line on the torus (traversed from \( k_x = 2\pi \) to \( 0 \)).
\end{itemize}
We choose smooth gauges for the wavefunctions in each region:
\begin{itemize}
    \item In \( R_I \): \( |u^I_{n\mathbf{k}}\rangle \) with Berry connection \( \mathbf{A}^I_n(\mathbf{k}) = i \langle u^I_{n\mathbf{k}} | \nabla_{\mathbf{k}} u^I_{n\mathbf{k}} \rangle \).
    \item In \( R_{II} \): \( |u^{II}_{n\mathbf{k}}\rangle \) with Berry connection \( \mathbf{A}^{II}_n(\mathbf{k}) \).
\end{itemize}
On the boundaries \( C_1 \) and \( C_2 \), the wavefunctions in different gauges are related by a \( U(1) \) gauge transformation:
\[
|u^{II}_{n\mathbf{k}}\rangle = e^{i\chi_1(\mathbf{k})} |u^I_{n\mathbf{k}}\rangle \quad \text{for } \mathbf{k} \in C_1
\]
\[
|u^{II}_{n\mathbf{k}}\rangle = e^{i\chi_2(\mathbf{k})} |u^I_{n\mathbf{k}}\rangle \quad \text{for } \mathbf{k} \in C_2
\]
Under this transformation, the Berry connections are related by:
\[
\mathbf{A}^{II}_n(\mathbf{k}) = \mathbf{A}^I_n(\mathbf{k}) - \nabla_{\mathbf{k}} \chi_{1,2}(\mathbf{k})
\]

We express the Chern number as an integral of the Berry curvature \( \Omega_n = (\nabla_{\mathbf{k}} \times \mathbf{A}_n)_z \). Using Stokes' theorem on each region:
\[
C_n = \frac{1}{2\pi} \iint_{R_I} \Omega_n \, d^2k + \frac{1}{2\pi} \iint_{R_{II}} \Omega_n \, d^2k = \frac{1}{2\pi} \oint_{\partial R_I} \mathbf{A}^I_n \cdot d\mathbf{k} + \frac{1}{2\pi} \oint_{\partial R_{II}} \mathbf{A}^{II}_n \cdot d\mathbf{k}
\]
The boundaries are traversed as follows:
\begin{itemize}
    \item \( \partial R_I \): along \( C_1 \) in the \( +\hat{k}_x \) direction, then \( C_2 \) in the \( -\hat{k}_x \) direction.
    \item \( \partial R_{II} \): along \( C_1 \) in the \( -\hat{k}_x \) direction, then \( C_2 \) in the \( +\hat{k}_x \) direction.
\end{itemize}
Adding the contributions carefully, the total line integral becomes:
\[
C_n = \frac{1}{2\pi} \left[ \int_{C_1} (\mathbf{A}^I_n - \mathbf{A}^{II}_n) \cdot d\mathbf{k} + \int_{C_2} (\mathbf{A}^{II}_n - \mathbf{A}^I_n) \cdot d\mathbf{k} \right]
\]
Using the gauge relation \( \mathbf{A}^{II}_n = \mathbf{A}^I_n - \nabla_{\mathbf{k}} \chi \), we find:
\[
\mathbf{A}^I_n - \mathbf{A}^{II}_n = \nabla_{\mathbf{k}} \chi_1 \quad \text{on } C_1
\]
\[
\mathbf{A}^{II}_n - \mathbf{A}^I_n = -\nabla_{\mathbf{k}} \chi_2 \quad \text{on } C_2
\]
Substituting these in:
\[
C_n = \frac{1}{2\pi} \left[ \int_{C_1} \nabla_{\mathbf{k}} \chi_1 \cdot d\mathbf{k} - \int_{C_2} \nabla_{\mathbf{k}} \chi_2 \cdot d\mathbf{k} \right]
\]
Each integral is now a total derivative along a closed path:
\[
\int_{C_1} \nabla_{\mathbf{k}} \chi_1 \cdot d\mathbf{k} = \chi_1(2\pi, \pi) - \chi_1(0, \pi)
\]
\[
\int_{C_2} \nabla_{\mathbf{k}} \chi_2 \cdot d\mathbf{k} = \chi_2(2\pi, 0) - \chi_2(0, 0)
\]
Therefore,
\[
C_n = \frac{1}{2\pi} \left[ (\chi_1(2\pi, \pi) - \chi_1(0, \pi)) - (\chi_2(2\pi, 0) - \chi_2(0, 0)) \right]
\]
The Bloch wavefunction must be single-valued on the torus. This imposes a quantization condition on the gauge transformation functions after a closed loop:
\[
\chi_1(2\pi, \pi) - \chi_2(0, 0) = 2\pi m_1, \quad \chi_2(2\pi, 0) - \chi_1(0, \pi) = 2\pi m_2
\]
for some integers \( m_1, m_2 \in \mathbb{Z} \).

Using these relations, the expression in the bracket simplifies to:
\[
[\cdots] = (\chi_1(2\pi, \pi) - \chi_1(0, \pi)) - (\chi_2(2\pi, 0) - \chi_2(0, 0)) = 2\pi m_1 - 2\pi m_2
\]
Substituting back, the factors of \( 2\pi \) cancel:
\[
C_n = \frac{1}{2\pi} (2\pi m_1 - 2\pi m_2) = m_1 - m_2
\]
Since \( m_1 \) and \( m_2 \) are integers, their difference \( C_n \) is also an integer.
\[
\boxed{C_n \in \mathbb{Z}}
\]
So
\begin{align*}
	\sigma_{xy} = -\nu \frac{e^2}{h}
\end{align*}
Where $\nu$ is an integer.

\section{Classical Theory of Dielectric Polarization}

Before delving into the modern quantum theory, it is instructive to review the classical understanding of polarization and its limitations, particularly for crystalline insulators.

\subsection{Classical Definition of Polarization}

In classical electromagnetism, the macroscopic polarization $\mathbf{P}$ is defined as the electric dipole moment per unit volume:

\begin{align}
\mathbf{P} = \frac{1}{V} \sum_i \mathbf{p}_i
\end{align}

where $\mathbf{p}_i = q_i \mathbf{r}_i$ are the individual dipole moments within volume $V$. For a continuous charge distribution $\rho(\mathbf{r})$:

\begin{align}
\mathbf{P} = \frac{1}{V} \int_V \mathbf{r} \rho(\mathbf{r}) \, d^3r
\end{align}

\subsection{Macroscopic Maxwell Equations and Bound Charges}

In macroscopic electrodynamics, the electric displacement field $\mathbf{D}$ is defined as:

\begin{align}
\mathbf{D} = \epsilon_0 \mathbf{E} + \mathbf{P}
\end{align}

The polarization is related to bound charge densities by:

\begin{align}
\rho_b &= -\nabla \cdot \mathbf{P} \quad \text{(volume bound charge density)} \\
\sigma_b &= \mathbf{P} \cdot \hat{\mathbf{n}} \quad \text{(surface bound charge density)}
\end{align}

where $\hat{\mathbf{n}}$ is the unit normal to the surface. This leads to the key classical result: \textbf{A uniformly polarized medium is completely characterized by bound charges on its surface}.

\subsection{The Surface Charge Interpretation}

For a uniformly polarized material ($\nabla \cdot \mathbf{P} = 0$ internally), the polarization manifests only as surface charges. This suggests an operational definition:

\begin{align}
\boxed{\text{Polarization } \mathbf{P} \text{ can be measured by the surface charge density } \sigma_b = \mathbf{P} \cdot \hat{\mathbf{n}}}
\end{align}

This interpretation works well for finite samples and forms the basis of textbook explanations of dielectrics.

\subsection{Fundamental Problems with the Classical Approach}

Despite its apparent simplicity, the classical definition encounters severe conceptual difficulties when applied to \textbf{periodic crystalline insulators}.

\subsubsection{Problem 1: Position Operator Ambiguity}

For a periodic crystal, the expectation value $\langle \psi | \hat{\mathbf{r}} | \psi \rangle$ for a Bloch electron is ill-defined. Consider a Bloch state:
\begin{align}
\psi_{n\mathbf{k}}(\mathbf{r}) = e^{i\mathbf{k}\cdot\mathbf{r}} u_{n\mathbf{k}}(\mathbf{r})
\end{align}
The integral $\int \mathbf{r} |\psi|^2 d^3r$ diverges because $|\psi|^2$ is periodic and non-zero everywhere. Regularization attempts lead to results that depend on the choice of unit cell boundaries.

\subsubsection{Problem 2: Gauge Dependence}

Under a gauge transformation $|\psi\rangle \rightarrow e^{i\phi}|\psi\rangle$, the wavefunction changes but physical observables should remain invariant. However:
\begin{align}
\langle \psi | \hat{\mathbf{r}} | \psi \rangle \rightarrow \langle \psi | \hat{\mathbf{r}} | \psi \rangle + \text{gauge-dependent terms}
\end{align}
This makes the classical definition non-unique for quantum states.

\subsubsection{Problem 3: Surface Termination Ambiguity}

The surface charge $\sigma_b$ depends critically on how the crystal is terminated. Different surface cuts give different $\sigma_b$, yet the \textbf{bulk polarization} should be an intrinsic property. This is the \textbf{surface charge paradox}:
\begin{quote}
How can a bulk property be defined by something (surface charge) that depends on surface details?
\end{quote}

\subsubsection{Problem 4: Quantization of Charge Transport}

Consider adiabatically deforming a crystal so it returns to its original state after time $T$. The total charge transported through any cross-section should be quantized in units of the electron charge $e$. The classical formula gives:
\begin{align}
\Delta Q = A \int_0^T \mathbf{j} \cdot \hat{\mathbf{n}} \, dt = A \Delta P
\end{align}
where $A$ is the cross-sectional area. For consistency, $\Delta P$ must be quantized as $e \times (\text{integer})/A$, but the classical definition gives continuous values.

\subsection{The Modern Perspective Shift}

The resolution, developed in the 1990s by Resta, Vanderbilt, and King-Smith, involves a crucial shift in perspective:

\begin{enumerate}
\item \textbf{Give up absolute polarization}: For an infinite periodic crystal, the absolute polarization $\mathbf{P}$ is \textit{not} a well-defined bulk property.

\item \textbf{Focus on changes}: Only \textbf{changes in polarization} $\Delta \mathbf{P}$ between two states of the same crystal are physically meaningful and measurable.

\item \textbf{Use integrated current}: $\Delta \mathbf{P}$ can be defined through the integrated current during an adiabatic transformation:
\begin{align}
\Delta \mathbf{P} = \int_0^T \mathbf{j}(t) \, dt
\end{align}

\item \textbf{Connect to Berry phase}: This current can be expressed in terms of the Berry phase of Bloch electrons, leading to a bulk formula independent of surface details.
\end{enumerate}

\subsection{A Simple Thought Experiment}

Consider a 1D chain of ions with alternating charges $+e$ and $-e$, spaced by $a/2$:

\begin{align}
\cdots + - + - + - \cdots
\end{align}

Two possible states:
\begin{itemize}
\item \textbf{State A}: Positive ions at positions $na$, negative ions at $(n+\frac{1}{2})a$
\item \textbf{State B}: Shift all electrons right by $\delta$ (small compared to $a$)
\end{itemize}

\textbf{Classical calculation}: Polarization change $\Delta P = \frac{e\delta}{a}$.

\textbf{Paradox}: In an infinite chain, States A and B have \textit{identical bulk charge densities}! The only difference appears at the boundaries. This demonstrates that polarization is inherently a \textbf{surface property} in classical theory, yet we seek a \textbf{bulk definition}.

\subsection{Summary: Why We Need a New Approach}

\begin{itemize}
\item The classical definition $\mathbf{P} = \frac{1}{V}\int \mathbf{r}\rho(\mathbf{r})d^3r$ fails for extended quantum systems.

\item Surface charge measurements are unreliable for defining bulk properties.

\item A proper theory must:
\begin{enumerate}
\item Be gauge-invariant
\item Depend only on bulk wavefunctions
\item Give quantized charge transport
\item Reduce to classical results in appropriate limits
\end{enumerate}
\end{itemize}

These requirements lead naturally to the Berry phase formulation, as developed in the following sections.
\section{Modern Theory of Polarization}
\subsection{Introduction}
The modern theory of polarization, developed by King-Smith, Vanderbilt, and Resta, shows that the \textbf{change} in macroscopic polarization \(\Delta \mathbf{P}\) of an insulator can be expressed as a geometric phase (Berry phase) of the electronic wavefunctions. This note outlines the derivation in one dimension for clarity.



\subsection{Derivation of Current Density}

\textbf{Step 1: General expression for the current operator}
The current density operator for a single electron is derived from the probability current in quantum mechanics. In one dimension:
\[
\hat{j}(x) = -\frac{e}{2m} \left[ \hat{p} \delta(\hat{x}-x) + \delta(\hat{x}-x) \hat{p} \right]
\]
where \( \hat{p} = -i\hbar \partial_x \) is the momentum operator, and \( e \) is the magnitude of the electron charge (so the electron's charge is \( -e \)).

We are interested in the \textbf{macroscopic current density} \( j(t) \), which is the spatial average of \( \hat{j}(x) \). For a state \( |\Psi\rangle \), the average current is:
\[
j(t) = \frac{1}{L} \int_0^L dx \, \langle \Psi | \hat{j}(x) | \Psi \rangle
\]

\textbf{Step 2: Current for a single Bloch state}
Consider first a single Bloch state \( |\psi_{nk}(t)\rangle \). Its contribution to the current density is:
\[
j_{nk}(t) = \frac{1}{L} \int_0^L dx \, \langle \psi_{nk}(t) | \hat{j}(x) | \psi_{nk}(t) \rangle
\]
After substituting the Bloch form and carrying out the integral (details omitted), one obtains the well-known result:
\[
j_{nk}(t) = -\frac{e}{\hbar} \frac{\partial E_{nk}(t)}{\partial k}
\]
This is the \textbf{group velocity} contribution. However, this formula assumes the state is a strict eigenstate of the instantaneous Hamiltonian. In an adiabatic process, the actual time-evolving state is \textbf{not} exactly the instantaneous eigenstate; it acquires an additional phase and possible corrections.

\textbf{Step 3: Adiabatic time evolution and first-order correction}
Under adiabatic evolution, the true time-dependent state \( |\Psi_{nk}(t)\rangle \) that starts in \( |\psi_{nk}(0)\rangle \) at \( t=0 \) is given, to first order in the adiabatic parameter, by:
\[
|\Psi_{nk}(t)\rangle \approx e^{i\gamma_{nk}(t)} e^{-\frac{i}{\hbar}\int_0^t E_{nk}(t') dt'} \left[ |\psi_{nk}(t)\rangle + i\hbar \sum_{m \neq n} \frac{|\psi_{mk}(t)\rangle \langle \psi_{mk}(t) | \partial_t \psi_{nk}(t) \rangle}{E_{nk}(t) - E_{mk}(t)} \right]
\]
where \( \gamma_{nk}(t) \) is the \textbf{Berry phase}:
\[
\gamma_{nk}(t) = i \int_0^t dt' \, \langle \psi_{nk}(t') | \partial_{t'} \psi_{nk}(t') \rangle
\]
The term proportional to \( \sum_{m \neq n} \) is the \textbf{first-order adiabatic correction} due to mixing with other bands.

\subsubsection{Step 4: Current to first order in adiabaticity}
We now compute the current for the true adiabatically evolved state \( |\Psi_{nk}(t)\rangle \), keeping terms up to first order in the time derivative.

The expectation value of the current operator is:
\[
j_{nk}^{\text{true}}(t) = \frac{1}{L} \int_0^L dx \, \langle \Psi_{nk}(t) | \hat{j}(x) | \Psi_{nk}(t) \rangle
\]
Substituting the expanded form of \( |\Psi_{nk}(t)\rangle \) and keeping terms linear in \( \partial_t \), we find (after substantial algebra) two contributions:

\begin{enumerate}
    \item \textbf{Zeroth-order term}: The group velocity term \( -\frac{e}{\hbar} \partial_k E_{nk}(t) \). This is the same as if the state were an exact eigenstate.
    \item \textbf{First-order correction}: Comes from the interband mixing. This correction is essential for capturing the full adiabatic current.
\end{enumerate}

\textbf{Step 5: The final compact form}
Remarkably, the sum of the zeroth-order and first-order corrections can be combined into a single, elegant expression involving only the periodic part \( u_{nk} \) of the Bloch function:

\[
\boxed{j_n(t) = \frac{e}{2\pi} \int_{-\pi/a}^{\pi/a} \frac{dk}{i} \, \langle u_{nk}(t) | \partial_t u_{nk}(t) \rangle}
\]

Here \( j_n(t) \) is the current density contribution from band \( n \), and we have summed over all \( k \)-states in the Brillouin zone (assuming the band is fully occupied, as in an insulator). The total current density is the sum over all occupied bands:
\[
j(t) = \sum_{n}^{\text{occupied}} j_n(t)
\]

\textbf{Step 6: Verification and interpretation}
To check this formula, consider the case where the time dependence comes solely from a uniform electric field \( E \). Then, via the acceleration theorem \( \hbar \dot{k} = -eE \), we have \( \partial_t = \dot{k} \partial_k = -\frac{eE}{\hbar} \partial_k \). Substituting:
\[
j_n = \frac{e}{2\pi} \int dk \, \left( -\frac{eE}{\hbar} \right) \frac{1}{i} \langle u_{nk} | \partial_k u_{nk} \rangle = -\frac{e^2 E}{2\pi\hbar} \int dk \, A_n(k)
\]
where \( A_n(k) = i \langle u_{nk} | \partial_k u_{nk} \rangle \) is the Berry connection. For a full cycle, this leads to the polarization change \( \Delta P_n = \frac{e}{2\pi} \Delta \phi_n \), where \( \phi_n = \oint A_n(k) dk \) is the Zak phase, consistent with the modern theory of polarization.

\textbf{Key Points}
\begin{itemize}
    \item The derivation assumes \textbf{adiabatic evolution}: the Hamiltonian changes slowly compared to the band gap.
    \item The formula \( j_n(t) = \frac{e}{2\pi} \int \frac{dk}{i} \langle u_{nk} | \partial_t u_{nk} \rangle \) captures both the \textbf{intraband} (group velocity) and \textbf{interband} (polarization) contributions to the current in a unified way.
    \item This current is a \textbf{transient} polarization current, not a steady-state transport current. It exists only while the Hamiltonian is changing.
    \item The expression is \textbf{gauge invariant} (modulo integer multiples of \( e \)) because it involves the time derivative of a Berry phase.
\end{itemize}

\subsection{Derivation of Polarization}
\subsubsection{Setup: Adiabatic Evolution}
Consider a one-dimensional insulator of length \(L = Na\), with lattice constant \(a\) and \(N\) unit cells (periodic boundary conditions). The Hamiltonian depends adiabatically on a parameter \(\lambda(t)\) (e.g., atomic positions, external field), with \(t\) parameterizing time or an adiabatic process.

The goal is to compute the \textbf{change in polarization} \(\Delta P\) between an initial state at \(t=0\) and a final state at \(t=T\):
\[
\Delta P = P(T) - P(0)
\]

\subsubsection{Step-by-Step Derivation}

\textbf{Step 1: Polarization Change from Integrated Current}
The fundamental relation between polarization and current density is:
\[
\frac{d P(t)}{dt} = j(t)
\]
where \(j(t)\) is the macroscopic current density. Integrating over the adiabatic process:
\[
\Delta P = \int_0^T j(t) \, dt
\]
Thus, the problem reduces to computing the adiabatic current \(j(t)\).

\textbf{Step 2: Current Density in the Independent Electron Picture}
Consider a one-dimensional crystal with lattice constant \( a \) and length \( L = Na \) (periodic boundary conditions). The time-dependent Hamiltonian \( H(t) \) depends adiabatically on a parameter \(\lambda(t)\), which could represent atomic displacements or a slowly varying external field.

We work within the \textbf{independent electron approximation} and the \textbf{adiabatic approximation}: the system remains in its instantaneous eigenstates as \( H(t) \) changes.

Let \( |\psi_{nk}(t)\rangle \) be the Bloch eigenstate of \( H(t) \) with band index \( n \) and crystal momentum \( k \):
\[
H(t) |\psi_{nk}(t)\rangle = E_{nk}(t) |\psi_{nk}(t)\rangle
\]
The Bloch wavefunction has the form:
\[
\psi_{nk}(x, t) = e^{ikx} u_{nk}(x, t)
\]
where \( u_{nk}(x, t) \) is the periodic part, satisfying \( u_{nk}(x+a, t) = u_{nk}(x, t) \).
For a system of independent electrons, the total current is the sum over all occupied Bloch states. For a single occupied band \(n\) (the multi-band case is a simple summation), the contribution is:
\[
j_n(t) = \frac{e}{2\pi} \int_{-\pi/a}^{\pi/a} \frac{dk}{i} \, \langle u_{nk}(t) | \partial_t u_{nk}(t) \rangle
\]
where \(|u_{nk}(t)\rangle\) is the periodic part of the Bloch wavefunction for band \(n\) and crystal momentum \(k\), evolving under the time-dependent Hamiltonian \(H(\lambda(t))\). The factor \(e\) is the electron charge (with sign convention such that \(-e\) is the electron's charge).

\textbf{Step 3: Express Current in Terms of Berry Connection}
Define the \textbf{Berry connection} in parameter space. In addition to the usual \(k\)-space Berry connection \(A_n^{(k)} = i \langle u_{nk} | \partial_k u_{nk} \rangle\), we have a "time" or "\(\lambda\)" connection:
\[
A_n^{(t)}(k,t) = i \langle u_{nk}(t) | \partial_t u_{nk}(t) \rangle
\]
Then the current becomes:
\[
j_n(t) = \frac{e}{2\pi} \int_{-\pi/a}^{\pi/a} dk \, \left[ -i \langle u_{nk}(t) | \partial_t u_{nk}(t) \rangle \right] = \frac{e}{2\pi} \int_{-\pi/a}^{\pi/a} dk \, A_n^{(t)}(k,t)
\]

\textbf{Step 4: Integrate Current to Get Polarization Change}
Integrate the current over time to get the total polarization change contributed by band \(n\):
\[
\Delta P_n = \int_0^T j_n(t) \, dt = \frac{e}{2\pi} \int_0^T dt \int_{-\pi/a}^{\pi/a} dk \, A_n^{(t)}(k,t)
\]
Interchange the order of integration (Fubini's theorem):
\[
\Delta P_n = \frac{e}{2\pi} \int_{-\pi/a}^{\pi/a} dk \left[ \int_0^T dt \, A_n^{(t)}(k,t) \right]
\]
The term in brackets, \(\int_C A_n^{(t)} dt\), is a line integral of the Berry connection along the time/\(\lambda\) direction for a fixed \(k\).

\textbf{Step 5: Link to Berry Phase via 2D Parameter Space}
Consider the two-dimensional parameter space \((k, t)\) (or \((k, \lambda)\)). Define a generalized Berry connection one-form:
\[
\boldsymbol{\mathcal{A}}_n = A_n^{(k)} dk + A_n^{(t)} dt
\]
Now consider a rectangular loop \(\mathcal{C}\) in this parameter space:
\begin{itemize}
    \item Path 1: At \(t=0\), \(k\) goes from \(-\pi/a\) to \(\pi/a\).
    \item Path 2: At \(k=\pi/a\), \(t\) goes from \(0\) to \(T\).
    \item Path 3: At \(t=T\), \(k\) goes from \(\pi/a\) to \(-\pi/a\).
    \item Path 4: At \(k=-\pi/a\), \(t\) goes from \(T\) to \(0\).
\end{itemize}
The Berry phase \(\gamma_n\) for this loop is:
\[
\gamma_n = \oint_{\mathcal{C}} \boldsymbol{\mathcal{A}}_n
\]
By Stokes' theorem, this equals the integral of the Berry curvature \(\Omega_n^{(k,t)} = \partial_k A_n^{(t)} - \partial_t A_n^{(k)}\) over the enclosed surface.

Evaluating \(\oint_{\mathcal{C}} \boldsymbol{\mathcal{A}}_n\) explicitly along the four paths, one finds that:
\[
\gamma_n = \left[ \phi_n(T) - \phi_n(0) \right] + \text{(terms that cancel or are zero due to periodicity in \(k\))}
\]
where \(\phi_n(t)\) is the \textbf{Zak phase} (the \(k\)-space Berry phase across the Brillouin zone) at a fixed time \(t\):
\[
\phi_n(t) = \int_{-\pi/a}^{\pi/a} dk \, A_n^{(k)}(k,t) = i \int_{-\pi/a}^{\pi/a} dk \, \langle u_{nk}(t) | \partial_k u_{nk}(t) \rangle
\]
Crucially, the time-integral term \(\int dk \int dt \, A_n^{(t)}\) appears as part of \(\gamma_n\).

\textbf{Step 6: Final Result: Polarization Change as Berry Phase Difference}
After carefully tracking the contributions from all four paths of the loop and using the periodicity of the wavefunction in \(k\)-space (\(|u_{n,k+2\pi/a}\rangle = |u_{nk}\rangle\)), one arrives at the central result:

\[
\boxed{\Delta P_n = \frac{e}{2\pi} \left[ \phi_n(T) - \phi_n(0) \right]}
\]

For a multi-band insulator, we sum over all occupied bands:
\[
\boxed{\Delta \mathbf{P} = \frac{e}{2\pi} \sum_{n}^{\text{occ}} \Delta \phi_n \, \hat{\mathbf{G}} }
\]
where in higher dimensions, \(\phi_n\) becomes a vectorial Zak phase integrated over the Brillouin zone, and \(\hat{\mathbf{G}}\) is a reciprocal lattice vector direction. In 1D, it simplifies to the scalar form above.

\subsubsection{Interpretation}
\begin{itemize}
    \item \textbf{Not absolute polarization}: The formula gives only the \textbf{change} \(\Delta P\), not the absolute value of \(P\). This resolves the long-standing ambiguity in defining polarization for a periodic crystal.
    \item \textbf{Geometric origin}: The change is expressed as the difference in a \textbf{geometric phase} (Berry/Zak phase) of the occupied Bloch states. It is a bulk property, independent of surface details.
    \item \textbf{Quantization}: In a cyclic adiabatic process where the Hamiltonian returns to itself, \(\Delta \phi_n\) must be an integer multiple of \(2\pi\), so \(\Delta P\) is quantized in units of \(e\) (times a lattice vector). This manifests in phenomena like charge pumping.
\end{itemize}
\end{document}