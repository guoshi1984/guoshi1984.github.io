\documentclass[a4paper]{article}
\usepackage{amssymb, amsmath}
\usepackage{graphicx}
\begin{document}
Author: Dr. Shi Guo  \hspace{30mm} Email: guoshi1984@hotmail.com\\
\line(1,0){350}\\
\noindent The purpose of this article is providing an review of Hall effect family. Understanding the mechanism of all the physics underlying in the Hall effect family is not an easy task, here we provide a step by step review. The whole article is organized as the following:\\
\noindent 1) We review the standard conductivity theory, from classical level to quantum level.\\
\noindent 2) At quantum level, we derive the current operator which has two contributions. The first term, the gradient of the energy eigenvalue in k space, contributes the conductivity of most metals. The second term, which relates to the Berry curvature, has non trivial contribution when the system breaks time-reversal symmetry.\\
\noindent 3) We show that the second term, which relates to the Berry curvature, has an interesting property. The integral of the Berry curvature over the Brillouin zone summed over all bands is an integer of $2\pi$.\\

\section{Brief Summary of Conductance Theory}\\
%\section{Current Operator}
\noindent In order to describe the conductivity, physicists have developed several models, starting from classical model, semiclassical model to quantum model. These model can be summarized as:\\
\noindent 1) In classical model, the electrons are treated classically, and the movement is governed by the Newton's law and the forces on the electrons are described by electromagnetism . This model is good enough to explain the Ohm's law.\\
\noindent 2) In semiclassical model, we consider the behavior of electron as a wave, and treat the movement of electrons as the propagation of the wavepacket, therefore the velocity of the electron is group velocity of the wave. This model utilize the particle-wave duality, and is capable of explaining the conduction in metal.\\
\noindent 3) In quantum model, the velocity is the expectation value of the velocity operator given a wavefunction. We need to use this theory to derive the Hall conductance, and understand the topological behavior of Hall conductance.\\

\section{Classical Conductance Theory Example: Hall Effect}
We consider the electrons inside conductors. When we apply both an electric field E and a magnetic filed B, the electrons have the equation of motion following the Newtons' law
\begin{align*}
	m \frac{d\bf v}{dt} = -e {\bf E} - e {\bf v} \times {\bf B} - m\frac{\bf v}{\tau}
\end{align*}
The first term in right hand side is the force by the electric field, and the second term is the force by the magnetic field. The third term electron collision by the ions. When collision happens, the momentum of the electron changes to zero within a certain mean free time $\tau$. At the equilibrium states, we have $\frac{d {\bf v}}{d t} = 0$. The velocity satisfies 
\begin{align}
	  \frac{e\tau}{m} {\bf v} \times {\bf B} + {\bf v} = - \frac{e \tau}{m} {\bf E}
\end{align}
As {\bf v} = ($v_x$, $v_y$), so the above equation can be written as
\begin{align*}
	v_x + \frac{e \tau}{m} v_y B = -\frac{e \tau}{m} E_x
\end{align*}
The current density J is related to the velocity by
\begin{align*}
	{\bf J} = - n e {\bf v}
\end{align*}
So 
\begin{align*}
	j_x + \frac{e \tau B} {m} j_y  = \frac{n e^2 \tau}{m} E_x
\end{align*}
We define the conductivity as 
\begin{align*}
	{\bf J} = \sigma {\bf E}
\end{align*}
so $\sigma_{xx} =\frac{n e^2 \tau}{m}$, $\sigma_{xy} =\frac{ne}{B}$.\\  

\section{Hall conductivity of 2D electrons }
{\bf Solution to 2D electron system subject to a magnetic field}\\
A Hamiltonian for 2D electrons in a magnetic field $A = xB\haty$ is\\
\begin{align*}
	H = \frac{1}{2m} (p_x^2 + (p_y + eBx)^2)
\end{align*}
Because this Hamiltonian commutes with $p_y$, so they share the same eigenstates, therefore, we can write the
solution for the Hamiltonian as
\begin{align*}
	\psi_k(x, y) = e^{iky}f_k(x)
\end{align*}
\begin{align*}
	H \psi_k(x, y) = \frac{1}{2m} (p_x^2 +(\hbar k + eBx)^2) \psi_k(x, y) = H_k \psi_k(x, y)
\end{align*}
\begin{align*}
		H_k = \frac{1}{2m} p_x^2 +\frac{m \omega_B^2}{2} (x+ \frac{\hbar k}{eB})^2
\end{align*}
This $H_k$ is the Hamiltonian for a harmonic oscillator in the x direction, with the center displaced form the origin. The solution to $H_k$ is very similar to harmonic oscillator. The energy eigenvalues are
\begin{align*}
	E_n = \hbar \omega_B (n + \frac{1}{2})
\end{align*}
where $\omega_B = \frac{eB}{m}$. And the eigenstate wavefunctions are
\begin{align*}
	\psi_{n,k}(x, y) \propto e^{iky} H_n(x + \frac{\hbar k}{eB})e^{-(x+ \frac{\hbar k}{eB})^2eB/2\hbar}
\end{align*}
{\bf Adding an electric filed for 2D electron system subjected to a magnetic field}\\
\begin{align*}
	H = \frac{1}{2m} (p_x^2 + (p_y + eBx)^2) + eEx
\end{align*}
Its solution is again similar to harmonic oscillator with additional shift
\begin{align*}
	\psi(x, y) = \psi_{n,k}(x+ mE/eB^2, y)	
\end{align*}
and the energies are 
\begin{align*}
	E_{n,k} = \hbar \omega_B (n+ \frac{1}{2}) -eE(\frac{\hbar k}{eB}+\frac{eE}{m\omega_B^2})+\frac{m}{2}\frac{E^2}{B^2}
\end{align*}
Since we get the wavefunction and eigenenergy, there are two ways to find out the current. One way is to use the semiclassical approach. In semiclassical approximation, we use a wave packet to describe the behavior of particles, and the velocity of the particle is the velocity of the wave pack. For given a wavevector k, the velocity is
\begin{align*}
	v_y = \frac{1}{\hbar} \frac{\partial E_{n,k}}{\partial k} = -\frac{E}{B}
\end{align*}
So we surprisingly see add an electric filed in x direction generates the movement in y!
To find out the total current in y direction, we have to know the degeneracy, which means how many electrons are in the state with the momentum k to k + dk. In y direction, electrons are free particle with momentum k confined in a finite size $L_y$. So
\begin{align*}
	\frac{dn}{dk} = \frac{L_y}{2 \pi}
\end{align*}
If the total density of states is n, then the density in a unit length in y direction becomes $\frac{n}{L_y}$, the The total current is
\begin{align*}
	I_y = \frac{n}{L_y} e v_y =\frac{1}{L_y}e \frac{E}{B} \int \frac{dn}{dk}dk = \frac{eE}{2\pi B} \int dk
\end{align*}
The range of k in the above integral is tricky. From the wavefunction, we see the center of the harmonic oscillator in x direction is $x = -k\hbar/eB$, while $ 0 \leq x \leq L_x$, then $-L_x eB/\hbar  \leq k \leq 0$.
\begin{align*}
	I_y = \frac{eE}{2\pi B} \int_{-L_xeB/\hbar}^0 dk = \frac{e^2}{h}EL_x
\end{align*}
Then the current density j is defined as the charge flows through a unit area within a unit of time. Since our system is 2 dimensional, the unit area becomes the unit length in x direction.
\begin{align*}
	j = \frac{I_y}{L_x} = \frac{e^2}{h}E = \sigma E
\end{align*}
where 
\begin{align*}
	\sigma = \frac{e^2}{h}
\end{align*}
The above expression of conductivity is for the electrons in one filled Landau energy level. So if there are n Landau levels filled, the conductivity is
\begin{align*}
	\sigma = n\frac{e^2}{h}
\end{align*}
The second way is purely quantum approach. We need the following steps in order to derive the expression of the current. 
1) In quantum mechanics, we represent every physical quantity using an operator. So the first step is to find out the expression of velocity operator. \\
2) The current is the expectation value of its operator, therefore we need to know the wavefunction. Here we employ the time-dependent perturbation theory in order to know how the wavefunction changes overtime.\\
3) Thirdly, we evaluate the expectation of the current operator then analyze why the conductivity is quantized.\\
\noindent{\bf Derivation of velocity operator}\\
In quantum mechanics, the velocity operator is defined as
\begin{align*}
	v = \frac{\partial H}{\partial p} = \frac{\partial H}{ \hbar \partial k}
\end{align*}
There exist several ways to understand this. First, we can recall the equation of motion in analytical mechanics. Given a Hamiltonian, the equation of motion is
\begin{align*}
	{\bf v} = \dot{{\bf r}} = \frac{\partial H}{\partial {\bf p}}
\end{align*}
Another way is using Heisenburg equation of motion, which is the counterpart of analytical mechanics' equation of motion in quantum mechanics. The velocity operator given by Heisenburg equation of motion is
\begin{align*}
	{\bf v} = \frac{d{\bf r}}{dt} = \frac{i}{\hbar}[H, {\bf r}]
\end{align*}
In momentum space, it becomes
\begin{align*}
	{\bf v}({\bf k}) = e^{-i{\bf k} \cdot {\bf r}} \frac{i}{\hbar}[H, {\bf r}] e^{i{\bf k} \cdot {\bf r}} = \frac{1}{\hbar} \nabla_{\bf k} H({\bf k},t)
\end{align*}
%Let us recall the semi-classical model where the movement of electrons is described as the movement of the wave-packet. And the velocity is given by the group velocity of the wave packet.
%\begin{align*}
%	v_g = \frac{\partial \omega}{ \partial k} = \frac{1}{\hbar} \frac{\partial E}{\partial k}
%\end{align*}
{\bf Wavefunction subject to adiabatic evolution }\\
The wave function is subject to the time-dependent Schrodinger equation.
\begin{align*}
	i \hbar \partial_t |\Psi(t) > = H(t) |\Psi(t)> 
\end{align*}
Suppose when $t= t_0$, the instantaneous eigenstates are $|u_n({\bf k}, t)>$, then
the wavefunction can be written as linear combination of all instantaneous states with coefficients $a_n(t)$ times a time evolution factor.  
\begin{align*}
	|\Psi(t)> = \sum_n exp(\frac{1}{i\hbar} \int_{t_0}^t dt^{'} E_n(t^{'})) a_n(t) |u_n({\bf k}, t)>
\end{align*}
Then we consider adiabatic approximation which means the vector {\bf R}(t) varies with time very slowly and apply the time-dependent perturbation theory. After a few steps we have
\begin{align*}
	|\Psi(t)>= exp(-\frac{i}{\hbar}\int_{t_0}^{t} dt^{'} E_{n}(t^{'}))(|u_n({\bf k})> 
	-i\hbar \sum_{n^{'}\neq n} |u_n^{'}({\bf k})>\frac{<u_n^{'}({\bf k})|\frac{\partial}{\partial t}|u_n({\bf k})>}{E_{n} - E_{n^{'}}})
\end{align*}
The expectation value of the velocity
\begin{align*}
	\bar v({\bf k},t) = \frac{1}{\hbar} \nabla_{\bf k} E_n({\bf k}) 
	- i \sum_{n^{'}\neq n} (<u_n|\frac{\partial H}{\partial {\bf k}}|u_n^{'}>\frac{<u_n^{'}|\frac{\partial}{\partial t}|u_n>}{E_{n} - E_{n^{'}}} - c.c.)
\end{align*}
Using the identity
\begin{align*}
	<u_n|\nabla_{\bf k} H | u_m> = (E_n - E_m) < \nabla_{\bf k} u_n| u_m>
\end{align*}
\begin{align}\label{velocity}
	\bar v({\bf k},t) = \frac{1}{\hbar} \nabla_{\bf k} E_n({\bf k}) 
	- i  (<\frac{\partial u_n}{\partial {\bf k}}| \frac{\partial u_n}{\partial t}> -<\frac{\partial u_n}{\partial t}| \frac{\partial u_n}{\partial {\bf k}}>)
\end{align}
Where the second term is the {\bf Berry phase}.
The current operator is
\begin{align*}
	j = -2e\sum_{all bands}\int_{BZ}\frac{dk}{2 \pi} f(k)v(k)
\end{align*}
The integral is taken over the first Brillouin zone denoted by BZ.
We need to consider several cases to discuss the current.\\
1) If the system preserves time reversal symmetry and space reversal symmetry, then the second term in Eqn. \ref{velocity} vanishes. Here is a quick proof.\\
When the system preserves the time reversal symmetry, we have $u_n(-k) = u_n(k)^{*}$. Let us call the second term in Eqn. \ref{velocity} $v_a(k)$, where subscript a stands for anomalous. Then
\begin{align*}
	v_a(-k) \\
	= & i  (<\frac{\partial u_n(-k)}{\partial {\bf k}}| \frac{\partial u_n(-k)}{\partial t}> -<\frac{\partial u_n(-k)}{\partial t}| \frac{\partial u_n(-k)}{\partial {\bf k}}>) \\
	= & i  (<\frac{\partial u_n(k)^{*}}{\partial {\bf k}}| \frac{\partial u_n(k)^{*}}{\partial t}> -<\frac{\partial u_n(k)^{*})}{\partial t}| \frac{\partial u_n(k)^{*}}{\partial {\bf k}}>) \\
	= & i  (<\frac{\partial u_n(k)}{\partial {\bf t}}| \frac{\partial u_n(k)}{\partial k}> -<\frac{\partial u_n(k)}{\partial k}| \frac{\partial u_n(k)}{\partial {\bf t}}>) \\
    = & -v_a(k)
\end{align*}

In this case, if the bands are fully occupied, it is an insulator. If the band are not fully occupied, it is an conductor. The semiconductor is something between this two where the thermal excitation can promote the electron going from valence band to the conduction band so the not fully occupied conduction band contributes the current.\\
2) If all the bands are filled, but the system breaks time and space reversal symmetry, then first term vanishes, the second term is non-trivial. This is what contributes the current in quantum Hall effect.\\
{\bf Quantized current}\\
When the first term of the velocity vanishes(in the case that all bands are fully occupied), the expression of velocity reduces to
\begin{align}
	\bar v({\bf k},t) = - i  (<\frac{\partial u_n}{\partial {\bf k}}| \frac{\partial u_n}{\partial t}> -<\frac{\partial u_n}{\partial t}| \frac{\partial u_n}{\partial {\bf k}}>)
\end{align}
Using the relationship 
\begin{align*}
	\partial_t = \partial_t {\bf k} \cdot \nabla_{\bf k} 
\end{align*}
Based on the definition of crystal momentum and Newton's law
\begin{align*}
	\partial_t(\hbar k) = -eE
\end{align*}
So
\begin{align*}
	\partial_t = -\frac{e}{\hbar} {\bf E} \cdot \nabla_{({\bf k})}
\end{align*}

\begin{align*}
	v_a({\bf k}) = -\frac{e}{\hbar} {\bf E} \times \Omega^{n}({\bf k})  
\end{align*}
where
\begin{align*}
	\Omega^{n}({\bf k}) = \nabla_{\bf k} \times <u_n({\bf k})|i\nabla_{\bf k}|u_n({\bf k})>
\end{align*}
\begin{align*}
	{\bf v_n}({\bf k}) = - \frac{e}{\hbar} {\bf E} \times \Omega^{n}({\bf k})
\end{align*}
We plug the expression of $v_n$ into the expression of current j, we have
\begin{align*}
	{\bf j} = -2e\sum_{all bands}\int_{BZ}\frac{d{\bf k}}{2 \pi} f({\bf k}){\bf v}_n({\bf k}) 
	 = \frac{e^2}{h} \frac{1}{2 \pi} \sum_n \int_{BZ} d{\bf k} {\bf E} \times \Omega^n({\bf k})
\end{align*}
Therefore the conductivity is
\begin{align*}
	\sigma = \frac{e^2}{h} \frac{1}{2 \pi} \sum_n \int_{BZ} d{\bf k} \Omega^n({\bf k})
\end{align*}
If we consider the case of 2 dimensional electron gas, then 
\begin{align*}
	\sigma = \frac{e^2}{h} \frac{1}{2 \pi} \sum_n \int_{BZ} d{\bf k} \Omega^n(k_x, k_y)
\end{align*}
Since the integral runs over the first Brillouin zone, and
\begin{align*}
	\Omega^n(k_x, k_y) = \Omega^n(k_x+\pi, k_y) = \Omega^n(k_x, k_y + \pi)
\end{align*}
Hence, the first Brillouin zone forms a closed torus. The integral over a closed torus gives an multiple integer of $2\pi$. So
\begin{align*}
	\sigma_H = \nu \frac{e^2}{h}
\end{align*}
Where $\nu$ is an integer.
\end{document}