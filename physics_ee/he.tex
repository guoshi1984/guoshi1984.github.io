\documentclass[a4paper]{article}
\usepackage{amssymb, amsmath}
\usepackage{graphicx}
\begin{document}
\noindent The objective of this article is to provide a comprehensive review of the Hall effect family. Grasping the intricate physics underlying the various manifestations of the Hall effect is no straightforward feat. In this regard, we present a step-by-step analysis. The structure of the entire article is as follows:\\
\noindent 1) We commence by delving into the conventional theory of conductivity, spanning from classical principles to the realm of quantum mechanics.\\
\noindent 2) Upon entering the quantum realm, we proceed to derive the current operator, encompassing two distinct contributions. The initial term corresponds to the gradient of the energy eigenvalue in k-space, impacting the conductivity of most metals. The subsequent term, connected to the Berry curvature, assumes significance in situations where time-reversal symmetry is violated.\\
\noindent 3) It is worth noting that the Berry curvature-associated second term possesses a captivating property. The integration of the Berry curvature across the Brillouin zone, summed over all bands, results in an integer value of $2\pi$.\\
\section{Brief Summary of Conductance Theory}
%\section{Current Operator}
\noindent To elucidate conductivity, physicists have formulated multiple models, spanning from the classical and semiclassical to the quantum approaches. These models can be summarized as follows:\\
\noindent 1) In classical model, the electrons are treated classically, and the movement is governed by the Newton's law and the forces on the electrons are described by electromagnetism . This model is good enough to explain the Ohm's law.\\
\noindent 2) The semiclassical model views electrons as both particles and waves. Electron movement is likened to wavepacket propagation, with the electron's velocity representing the group velocity of the wave. This model leverages particle-wave duality and effectively accounts for conduction in metals.\\
\noindent 3) In the quantum model, velocity is represented by the expectation value of the velocity operator within a given wavefunction. This theoretical approach is essential for deriving Hall conductance and comprehending the topological intricacies of Hall conductance.\\
\section{Classical Conductance Theory Example: Hall Effect}
We consider the electrons inside conductors. When we apply both an electric field E and a magnetic filed B, the electrons have the equation of motion following the Newtons' law
\begin{align*}
	m \frac{d\bf v}{dt} = -e {\bf E} - e {\bf v} \times {\bf B} - m\frac{\bf v}{\tau}
\end{align*}
The first term in right hand side is the force by the electric field, and the second term is the force by the magnetic field. The third term electron collision by the ions. When collision happens, the momentum of the electron changes to zero within a certain mean free time $\tau$. At the equilibrium states, we have $\frac{d {\bf v}}{d t} = 0$. The velocity satisfies 
\begin{align}
	  \frac{e\tau}{m} {\bf v} \times {\bf B} + {\bf v} = - \frac{e \tau}{m} {\bf E}
\end{align}
As {\bf v} = ($v_x$, $v_y$), so the above equation can be written as
\begin{align*}
	v_x + \frac{e \tau}{m} v_y B = -\frac{e \tau}{m} E_x
\end{align*}
The current density J is related to the velocity by
\begin{align*}
	{\bf J} = - n e {\bf v}
\end{align*}
So 
\begin{align*}
	j_x + \frac{e \tau B} {m} j_y  = \frac{n e^2 \tau}{m} E_x
\end{align*}
We define the conductivity as 
\begin{align*}
	{\bf J} = \sigma {\bf E}
\end{align*}
so $\sigma_{xx} =\frac{n e^2 \tau}{m}$, $\sigma_{xy} =\frac{ne}{B}$.\\  

\section{Hall conductivity of 2D electrons }
{\bf Solution to 2D electron system subject to a magnetic field}\\
A Hamiltonian for 2D electrons in a magnetic field $A = xB\hat y$ is\\
\begin{align*}
	H = \frac{1}{2m} (p_x^2 + (p_y + eBx)^2)
\end{align*}
Because this Hamiltonian commutes with $p_y$, so they share the same eigenstates, therefore, we can write the
solution for the Hamiltonian as
\begin{align*}
	\psi_k(x, y) = e^{iky}f_k(x)
\end{align*}
\begin{align*}
	H \psi_k(x, y) = \frac{1}{2m} (p_x^2 +(\hbar k + eBx)^2) \psi_k(x, y) = H_k \psi_k(x, y)
\end{align*}
\begin{align*}
		H_k = \frac{1}{2m} p_x^2 +\frac{m \omega_B^2}{2} (x+ \frac{\hbar k}{eB})^2
\end{align*}
This $H_k$ is the Hamiltonian for a harmonic oscillator in the x direction, with the center displaced form the origin. The solution to $H_k$ is very similar to harmonic oscillator. The energy eigenvalues are
\begin{align*}
	E_n = \hbar \omega_B (n + \frac{1}{2})
\end{align*}
where $\omega_B = \frac{eB}{m}$. And the eigenstate wavefunctions are
\begin{align*}
	\psi_{n,k}(x, y) \propto e^{iky} H_n(x + \frac{\hbar k}{eB})e^{-(x+ \frac{\hbar k}{eB})^2eB/2\hbar}
\end{align*}
{\bf Adding an electric filed for 2D electron system subjected to a magnetic field}\\
\begin{align*}
	H = \frac{1}{2m} (p_x^2 + (p_y + eBx)^2) + eEx
\end{align*}
Its solution is again similar to harmonic oscillator with additional shift
\begin{align*}
	\psi(x, y) = \psi_{n,k}(x+ mE/eB^2, y)	
\end{align*}
and the energies are 
\begin{align*}
	E_{n,k} = \hbar \omega_B (n+ \frac{1}{2}) -eE(\frac{\hbar k}{eB}+\frac{eE}{m\omega_B^2})+\frac{m}{2}\frac{E^2}{B^2}
\end{align*}
Since we get the wavefunction and eigenenergy, there are two ways to find out the current. One way is to use the semiclassical approach. We can calculate group velocity given a wavevector k
\begin{align*}
	v_y = \frac{1}{\hbar} \frac{\partial E_{n,k}}{\partial k} = -\frac{E}{B}
\end{align*}
So we surprisingly see add an electric filed in x direction generates the movement in y!
To find out the total current in y direction, we have to know the degeneracy, which means how many electrons are in the state with the momentum k to k + dk. In y direction, electrons are free particle with momentum k confined in a finite size $L_y$. So
\begin{align*}
	\frac{dn}{dk} = \frac{L_y}{2 \pi}
\end{align*}
The total current is
\begin{align*}
	I_y = e \frac{E}{B} \int \frac{dn}{dk}dk = \frac{eEL_y}{2\pi B} \int k
\end{align*}
The range of k in the above integral is tricky. From the wavefunction, we see the center of the harmonic oscillator in x direction is $x = -k\hbar/eB$, while $ 0 \leq x \leq L_x$, then $-L_x eB/\hbar  \leq k \leq 0$.
\begin{align*}
	I_y = \frac{eEL_y}{2\pi B} \int_{-L_xeB/\hbar}^0 k = \frac{e^2}{h}EA
\end{align*}
\section{Current derivation in quantum approach}
The second way is purely quantum approach. In quantum mechanics, the average position is defined as
\begin{align*}
<r(t)> = <\Psi(t)|\hat r|\Psi(t)>
\end{align*}
and the average velocity is the total derivative of above.
\begin{align*}
<\dot r(t)> = \frac{d}{dt}<\Psi(t)|\hat r|\Psi(t)>
\end{align*}
We first need to derive $\Psi(t)$.\\
{\bf Wavefunction subject to adiabatic approximation }\\
Based on Bloch theorem,  the wave function can be written as
\begin{align*}
|\Psi_n(t)> = e^{i\theta_n(t)} | u_n(R(t)>
\end{align*}
and its derivative is
\begin{align*}
\frac{d}{dt}|\Psi_n(t)> = i \dot \theta_n e^{i\theta_n(t)}|u_n>+ e^{i \theta_n} \frac{d}{dt}|u_n>
\end{align*}
Substituting $|\Psi_n(t)$ into Schrodinger equation, 
\begin{align*}
i \hbar \frac{d}{dt}|\Psi_n(t)> = H(R(t)) |\Psi_n(t)>
\end{align*}
yields
\begin{align*}
i\hbar[i\dot \theta_n |u_n> + \frac{d}{dt}|u_n>]e^{i \theta_n(t)} = H(R(t)) e^{i\theta_n}|u_n>
\end{align*}
Canceling $e^{i \theta_n(t)}$ on both sides:
\begin{align*}
-\hbar\dot \theta_n |u_n> + i \hbar \frac{d}{dt}|u_n> = E_n(R(t)) |u_n>
\end{align*}
Multiplying $|u_n>$ on the left:
\begin{align*}
-\hbar \dot \theta_n + i \hbar <u_n|\frac{d}{dt}|u_n> = E_n(R(t))
\end{align*}
So
\begin{align*}
\dot \theta_n = i <u_n|\frac{d}{dt}|u_n> - \frac{1}{\hbar} E_n(R(t))
\end{align*}
Integrating over time gives
\begin{align*}
\theta_n(t) = - \frac{1}{\hbar} \int_0^t E_n(t')dt'+ i \int_0^t <u_n(t')|\frac{d}{dt'}u_n(t')>dt'
\end{align*}
Define
\begin{align*}
\gamma_n(t) = i\int_0^t <u_n(t')|\frac{d}{dt'}u_n(t')>dt'
\end{align*}
which is the Berry phase.
The final result of adiabatic solution is
\begin{align*}
|\Psi_n(t)> = e^{i \gamma_n(t)} exp(-\frac{i}{\hbar} \int_0^t E_n(t')dt')|u_n(R(t))>
\end{align*}
The final state at time t acquires both a dynamical phase from the energy eighenvalue and a geometric Berry phase from the parametric evolution of the eigenstate.\\
{\bf Expectation value of velocity given the wavefunction}
\begin{align*}
<\dot r(t)> & = \frac{d}{dt}<\Psi(t)|\hat r|\Psi(t)>\\
                 & = <\frac{d}{dt}\Psi(t)|\hat r| \Psi(t)> + <\Psi(t)|\frac{d}{dt}\hat r|\Psi(t)> +  <\Psi(t)|\hat r| \frac{d}{dt}\Psi(t)>
\end{align*}
{\bf Derivation of the 2nd term $<\Psi(t)|\frac{d}{dt}\hat r|\Psi(t)>$ }\\
1)Based on Heisenburg equation
\begin{align*}
\frac{d}{dt} \hat r = \frac{i}{\hbar}[H, r] = \frac{1}{i\hbar}[\hat r, \hat H]
\end{align*}
So
\begin{align*}
<\Psi(t)|\frac{d}{dt}\hat r|\Psi(t)> = \frac{1}{i\hbar}<\Psi(t)|[\hat r,\hat H]|\Psi(t)>
\end{align*}
2)Schrodinger equation with periodic Hamiltonian.
Consider Bloch eigenstate
\begin{align*}
|\Psi_{nk}> = e^{ikr} |u_{nk}> 
\end{align*}
Then Schrodinger equation give the Block eigenstate becomes
\begin{align*}
\hat H(k) e^{ikr} |u_{nk}> = \epsilon_{nk} e^{ikr}|u_{nk}>
\end{align*}
Multiplying $e^{-ikr}$ on the left hand side,
\begin{align*}
e^{-ikr}\hat H(k) e^{ikr} |u_{nk}> = e^{-ikr}\epsilon_{nk} e^{ikr}|u_{nk}> = \epsilon_{nk} |u_n{nk}>
\end{align*}
Define the cell-periodic Hamiltonian
\begin{align*}
\hat H(k) \equiv e^{-ikr} \hat H e^{ikr}
\end{align*}
then we have
\begin{align*}
\hat H(k) |u_{nk}> = \epsilon_n(k) |u_{nk}>
\end{align*}
3)link $[r, \hat H]$ to $\partial_k H(k)$,\\  
Differentiating $\hat H(k)$:
\begin{align*}
\frac{d}{dk}\hat H(k) & = \frac{d}{dk}[e^{-ik\dot r} \hat H e^{ik \dot r}] \\
& = -i[e^{-ik\dot r} r\hat H e^{ik \dot r}] + [e^{-ik\dot r} \hat H  r e^{ik \dot r}] \\
& = -ie^{-ikr} [r, \hat H]e^{ikr}
\end{align*}
4)Evaluating the expectation value of $ [r, \hat H] $,
\begin{align*}
<\Psi_{nk}| [\hat r, \hat H ]| \Psi_{nk}> 
&= <u_{nk}| e^{-ikr}[\hat r, \hat H] e^{ikr}|u_{nk}>\\
& = <u_{nk}| i \nabla_k \hat H(k)|u_{nk}> \\
\end{align*}
According to Hellmann–Feynman theorem
\begin{align*}
<u_{nk}| i \nabla_k \hat H(k)|u_{nk}> = i \nabla_k \hat \epsilon(k)
\end{align*}
Combining 1) through 4)
\begin{align*}
<\Psi(t)|\frac{d}{dt}\hat r|\Psi(t)> = \frac{1}{i\hbar}<\Psi(t)|[\hat r,\hat H]|\Psi(t)> = \frac{1}{\hbar}\nabla_k \epsilon_n(k)
\end{align*}
This shows the current contribution coming directly from band structure, and it allows us to explain the conductivity of metal, which will be explained in later sections.\\
{\bf Derivation of 1st and 3rd term\\}
The 3rd term is the complex conjugate of the 1st term. \
{(1)Derivative of wavefunction\\}
The time-dependent wavefunction is the Bloch form with a Berry phase.
\begin{align*}
|\Psi> = e^{i\theta(t)}e^{ik(t)r}|u(k(t))>
\end{align*}  
\begin{align*}
 | \partial_t\Psi> & =[i\dot \theta(t)  + (i\dot k(t)r)] e^{i\theta}e^{ik(t)r}|u(k(t))>+ e^{i\theta}e^{ik(t)r} \dot k(t)|\partial_ku(k(t))> \\
<\partial_t \Psi | & = <u(k(t))|e^{-ik(t)r}e^{-i\theta}[-i\dot \theta(t) + (-ik(t)r)] +<\partial_k u(k(t))|e^{-ik(t)r}e^{-i\theta}\dot k(t) 
\end{align*}
So
\begin{align*}
<\Psi|r|\partial_t\Psi> &= <u(k(t))|e^{-ik(t)r}e^{-i\theta(t)} i r \dot \theta(t) e^{i\theta(t)} e^{ik(t)r}|u(k(t))>  \\
                                &+ <u(k(t))|e^{-ik(t)r}e^{-i\theta(t)} i r k(t)r e^{i\theta(t)} e^{ik(t)r}|u(k(t))>\\
                                &+ <u(k(t))|e^{-ik(t)r}e^{-i\theta(t)} r \dot k(t) e^{i\theta(t)} e^{ik(t)r}|\partial_k u(k(t))>\\
<\partial_t \Psi|r|\Psi> & =  <u(k(t))|e^{-ik(t)r}e^{-i\theta(t)}(-ir  \dot \theta(t)) e^{i\theta(t)} e^{ik(t)r}|u(k(t))> \\
                                 &  + <u(k(t))|e^{-ik(t)r}e^{-i\theta(t)} (-i r k(t)r) e^{i\theta(t)} e^{ik(t)r}|u(k(t))> \\
                                 & + <\partial_k u(k(t))|e^{-ik(t)r}e^{-i\theta(t)} r \dot k(t) e^{i\theta(t)} e^{ik(t)r}|u(k(t))>\\
<\Psi|r|\partial_t\Psi> + <\partial_t \Psi|r|\Psi>
& =<u(k(t))|e^{-ik(t)r}e^{-i\theta(t)}( i r \dot \theta(t) -ir  \dot \theta(t)) e^{i\theta(t)} e^{ik(t)r}|u(k(t))>\\
& +  <u(k(t))|e^{-ik(t)r}e^{-i\theta(t)} (i r k(t)r -i r k(t)r)  e^{i\theta(t)} e^{ik(t)r}|u(k(t))>\\
& + <u(k(t))|e^{-ik(t)r}e^{-i\theta(t)} r \dot k(t) e^{i\theta(t)} e^{ik(t)r}|\partial_k u(k(t))>\\
& + <\partial_k u(k(t))|e^{-ik(t)r}e^{-i\theta(t)} r \dot k(t) e^{i\theta(t)} e^{ik(t)r}|u(k(t))> \\
& = <u(k(t))|e^{-ik(t)r}e^{-i\theta(t)} r \dot k(t) e^{i\theta(t)} e^{ik(t)r}|\partial_k u(k(t))>\\
 & + <\partial_k u(k(t))|e^{-ik(t)r}e^{-i\theta(t)} r \dot k(t) e^{i\theta(t)} e^{ik(t)r}|u(k(t))>\\
\end{align*}
Define $S_{geom}$
\begin{align*}
S_{geom} = <u(k(t))|e^{-ik(t)r}e^{-i\theta(t)} r \dot k(t) e^{i\theta(t)} e^{ik(t)r}|\partial_k u(k(t))> + c.c
\end{align*}
Based on an identity
\begin{align*}
<u_k|e^{-ikr} \hat r e^{ikr}| u_k> = i \partial_k + <u_k| i\partial_k u_k>
%<u_k​%|%e^{−ikr} %\hat r e^{ikr}|u_k​>%=i\partial k​+<u_k​∣i\partial_k ​u_k​>
\end{align*}
The component of $S_{geom,i}$ is
\begin{align*}
S_{geom, \alpha} = \dot k_{\beta} [\partial_{k_{\beta}}(i<u|\partial_{k_{\alpha}}u>) - \partial_k{k_\alpha}(i<\partial_{k_{\beta}}u|u> )]
= \dot k_{\beta} (\partial_{k_{\beta}} A_{\alpha} - \partial_{k_{\alpha}}A_{\beta})
\end{align*}
Define Berry connection
\begin{align*}
A_{\alpha}(k) = i<u|\partial_{k_{\alpha}}u>
\end{align*}
The corresponding Berry curvature is
\begin{align*}
\Omega_{\gamma} = (\nabla_k \times A)_{\gamma}
\end{align*}
\begin{align*}
S_{geom} = - \dot k \times \Omega(k)
\end{align*}
When the electric field E is present
\begin{align*}
\hbar \dot k = -e E
\end{align*}
The expectation value of the velocity
\begin{align*}
	%\bar v({\bf k},t) = \frac{1}{\hbar} \nabla_{\bf k} E_n({\bf k}) 
	%- i \sum_{n^{'}\neq n} (<u_n|\frac{\partial H}{\partial {\bf k}}|u_n^{'}>\frac{<u_n^{'}|\frac{\partial}{\partial t}|u_n>}{E_{n} - E_{n^{'}}} - c.c.)
<v> = \frac{1}{\hbar}\nabla_k \epsilon_n(k) + \frac{e}{\hbar} E \times \Omega_n(k)
       = v_{Bloch} + v_{anomalous}
\end{align*}
So the contribution of electron velocity comes from two terms: one is Bloch velocity $v_{Bloch} =  \frac{1}{\hbar}\nabla_k \epsilon_n(k)$ which can explan the conductivity of metal, the other one is anomalous velocity $\frac{e}{\hbar} E \times \Omega_n(k)$, which explains the conductivity for Hall effect, topological insulators, etc.\\
{\bf Current expression}
\begin{align*}
j = -e \int_{BZ}\frac{d^3k}{(2\pi)^3} f({\bf k}) v_n({\bf k})
\end{align*}
where $f({\bf k})$ is Fermi-Dirac distribution.
We can decompose the current into two terms,\\
Bloch current:\\
\begin{align*}
j_{Bloch} = -e \int_{BZ}\frac{d^3k}{(2\pi)^3} f({\bf k}) v_{Bloch}({\bf }k)
\end{align*}
and anomalous current:\\
\begin{align*}
j_{anomalous} = -e \int_{BZ}\frac{d^3k}{(2\pi)^3} f({\bf k}) v_{Bloch}({\bf k})
\end{align*}
{\bf Bloch Current}\\
1) Electric field is not present. We have symmetrical engery band $E_n({\bf k}) = E_n(-{\bf k})$, so $\nabla_k E_n({\bf k})$ is odd function. $\nabla_{\bf k} E_n(-{\bf k}) = - \nabla_{\bf k} E_n({\bf k})$. Also $f({\bf k})$ is symmetrical with respect to ${\bf k}$, $f({\bf k}) = f(E({\bf k}))=f(e({\bf -k}))$. So $ f({\bf k}) v_{Bloch}({\bf k})$ is an odd function with respect to k. As a consequence the integral over the Brillion zone is zero
\begin{align*}
j_{Bloch} = -e \int_{BZ}\frac{d^3k}{(2\pi)^3} f({\bf k}) v_{Bloch}({\bf k})  = 0
\end{align*}
So the net current without eletric field.\\
2) Electric field is present\\
a) If all the bands are fully occupied, the distribution function does not change when electric field is applied because all the states are occupied. Therefore the net current is still zero.\\
b) When some bands are partially filled, the distribution function becomes
\begin{align*}
f({\bf k}) = f_0({\bf k}) + \delta f({\bf k})
\end{align*}
Where $f_0({\bf k})$ is the distribution function without electric field.
Based on Boltzman relaxation time approximation
\begin{align*}
\delta f({\bf k}) = -e \tau {\bf \hat E} \cdot \nabla_k f_0(k)
\end{align*}
\begin{align*}
j_{Bloch} & = -e \int_{BZ}\frac{d^3k}{(2\pi)^3} \delta f({\bf k}) v_{Bloch}(k) \\
& = e^2 \tau \int_{BZ}\frac{d^3k}{(2\pi)^3}  (E \cdot \nabla_k f_0) v_n(k) \\
& =e^2 \tau \int_{BZ}\frac{d^3k}{(2\pi)^3}  (E \cdot \frac{\partial f}{\partial \epsilon} \frac{\partial \epsilon}{\partial k} ) v_n(k)
\end{align*} 
define 
\begin{align*}
\sigma_{ij} = e^2 \tau \hbar \int \frac{d^3k}{(2\pi)^3} v_i(k) v_j(k) (-\frac{d f^0}{d \epsilon})
\end{align*}
We have
\begin{align*}
{\bf j} = \sigma {\bf E}
\end{align*}
{\bf Anomalous Current}\\
\begin{align*}
v_{anomalous} = - \frac{e}{\hbar} E \times \Omega_n{\bf k}
\end{align*}
Imagine we apply an electric field in x direction, and we would like to know the current in y
\begin{align*}
j_y = -e \sum_n \int \frac{d^2k}{(2\pi)^2}f_n(\bf k)[-\frac{e}{\hbar}(E \times \Omega_n({\bf k}))_y]
\end{align*}
For 2D system, $E = (E_x, E_y)$, and $\Omega$ is in z direction. So
\begin{align*}
(E \times \Omega_n)_y = E_z \Omega_n^x - E_x \Omega_n^z
\end{align*}
Because $E_z = 0$, and $\Omega_n^x=0$
\begin{align*}
(E \times \Omega_n)_y = - E_x \Omega_n^z({\bf k})
\end{align*}
Plugging this into the formula of $j_y$
\begin{align*}
j_y = [-\frac{e^2}{\hbar}\sum_n \int_{BZ} \frac{d^2 k}{(2\pi)^2}f_n({\bf k})\Omega_n({\bf k})]E_x
\end{align*}
The conductivity is
\begin{align*}
\sigma_{xy} =-\frac{e^2}{\hbar}\sum_n \int_{BZ} \frac{d^k}{(2\pi)^2}f_n({\bf k})\Omega_n(k_x, k_y)
\end{align*}
%It can be proved the integral below is an integer, called Chern number
%\begin{align*}
%C_n = \frac{1}{2\pi} \int_{BZ} d^2 k \Omega_n({\bf k})
%\end{align*}

%The current operator is
%\begin{align*}
%	j = -2e\sum_{all bands}\int_{BZ}\frac{dk}{2 \pi} f(k)v(k)
%\end{align*}
%The integral is taken over the first Brillouin zone denoted by BZ.
%We need to consider several cases to discuss the current.\\
%1) If the system preservers time reversal symmetry and space reversal symmetry, then the second term in Eqn. \ref{velocity} vanishes. In this case, if the bands are fully occupied, it is an insulator. If the band are not fully occupied, it is an conductor. The semiconductor is something between this two where the thermal excitation can promote the electron going from valence band to the conduction band so the not fully occupied conduction band contributes the current.\\
%2) If all the bands are filled, but the system breaks time and space reversal symmetry, then first term vanishes, the second term is non-trivial. This is what contributes the current in quantum Hall effect.\\
%{\bf Quantized current}\\
%When the first term of the velocity vanishes(in the case that all bands are fully occupied), the expression of velocity reduces to
%\begin{align}
%	\bar v({\bf k},t) = - i  (<\frac{\partial u_n}{\partial {\bf k}}| \frac{\partial u_n}{\partial t}> -<\frac{\partial u_n}{\partial t}| %%\frac{\partial u_n}{\partial {\bf k}}>)
%\end{align}
%Using the relationship 
%\begin{align*}
%	\partial_t = \partial_t {\bf k} \cdot \nabla_{\bf k} = -\frac{e}{\hbar} {\bf E} \times \Omega^n({\bf k})
%\end{align*}
%where 
%\begin{align*}
%	\Omega^{n}({\bf k}) = \nabla_{\bf k} \times <u_n({\bf k})|i\nabla_{\bf k}|u_n({\bf k})>
%\end{align*}
%\begin{align*}
%	{\bf v_n}({\bf k}) = - \frac{e}{\hbar} {\bf E} \times \Omega^{n}({\bf k})
%\end{align*}
%We plug the expression of $v_n$ into the expression of current j, we have
%\begin{align*}
%	{\bf j} = -2e\sum_{all bands}\int_{BZ}\frac{d{\bf k}}{2 \pi} f({\bf k}){\bf v}_n({\bf k}) 
%	 = \frac{e^2}{h} \frac{1}{2 \pi} \sum_n \int_{BZ} d{\bf k} {\bf E} \times \Omega^n({\bf k})
%\end{align*}
%Therefore the conductivity is
%\begin{align*}
%	\sigma = \frac{e^2}{h} \frac{1}{2 \pi} \sum_n \int_{BZ} d{\bf k} \Omega^n({\bf k})
%\end{align*}
%If we consider the case of 2 dimensional electron gas, then 
%\begin{align*}
%	\sigma = \frac{e^2}{h} \frac{1}{2 \pi} \sum_n \int_{BZ} d{\bf k} \Omega^n(k_x, k_y)
%\end{align*}
Since the integral runs over the first Brillouin zone, and
\begin{align*}
	\Omega^n(k_x, k_y) = \Omega^n(k_x+\pi, k_y) = \Omega^n(k_x, k_y + \pi)
\end{align*}
Hence, the first Brillouin zone forms a closed torus. The integral over a closed torus gives an multiple integer of $2\pi$. So
\begin{align*}
	\sigma_H = \nu \frac{e^2}{h}
\end{align*}
Where $\nu$ is an integer.
\end{document}