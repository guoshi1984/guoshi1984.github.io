\documentclass[a4paper]{article}
\usepackage{amssymb, amsmath}
\usepackage{graphicx}
\begin{document}
\section{Basic of Fourier Transform}
If $f(x)= f(x+T)$ then $f(x)$ can be written as
\begin{align*}
f(x) = \sum_{-\infty}^{+\infty} c_k e^{\frac{2 \pi i k x}{T}}	
\end{align*}
i is the imaginary unit, and k is an integer.  The above expression is eligible 
\begin{align*}
	e^{\frac{2 \pi i k x}{T}} = e^{\frac{2 \pi i k(x+T)}{T}}
\end{align*}
Each basis $e^{\frac{2 \pi i }{T}}$ represents a signal with frequency $F_k = \frac{k}{T}$. So the interval between each adjacent frequency $\Delta F = \frac{1}{T}$.

Based on orthogonality,
\begin{align*}
	c_k = \frac{1}{T} \int_0^T f(x) e^{-i\frac{2 \pi k x}{T}} dx
\end{align*}
The above is the Fourier transform in continuous case, in discrete case
If $x = n \Delta t$, where n = 1...N, and $T= N \Delta t$, then the Fourier series can be written as
\begin{align*}
	f(n) & = \sum_{-\infty}^{+\infty} c_k e^{\frac{2 \pi i k n \Delta t}{N \Delta t}}\\
	     & = \sum_{-\infty}^{+\infty} c_k e^{\frac{2 \pi i k n }{N }}\\
\end{align*}
\begin{align*}
	c_k = \frac{1}{N \Delta t} \sum_{n=1}^{N} f(n \Delta t) e^{-i 2\pi k \frac{1}{N\Delta t} n \Delta t} d(n \Delta t)
	=\frac{1}{N} \sum_{n = 1} ^ {N} f(n) e^{-i 2 \pi k \frac{n}{N}}
\end{align*}
This is the discrete Fourier transform.
\begin{align*}
	\Delta F = \frac{1}{T} = \frac{1}{N \Delta t}
\end{align*}
N is the total sample within time T.\\
{\bf Properties}\\
\noindent 1) To be eligible, f(x) has to be a period function with time T(with frequency $F=\frac{1}{T}$) in both continuous case and discrete case.  The requirement in discrete case leads to uniform sampling theorem used in signal processing. The total sampling time $T_{sampling}$ has to be an integer multiple of $T$.
\begin{align*}
T_{sampling} = M T 
\end{align*}
while $T= \frac{N}{F_s}$
So
\begin{align*}
	M T = N \delta t
\end{align*}
if we let $\delta t= \frac{1}{F_s}$, where $F_s$ is the sampling frequency, and $T=\frac{1}{F}$, we have
\begin{align*}
	\frac{M}{F} = \frac{N}{F_s}
\end{align*}
\noindent 2) If $f(x)$ is real, which means $f(x) = f^{*}(x)$. We then substitute Fourier series for both $f(x)$ and $f*(x)$,  
\begin{align} \label{eq1}
	\sum_{-\infty}^{+\infty} c_k e^{2 \pi i \frac{1}{T} kx} = \sum_{-\infty}^{+\infty} c_k^* e^{-2 \pi i \frac{1}{T} kx}
\end{align}
Since the summation on the right hand side is from $-\infty$ to $\infty$, it is eligible to replace $k$ with $k$.
\begin{align} \label{eq2}
	\sum_{-\infty}^{+\infty} c_k^* e^{-2 \pi i \frac{1}{T} kx}
= \sum_{\infty}^{-\infty} c_{-k}^* e^{2 \pi i \frac{1}{T} kx}
\end{align}
Combine the above two equations \ref{eq1} and \ref{eq2}, we can see $c_k = c_{-k}^*$. This means they are complex conjugate: their magnitude are equal, their phase are opposite.  Namely $||c_k|| = ||c_{-k}||$, $\phi(c_k) = \phi(c_{-k})$.\\
\noindent 3) Connection between complex representation and real representation.\\
We have shown that for real signal $c_k = c_{-k}^*$ and $c_k = |c_k| e^{j \theta_k}$, 
$c_{-k} = |c_k| e^{- j \theta_k}$. And in complex representation, we can combine the term with index k and -k,
\begin{align*}
	c_k e^{j 2 \pi k F_0 t} + c_{-k} e^{- j 2 \pi k F_0 t} = 2 |c_k| cos(2 \pi k F_0 t + \theta_k)
\end{align*}

\begin{align*}
f(x) & = \sum_{-\infty}^{+\infty} c_k e^{\frac{2 \pi i k x}{T}}	 \\
& = c_0 + 2 \sum_{k=1}^{\infty} |c_k| cos(2 \pi k F_0 t + \theta_k) \\
& = a_0 + \sum_{k=1}^{\infty} (a_k cos(2 \pi k F_0 t)  - b_k sin(2 \pi k F_0 t) ) \\
\end{align*}
where $a_0 = c_0$, $a_k = 2 |c_k|cos \theta_k$, $b_k = 2 |c_k| sin \theta_k$. 
\noindent 4) $c_k = c_{k+N}$. So when a signal contains frequency component no larger than $B$, in other words, the bandwidth of the signal is $2B(-B\textrm( to )B)$, then in order to capture the whole bandwidth of the signal, $N \Delta f > 2B$. This leads to Nyquist sampling theorem $F_s > 2B(bandwidth)$.\\
\noindent 5) Power density
\begin{align*}
	P_x & = \frac{1}{T} \int |x(t)|^2 dt \\
	    & = \frac{1}{T} \int x(t) \sum_{-\infty}^{\infty} c_k^* e^{-j 2 \pi k F_0 t} \\
	    & = \sum_{-\infty}^{\infty} c_k^*  [\frac{1}{T} \int x(t) e^{-j 2 \pi k F_0 t}] \\
	    & = \sum_{-\infty}^{\infty} |c_k|^2\\
\end{align*}
When signal is real, then 
\begin{align*}
	P_x & = \sum_{-\infty}^{\infty} |c_k|^2\\
	    & = a_0^2 + \frac{1}{2} \sum_{k=1}^{\infty} (a_k^2 + b_k^2)\\
\end{align*}
\section{Fast Fourier Transform}
\begin{align*}
	X_k =  \sum_{n = 0} ^ {N-1} x_n e^{-i 2 \pi k \frac{n}{N}}
\end{align*}
let
\begin{align*}
	u_k =e^{-i 2 \pi k \frac{n}{N}}
\end{align*}
then we have the basis orthogonality
\begin{align*}
	u_{k1}^T u_{k2} = N \delta_{k_1, k_2}
\end{align*}
We recognize we can write $X_k$ with even index terms and odd index terms
\begin{align*}
	X_k &=  \textrm{ Even index parts } + \textrm{ Odd index parts }\\
	    &= \sum_{m=0} ^{N/2-1} x_{2m} e^{-\frac{2\pi i}{N} 2mk}          
         + \sum_{m=0} ^{N/2-1} x_{2m+1} e^{-\frac{2\pi i}{N} (2m+1)k}\\
        &= \sum_{m=0} ^{N/2-1} x_{2m} e^{-\frac{2\pi i}{N/2} mk}\\
       &\textrm{(We can view this as Fourier Transform of N/2 even indexed points, 
       where k is 0,1��N/2)}\\
  			&+  e^{-\frac{2\pi i}{N} k}\\
      &\sum_{m=0} ^{N/2-1}  x_{2m+1} e^{-\frac{2\pi i}{N/2} mk}\\
      &\textrm{(We can view this as Fourier Transform of N/2 odd indexed points, where k is 0,1��N/2)}\\
      &\textrm{(Since each part is a Fourier transform of N/2 points, k has to be smaller than N/2)}\\
      &= E_k + e^{-\frac{2\pi i}{N} k} O_k
\end{align*}
As noted, the above derivation is for $k<N/2$, a very similar derivation for $N/2<=k<N$ leads to 
\begin{align*}
	X_{k+N/2} =  E_k - e^{-\frac{2\pi i}{N} k} O_k
\end{align*}
Now we have divided the FFT of N points to two FFT with N/2 points
Keep going till we reach the size to one, then combine together recursively.
\subsection{Connection with Uncertainty Principle}
{\bf Relationship between time length and frequency bandwidth}\\
We consider a few examples\\
1) We consider a function g(t) which is infinitely long in time domain\\
\begin{align*}
	g(t) = cos(2\pi f_0 t)
\end{align*}
Its Fourier transform is 
\begin{align*}
	F(f) = \frac{1}{2} \delta(f-f_0) + \frac{1}{2} \delta(f + f_0)
\end{align*}
Since the delta function has width zero, so the the bandwidth in frequency domain is zero. We see a signal which is infinitely long in time domain has zero bandwidth in frequency domain. 
2) We consider a function g(t) which has zero width in time, namely an impulse function.\\
\begin{align*}
	g(t) = \delta(t)
\end{align*}
Since this function is not a periodic function, we assume its period is infinity. Its Fourier transform is 
\begin{align*}
	F(f) = \int_{-\infty}^{\infty} \delta(t) e^{-2\pi f t} = 1
\end{align*}
Now we see a signal which has zero width in time has infinitely long frequency bandwidth. This leads to the uncertainty principle.\\
{\bf Uncertainty Principle}
In quantum mechanics, if there is a particle with position x and momentum p, then uncertainty principle states
\begin{align*}
	\Delta x \Delta p \geq \frac{\hbar}{2}
\end{align*}
Similar relationship holds for time t and Energy.
\begin{align*}
	\Delta t \Delta E \geq \frac{\hbar}{2}
\end{align*}
We can modify this expression to get the time and frequency relationship in our Fourier transform. Since 
$E = \hbar \omega$. Then 
\begin{align*}
	\Delta t \Delta \omega \geq \frac{1}{2}
\end{align*}
\end{document}