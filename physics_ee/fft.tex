\documentclass[a4paper]{article}
\usepackage{amssymb, amsmath}
\usepackage{graphicx}
\begin{document}
\section{Basic of Fourier Transform}
If $f(x)= f(x+T)$ then $f(x)$ can be written as
\begin{align*}
f(x) = \sum_{-\infty}^{+\infty} c_k e^{\frac{2 \pi i k x}{T}}	
\end{align*}
because
\begin{align*}
	e^{\frac{2 \pi i k x}{T}} = e^{\frac{2 \pi i k(x+T)}{T}}
\end{align*}
Based on orthogonality,
\begin{align*}
	c_k = \frac{1}{T} \int_0^T f(x) e^{-i\frac{2 \pi k x}{T}} dx
\end{align*}
The above is the Fourier transform in continuous case, in discrete case
If $x = n \Delta t$, where n = 1...N, and $T= N \Delta t$, then the Fourier series can be written as
\begin{align*}
	f(n) & = \sum_{-\infty}^{+\infty} c_k e^{\frac{2 \pi i k n \Delta t}{N \Delta t}}\\
	     & = \sum_{-\infty}^{+\infty} c_k e^{\frac{2 \pi i k n }{N }}\\
\end{align*}
\begin{align*}
	c_k = \frac{1}{N \Delta t} \sum_{n=1}^{N} f(n \Delta t) e^{-i 2\pi k \frac{1}{N\Delta t} n \Delta t} d(n \Delta t)
	=\frac{1}{N} \sum_{n = 1} ^ {N} f(n) e^{-i 2 \pi k \frac{n}{N}}
\end{align*}
This is the discrete Fourier transform.
\begin{align*}
	\Delta f = \frac{1}{T} = \frac{1}{N \Delta t} = \frac{F_s}{N}
\end{align*}
Where $F_s$, $N$ are the sample frequency and number of samples.\\
{\bf Properties}\\
\noindent 1) To be eligible, f(x) has to be a period function with time T. This leads to uniform sampling theorem used in signal processing. The uniform sampling theorem states w\\
\noindent 2) If $f(x)$ is real, which means $f(x) = f^{*}(x)$. We then substitute Fourier series for both $f(x)$ and $f*(x)$,  
\begin{align} \label{eq1}
	\sum_{-\infty}^{+\infty} c_k e^{2 \pi i \frac{1}{T} kx} = \sum_{-\infty}^{+\infty} c_k^* e^{-2 \pi i \frac{1}{T} kx}
\end{align}
Since the summation on the right hand side is from $-\infty$ to $\infty$, it is eligible to replace $k$ with $k$.
\begin{align} \label{eq2}
	\sum_{-\infty}^{+\infty} c_k^* e^{-2 \pi i \frac{1}{T} kx}
= \sum_{\infty}^{-\infty} c_{-k}^* e^{2 \pi i \frac{1}{T} kx}
\end{align}
Combine the above two equations \ref{eq1} and \ref{eq2}, we can see $c_k = c_{-k}^*$. This means they are complex conjugate: their magnitude are equal, their phase are opposite.  Namely $||c_k|| = ||c_{-k}||$, $\phi(c_k) = \phi(c_{-k})$.\\
\noindent 3) Connection between complex representation and real representation.
\noindent 4) $c_k = c_{k+N}$. So when a signal contains frequency component no larger than $B$, in other words, the bandwidth of the signal is $2B(-B\textrm( to )B)$, then in order to capture the whole bandwidth of the signal, $N \Delta f > 2B$. This leads to Nyquist sampling theorem $F_s > 2B(bandwidth)$.\\
\section{Fast Fourier Transform}
\begin{align*}
	X_k =  \sum_{n = 0} ^ {N-1} x_n e^{-i 2 \pi k \frac{n}{N}}
\end{align*}
let
\begin{align*}
	u_k =e^{-i 2 \pi k \frac{n}{N}}
\end{align*}
then we have the basis orthogonality
\begin{align*}
	u_{k1}^T u_{k2} = N \delta_{k_1, k_2}
\end{align*}
We recognize we can write $X_k$ with even index terms and odd index terms
\begin{align*}
	X_k &=  \textrm{ Even index parts } + \textrm{ Odd index parts }\\
	    &= \sum_{m=0} ^{N/2-1} x_{2m} e^{-\frac{2\pi i}{N} 2mk}          
         + \sum_{m=0} ^{N/2-1} x_{2m+1} e^{-\frac{2\pi i}{N} (2m+1)k}\\
        &= \sum_{m=0} ^{N/2-1} x_{2m} e^{-\frac{2\pi i}{N/2} mk}\\
       &\textrm{(We can view this as Fourier Transform of N/2 even indexed points, 
       where k is 0,1��N/2)}\\
  			&+  e^{-\frac{2\pi i}{N} k}\\
      &\sum_{m=0} ^{N/2-1}  x_{2m+1} e^{-\frac{2\pi i}{N/2} mk}\\
      &\textrm{(We can view this as Fourier Transform of N/2 odd indexed points, where k is 0,1��N/2)}\\
      &\textrm{(Since each part is a Fourier transform of N/2 points, k has to be smaller than N/2)}\\
      &= E_k + e^{-\frac{2\pi i}{N} k} O_k
\end{align*}
As noted, the above derivation is for $k<N/2$, a very similar derivation for $N/2<=k<N$ leads to 
\begin{align*}
	X_{k+N/2} =  E_k - e^{-\frac{2\pi i}{N} k} O_k
\end{align*}
Now we have divided the FFT of N points to two FFT with N/2 points
Keep going till we reach the size to one, then combine together recursively.
\end{document}