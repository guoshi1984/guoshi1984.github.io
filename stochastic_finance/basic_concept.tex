\documentclass[a4paper]{article}
\usepackage{amssymb, amsmath}
\usepackage{graphicx}
\begin{document}
Author: Dr. Shi Guo  \hspace{30mm} Email: guoshi1984@hotmail.com\\
\line(1,0){350}\\
\noindent The reason of the article is to provide a basic review of several key concepts in measure theory. Each concept is illustrated with an example so that one can easily understand.
\section{$\sigma$ Algebra}
{\bf a.	Sigma algebra definition} \\ 
Given a non-empty set $\Omega$, a sigma algebra is defined\\
1)	Include empty set and whole set \\ 
2)	Include the complement of any element itself \\
3)	Closed under countable union \\


\noindent {\bf  b.	Sigma algebra example by tossing a coin} \\
Tossing 0 time\\    
Check 1)\\    
$F_0$ = { 0,}\\
Tossing once    \\
Check 1   $\Omega$ = {H, T} \\
Check 2 $H_c$ = T, $T_c$ = H \\
Check 3 H U T = $\Omega$ \\
So  \\ 
$F_1$ = {0, $\Omega$, H, T} \\
Tossing twice \\
Check 1 $\Omega$ = {HH, HT, TH, TT} \\
Check 2 $HH_c$, $HT_c$, $TH_c$, $TT_c$ \\
Check 3 HH U HT, HH U TH, HH U TT, HT U TH, HT U TT, TH U TT \\
HH U HT U TH = $TT_c$, \\
HH U HT U TT = $TH_c$, \\
HH U TH U TT = $HT_c$,  \\
HT U TH U TT  = $HH_c$ \\
              So \\
              $F_2$ = { 0, $\Omega$, HH, HT, TH, TT, \\
                          $HH_c$, $HT_c$, $TH_c$, $TT_c$ \\
                         HH U HT, HH U TH, HH U TT, HT U TH, HT U TT, TH U TT,  \\
                          $TT_c$, $TH_c$, $HT_c$, $HH_c$} \\

\noindent{ \bf c.	Why define sigma algebra? }\\
On top of the sigma algebra, we can define the probability, because the object that probability measure takes is the sigma algebra. \\

\section{Filtration}
Consider a sequence of coin toss \\
For the first toss, we get $F_1$\\
For the first and second toss, we get $F_2$\\
For the first n tosses, we get $F_n$\\
The collection of sigma algebra $F_1$, $F_2$�� $F_n$ is called a Filtration.\\

\section{Random variable}
{\bf a. Definition} \\
A random variable is function from $\Omega$ to R, with the property that for every Borel subset B of R, the subset of $\Omega$ is in $\sigma$-algebra F. \\

\noindent {\bf b. Example} \\
Consider 3 toss case, H with prob p, T with prob q\\
Def. random variable S
$S_0(w_0)$  = 4 for all $\omega$ \\
\begin{align*}
S_{n+1}(w_{n+1}) =& 2S_{n}(w_n)  \hspace{2mm} \text{if} \hspace{2mm} w_{n+1}= H \\
                        	      &\frac{1}{2}S_{n}(w_n) \hspace{2mm} \text{if} \hspace{2mm} w_{n+1} = T \\
\end{align*}
so\\           
$S_0(w_1 w_2 w_3)$ = 4  for all $w_i$ \\
$S_1(w_1 w_2 w_3)$ = 8  if $w_1$ = H \\
$S_1(w_1 w_2 w_3)$ = 2  if $w_1$ = T \\
$S_2(w_1 w_2 w_3)$ = 16  if $w_1 = w_2 = H$ \\
$S_2(w_1 w_2 w_3)$ = 4   if $w_1 \neq w_2$\\ 
$S_2(w_1 w_2 w_3)$ = 1   if $w_1 = w_2 = T$ \\

          
\section{ \bf $\sigma$ algebra generated by a random variable and measurable function } 
Give consider a random variable S: $\Omega$ to R, for every open set in R, the collection of their inverse image forms an sigma algebra, and it is called the sigma algebra generated by S. And S is called F-measurable. The concept measurable is not very intuitive to understand. An easy way to understand this is S
is completely determined by F, then S is F measurable. \\



\section{Conditional Expectation}

\noindent {\bf a. Definition }\\
1)	$E[X|G ]$ is G measurable, which means the value of $E[X\G]$ is completely determined by G$ \\
2)	$\int_A E[X|G](w)dP(w)= \int_A X(w)dP(w)$ for all A which belongs to G\\


\noindent{ \bf b.	Example to understand 2)}\\
Consider 3 toss case, H with prob p, T with prob q \\
Define random variable S\\
$S_0(w)  = 4$ for all w \\
$S_{n+1}(w) = 2S_{n}(w)$ if $w_{n+1}= H$ \\
$S_{n+1}(w) = \frac{1}{2} S_{n}(w)$  if $w_{n+1} = T$ \\
Expectation of 3 tosses random variable $S_3$ give the first two is HH\\
\begin{align*}
& E_2(S_3 | HH) = p S_3(HHH) + qS_3(HHT) \\
& E_2(S_3 | HT)  = p S_3(HTH) + qS_3(HTT) \\
& E_2(S_3 | TH)  = p S_3(THH) + qS_3(THT) \\
& E_2(S_3 | TT)  = p S_3(TTH) + qS_3(TTT) \\
& E_2(S_3 | HH) P(HH) = prob(HHH) S_3(HHH) + prob(HHT) S_3(HHT) \\
& E_2(S_3 | HT)  P(HT)= prob(HTH) S_3(HTH) + prob(HTT) S_3(HTT) \\
& E_2(S_3 | TH)  P(TH)= prob(THH) S_3(THH) + prob(THT )S_3(THT) \\
& E_2(S_3 | TT)  P(TT)= prob(HTH) S_3(TTH) + prob(TTT)  S_3(TTT) \\
\end{align*}
This confirms def 2),  for A = HH or HT or TH or TT \\
$\int \E_2(S_3|G)(w) dP(w) = \int _A X(w)dP(w)$\\

\noindent{\bf c. Properties}\\

\noindent 1)  The conditional expectation is a random variable. Because the value is dependent on G.\\

\noindent 2)  If X is G measurable, then  $E[X|G] = X$.\\

\noindent 3)  If X is G measurable $E[XY|G] = X E[Y|G]$, this is to take out what is known.\\

\noindent 4)	If X is independent of G, $E[X|G] = EX$\\

	\noindent To understand 2), 3) and 4), consider two extreme cases\\
	Define random variable S \\
	$S_0(w)  = 4$ for all w \\
	$S_{n+1}(w) = 2S_{n}(w)$ if $w_{n+1}= H$ \\
	$S_{n+1}(w) = \frac{1}{2} S_{n}(w)$  if $w_{n+1} = T$ \\
	Then a condition expectation can be defined as\\ 
	$E[S_n| F_t] = E[S_n|\omega_1, \omega_2,..., \omega_t]$\\
	
	\noindent If t=n, then $E[S_n|F_n] = S_n$, this is because when $F_n$ is known, then $S_n$ is known, there is nothing to average. This corresponds to Property 2) and 3)\\
	
	\noindent If t=0, then $E[S_n|F_0] = E[S_n]$, this is because $F_0$ provides no restriction to average $S_n$, the conditional expectation needs to average all possible cases, it is a general expectation. This corresponds to Property 4). \\

\noindent 
5)	If G is a subset of  H\\
	$E[E[X|G|H]] = E[X|H]$\\
	
\section{Law of Large Numbers}

\noindent{\bf a. Weak law of large number}\\
Suppose $X_1$, $X_2$,..., $X_n$ are iid, and $u$ is the expectation.\\
$lim_{n \to \infty} Pr(|\bar X - u| > \textgreater \epsilon) = 0$\\

\noindent{\bf b. Strong law of large number}\\
$Pr(lim_{n \to \infty} \bar X = u) = 1$\\

\noindent{\bf c. Difference}\\
In weak case, $|X-u|> \epsilon$ can happen infinite times, however, in strong case, it does not. There exist in certain case where $X_n$ converges in weak case but does not converge in strong case.  An example would be a series of $X_n$ that is conditionally convergent, which means the series does not converge absolutely, and by rearranging terms, the series converges to a different value. For example, if X be random variable following geometric distribution with probability 0.5. Then the expectation of a new random variable $2^X (-1)^X X^{-1}$ is 
\begin{align*}
	&E[2^X (-1)^X X^{-1}] = \sum_1 ^{\infty} \frac{(-1)^x}{x}\\
   =& -1 + \frac{1}{2} - \frac{1}{3} ...\\
   =&-ln2    \\
\end{align*}
By rearranging the terms, 
\begin{align*}
	 &-1 + \frac{1}{2} - \frac{1}{3} + \frac{1}{4} - \frac{1}{5} \\
	 =& (-1 + \frac{1}{2}) + \frac{1}{4} +(- \frac{1}{3} +  \frac{1}{6}) + \frac{1}{8}\\
	 =&-\frac{1}{2} ln2\\
\end{align*}   
Therefore, this is conditionally convergent, meaning it satisfies the weak law not the strong law.


\end{document}