\documentclass[a4paper]{article}
\usepackage{amssymb, amsmath}
\usepackage{graphicx}
\begin{document}
Author: Dr. Shi Guo  \hspace{30mm} Email: guoshi1984@hotmail.com\\
\line(1,0){350}\\

The goal of this article is to present a review of change of measure. By changing of measure, we can construct risk-free measure which is widely used in finance. The Girsanov theorem formulates clearly how the stochastic process changes when probability measure changes. However the theorem itself
is not easy to understand and one needs some background before diving into it. In order to understand change of measure in a stochastic process, we first discuss the change of measure for random variables by showing change of probability measure for both discrete random variables and continuous random variables. Then we move to stochastic process, and we show how to construct risk-free measure for discrete binomial model. Finally we present Girsanov theorem and illustrate the theorem using geometric Brownian motion as an example.

\section{Change of Measure for Random Variable }

{\bf The probability measure defined on the sigma algebra does not have to be unique.} Given a sigma algebra, one can define as many as different probability measure as long as they satisfy the probability measure definition. See examples below:
\subsection{Change of measure for a discrete random variable}
\noindent 1) Give a binomial random variable S associated with a coin toss, define 
\begin{align*}
&S = 1  \textrm{ if } \omega = \textrm{ Head } \\
&S = -1 \textrm{ if } \omega = \textrm{ Tail } \\
\end{align*}

\noindent 2)We define the probability measure of getting head and tail (P(Head) and P(tail)) in two ways. \\
Probability measure 1:\\
\begin{align*}
P(head) & = \frac{1}{2}\\ 
P(tail) & = \frac{1}{2}\\
\end{align*}
We can also define the probability measure $\tilde P$, under $\tilde P$\\ 
\begin{align*}
\tilde P(head) & = \frac{2}{3}\\ 
\tilde P(tail) & = \frac{1}{3}\\
\end{align*}
 

\noindent 3)We can define change of measure to connect P and $\tilde P$. Consider the transformation 
\begin{align*}
	Z(\omega) = \frac{\tilde P(\omega)}{P(\omega)}
\end{align*}
so 
\begin{align*}
	Z(H) & = \frac{\tilde P(H)}{P(H)} = \frac{4}{3}\\
	Z(T) & = \frac{\tilde P(T)}{P(T)} = \frac{2}{3}\\
\end{align*}
This is the change of measure for a discrete random variable. \\
\subsection{Change of Measure for a Continuously Random Variable(Uniformly Distributed) } 
Give a random variable  X($\omega$)=x where x in [0,1]\\
1)\\
Define probability measure P \\
P(a \textless x \textless b) = b - a, the pdf is 
\begin{align*}
p(x) = 1	
\end{align*}
This is a uniform measure.\\

\noindent 2)\\
Define another probability measure $\tilde P$\\
$\tilde P(a \textless x  \textless b) = b^2 - a^2$, the pdf is 
\begin{align*}
p(x) = 2x
\end{align*}
So this is non-uniform measure\\


\noindent 3)	To justify they both are probability measure\\
Check  $P[0,1] = 1$; $P(0)=0$;\\
       $\tilde P[0,1] =1$; $\tilde P(0)=0$;\\


\noindent 4) we can define change of measure to connect P and $\tilde P$ \\
Consider the transformation \\
\begin{align*}
	&\tilde P(a < X(\omega) = x < b) = \int _a ^b 2x dx = \int _a ^b 2x dx = \int_a ^b 2x dP(X(\omega))\\
	&d\tilde P(X(\omega)) = Z(X(\omega)) dP(X(\omega)) \\
\end{align*}
So
\begin{align*}
	Z(X(\omega))= 2x
\end{align*}
This is the change of measure for a continuously random variable. \\


\subsection{Change of Measure for a Normal Distributed Random Variable}
We show an example of change of measure in normal distribution. If X is N(0,1), let Y = X+ u, so Y is N(u,1), so the random variable Y does not have mean 0. However, based
on the definition of expectation\\
\begin{align*}
	E(Y(\omega)) = \int Y(\omega) dP(\omega)
\end{align*}
we can change the probability measure $P(\omega)$, such that E(Y) becomes zero.
Define $Z(w) = exp(-uX(\omega) - \frac{1}{2}u^2)$
We are able to show two things\\ 
1 Z \textgreater 0 \\
2 E(Z) =1 i.e. $\int Z(w)dP(X(w)) = 1$ \\
Because 
\begin{align*}
E(Z)  & = \int exp(-ux-1/2u^2) \frac{1}{\sqrt{2\pi}} exp(-1/2x^2) dx\\
& = \frac{1}{\sqrt{2\pi}}  \int exp(-1/2(x+u)^2) dx \\
& = \frac{1}{\sqrt{2\pi}}  \int exp(-1/2(y)^2) dy \\
& = 1 \\
\end{align*}

So $\tilde P(w) = \int Z(w) dP(w)$ is a new probability measure

The pdf of Y under the new measure is

\begin{align*}
	\tilde P(Y(\omega) \leq b) & = \int_{Y(\omega) \leq b} d \tilde P(\omega) \\
&= \int_{Y(\omega)<=b} Z(\omega) dP(\omega) \\
&= \int 1_{X(\omega)<=b-u} exp(-uX- \frac{1}{2} u^2 )dP(\omega) \\
&=\int 1_{X(\omega)<=b-u} exp(-uX- \frac{1}{2} u^2 ) pdf(N(0,1)) dx \\
&=  {\sqrt{2\pi}}^{-1} \int_{-\infty}^{b-u} exp(-ux -\frac{1}{2}u^2 -1/2x^2) dx \\
&=  {\sqrt{2\pi}}^{-1}\int_{-\infty}^{b-u} exp(-\frac{1}{2} (x+u)^2) dx \\
&\textrm{(changing x back to y)}\\
&=  {\sqrt{2\pi}}^{-1}\int _{-\infty}^{b} exp(-\frac{1}{2} (y)^2) dy \\
&= \textrm{cdf of N(0,1)} \\
\end{align*}
This shows it is a standard normal distribution with mean 0.\\

\section{Change of Measure for a Filtration(Series of Events in Time)}
\subsection{Change of measure for Stock under binomial model - Risk neutral measure}
Suppose we have the following stock $S_0$ at t=0. At t=1, we can associate the value of $S_1$ to outcome of tossing a coin. When we toss a coin and if the coin is fair, we can get Head and Tail and each has 50\% probability. If we get a head, the stock moves to $S_1(H)$, and if we get a tail, the stock moves to $S_1(T)$. Clearly, the stock has 50\% to move up, and 50\% to move down.
\begin{align*}
S_1(H) & = (1+\alpha + \sigma) S_0\\  
S_1(T) & = (1+\alpha - \sigma) S_0\\     
\end{align*}
In the sense of risk neutral pricing, we would like to have the stock values grows as the same as a saving account with interest rate r.  Namely, we need
\begin{align*}
S_0 (1+r) = \frac{1}{2}S_1(H) + \frac{1}{2} S_1(T) 
\end{align*}
Simply plug in the definition of $S_1$, we easily see the equation does not hold except the special case when $\alpha = r$.When $\alpha$ does not equal to r, we artificially create two probabilities p and q with p + q =1, 
define 
\begin{align*}
S_0 (1+r) = p S_1(H) + q S_1(T) 	
\end{align*}
Then solve for p and q, we have
\begin{align*}
p= \frac{r - \alpha + \sigma}{2 \sigma} \\
q= \frac{\alpha - r + \sigma}{2 \sigma} \\
\end{align*}
We call $p$ and $q$ the probabilities under {\bf risk-neutral measure}. Under this measure, the expectation of the stock return is the same as the return of saving account. To understand this measure, we can see when $\alpha > r$ then $q(H)< q(T)$, so we lower the prob of stock moving up and raise the prob of the stock moving down such that the return is exactly 1+r. The same argument holds for $r < \alpha$. 

A very important property of risk-neutral measure is the discounted price of the portfolio with respect to time is a {\bf martingale}. From the above equation, if we define the discounted price of S is DS, where 
\begin{align*}
DS_0 = S_0\\
DS_1 = \frac{1}{1+r} S_1
\end{align*}
we can see
\begin{align*}
S_0 = \frac{1}{1+r} (p S_1(H) + q S_1(T)) = p DS_1(H) + q DS_1(T) = \tilde E[DS_1]
\end{align*}
Where $\tilde E[DS_1]$ stands for expectation of the discounted price of asset S at time 1 under risk-neutral measure.\\

\subsection{Girsanov's Theorem}
\noindent{\bf Define change of measure for continous variable}\\
For $(\Omega, F, P)$, given random variable Z with $E(Z) = 1$, define new probability measure
\begin{align}
	\tilde P = \int_A Z(\omega) dP(\omega)\\
\end{align}
We have two expectation defined, one is under P, the other under $P^{'}$
\begin{align*}
	&\tilde E[X] = E[XZ]\\ 
	&d\tilde P(\omega) = Z(\omega) dP(\omega)\\
	&Z(\omega) = \frac{d\tilde P(\omega)}{dP(\omega)}\\
\end{align*}

\noindent{\bf Define change of measure for filtration}\\
$E[Z] = 1$ and $Z(t) = E[Z|F(t)]$\\

\noindent{\bf Properties of Z(t)}\\
1) Martingale\\
Given $0 <= s <=t <= T$\\
$E[Z(t)|F(s)] = E[E[Z|F(t)]|F(s)] = E[Z|F(s)] = Z(s)$\\
2) $\tilde E[Y] = E[YZ(t)]$\\
$\tilde E[Y]=E[YZ] = E[E[YZ|F(t)]]=E[Y E[Z|F(t)]] = E[YZ(t)]$\\
3) Given $0<=s<=t<=T$, Y is F(t)- measurable, then\\
$\tilde E[Y|F(s)] = \frac{1}{Z(s)} E[YZ(t)|F(s)]$\\

\noindent{\bf Girsanov's Theorem}\\
Suppose W(t) is Brownian Motion given $\Omega, F, P$ and F(t) is the filtration. Let $\Theta(t)$, $0<=t<=T$ is adapted process, define $Z(t) = exp(-\int_0^{t} \Theta(u) dW(u))$, and $\tilde W(t) = W(t) + \int_0^t \Theta(u) du$, s.t. $E[\int_0^T \Theta^2(u) Z^2(u) du < \infty]$. Then $E[Z]=1$, and under $\tilde P$, $\tilde W(t)$ is Brownian motion.\\

\subsection{Risk Neutral Measure with a Filtration}
We assume the stock price using geometric Brownian motion\\
\begin{align*}
dS(t) = \alpha(t) S(t) dt + \sigma(t) S(t) d W(t)
\end{align*}
Its integrated form is\\
\begin{align*}
	S(t) = S(0) exp(\int _0 ^t \sigma(s) dW(s) + \int _0 ^t (\alpha(s) - \frac{1}{2} \sigma^2(s)) ds)\\
\end{align*}
define $D(t) = exp(-\int_0^t R(s) ds) $, then $dD(t) = R(t)D(t)dt$\\
Discounted stock price\\
\begin{align*}
	D(t)S(t) = S(0) exp(\int_0^t \sigma(s) dW(s) + \int_0^t (\alpha(s) - R(s) - \frac{1}{2} \sigma^2(s)) ds)
\end{align*}
\begin{align*}
	d(D(t) S(t)) = \sigma(t) D(t) S(t) [\Theta(t) dt + dW(t)]\\
\end{align*}
where $\Theta(t) = \frac{\alpha(t) -R(t)}{\sigma(t)}$\\
We define a new Brownian motion with drift $d\tilde W(t) = dW(t) + \theta(t) dt$, this brownian motion consists of a standard brownian motion and a drift term $\theta(t)$. So if $\tilde W(0) = 0$, due to the drift term, $E[\tilde W(t)] != 0$.However, using the change of measure theorem, we can define another probability measure $\tilde P$, such that under $\tilde P$, $\tilde W(t)$ becomes a standard brownian motion.\\
Then we have $d(D(t)S(t)) = \sigma(t) D(t) S(t) d\tilde W(t)$ under $\tilde P$, {\bf which implies the discount price is a martingale under the risk-neutral measure $\tilde P$}, then $dS(t) = R(t) S(t) dt + \sigma(t) S(t) d  \tilde W(t)$. The drift term now change from $\alpha$ to risk-free interest Rate R. Knowing $d(D(t)S(t))$ is a martingale, if we consider a portfolio $X(t)$ composed with a $\Delta(t)$ share of stock and interesting account, 
\begin{align*}
	d X(t) = \Delta(t) dS(t) + r(1 - Delta(t)S(t))dt
\end{align*}
we can work out a little of math to show
\begin{align*}
	d(D(t)X(t)) = \Delta(t) d(D(t)S(t))
\end{align*}
This means $D(t)X(t)$ is a martingale under $\tilde P$. We have
\begin{align*}
	D(t)X(t) = \tilde E [D(T) X(T) | F(t)]
\end{align*}
We can use this martigale property to price financial instruments.
{\bf This is our risk-neutral pricing formula.}
\subsection{Risk Neutral Pricing Examples}
{\bf 1. European Call Option}\\
For European call option, Let V(t) be the price of the option at time t.  $V(T) = (S(T) - K)^{+}$ where K is the striking price. Then under risk neutral measure, suppose interest rate is a constant, then
\begin{align*}
	V(t) = \tilde E[e^{r(t-T)}V(T)|F(t)]
\end{align*}
{\bf 2. Forward Price}\\
Suppose K is the forward price at time t for delivering one share of stock at time T, whose price is S(T). Then under risk-neutral measure, the discount stock price is a martingale, which means
\begin{align*}
	D(t)S(t) = \tilde E[D(T) S(T)|F(t)]
\end{align*}
again, if we assume interest rate r is a constant,
\begin{align*}
	S(t) = \tilde E[e^{r(t-T)S(T)}|F(t)]
\end{align*}
This expectation value of the discounted stock has to equal to the present value of K to ensure no arbitrage, so
\begin{align*}
	S(t) = \tilde E[e^{r(t-T)S(T)}|F(t)] = e^{r(t-T)}K
\end{align*}
So
\begin{align*}
	K = e^{r(T-t)}S(t)
\end{align*}
\subsection{Risk Neutral Measure with Foreign Currency}
There is an exchange rate Q(t), which gives units of domestic currency per unit of foreign currency. We assume the exchange rate satisfies
\begin{align*}
	dQ(t) = \gamma(t)Q(t)dt + \sigma_2(t)Q(t)dW_3(t)
\end{align*}
There is also a foreign interest rate $R^f(t)$, which leads to foreign money market account price
\begin{align*}
	M^f(t) = e^{\int_0^t R^f(u) du}
\end{align*}
The value of the foreign money market account in domestic currency is $M^f(t)Q(t)$, and its discounted value is $D(t)M^f(t)Q(t)$
\begin{align*}
	d(D(t)M^f(t)Q(t)) = D(t)M^f(t)Q(t)[(R^f - R(t) + \gamma(t))dt + \sigma_2(t) dW_3(t)]
\end{align*}
Now we create a risk-neutral measure $\tilde P$, under which $D(t)M^f(t)Q(t)$ is a martingale. This means
\begin{align*}
	& D(t)M^f(t)Q(t) = \tilde E[D(T)M^f(T)Q(T)|F(t)] \\
	& d(D(t)M^f(t)Q(t)) = \sigma_2(t) D(t)M^f(t)Q(t)d\tilde W_3(t) \\
\end{align*}
And
\begin{align*}
	d(M^f(t)Q(t)) = M^f(t)Q(t)(R(t)dt + \sigma_2(t) d\tilde W_3(t))
\end{align*}
Finally, we obtain the exchange rate Q(t) and risk neutral measure
\begin{align*}
	dQ(t) = Q(t)((R(t) - R^f(t))dt + \sigma_2(t) d\tilde W_3(t))
\end{align*}
{\bf Example}
Suppose $R(t)$ and $R^f(t)$ are fixed, then based on martingale property, 
\begin{align*}
	Q(0) = D(T) M^f(T)Q(T) = e^{-rT} e^{r^fT}Q(T)
\end{align*}
In the above if we call Q(0) the spot price(S) of the exchange rate and Q(T) the forward price F of the exchange rate, we then get the price of the forward
\begin{align*}
	& S = e^{-rT} e^{r^fT}F \\
	& F = e^{(r - r^f)T}S \\
\end{align*}
Here is how we intepret negative sign of $r^f$. Consider one unit of foreign currency, when $r^f = 0$, then return of one unit of foreign currency has to match the return of the saving account in domestic currency based on risk-neutral measure. When we invest amount of S in domestic save account, the value at time T is $S(t) = e^{rt}S$, and the forward price of the foreign currency at time T has to match it, therefore $F = e^{rt} S$. When $r^f = r$, then if we invest one unit of foreign currency, after time T we get $e^{r^fT} = e^{rT}$ in foreign currency. This amount of foregin currecy is equal to $e^{rT}F$ amount in domestic currency and it has to match the value of the saving account in domestic currency $e^{rT}S$, so $F = S$.
\end{document}

