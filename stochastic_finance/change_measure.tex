\documentclass[a4paper]{article}
\usepackage{amssymb, amsmath}
\usepackage{graphicx}
\begin{document}
Author: Dr. Shi Guo  \hspace{30mm} Email: guoshi1984@hotmail.com\\
\line(1,0){350}\\

\noindent The goal of this article is to present a review of change of measure. By changing of measure, we can construct risk-free measure which is widely used in finance. The Girsanov theorem formulates clearly how the stochastic process changes when probability measure changes. However the theorem itself
is not easy to understand and one needs some background before diving into it. In order to understand change of measure in a stochastic process, we first discuss the change of measure for random variables by showing change of probability measure for both discrete random variables and continuous random variables. Then we move to stochastic process, and we show how to construct risk-free measure for discrete binomial model. Finally we present Girsanov theorem and illustrate the theorem using geometric Brownian motion as an example.

\section{Change of Measure for Random Variable }
{\bf The probability measure defined on the sigma algebra does not have to be unique.} Given a sigma algebra, one can define as many as different probability measure as long as they satisfy the probability measure definition. See examples below:
\subsection{Change of measure for a discrete random variable}
\noindent 1) Give a binomial random variable S associated with a coin toss, define S = 1 if $\omega$ = Head,
S = -1 if $\omega$ = Tail. \\

\noindent 2)We define the probability measure P. In this measure getting a head has prob $\frac{1}{2}$, and getting a tail has $\frac{1}{2}$. We can also define the probability measure $P^{'}$, under $P^{'}$, getting a head has prob $\frac{2}{3}$, and getting a tail has $\frac{1}{3}$. \\

\noindent 3)We can define change of measure to connect P and $P^{'}$. Consider the transformation $Z(\omega) = \frac{P^{'}(\omega)}{P(\omega)}$, so $Z(H) = \frac{4}{3}$, $Z(T) = \frac{2}{3}$. This is the change of measure for a discrete random variable. \\
\subsection{Change of measure for a continuously random variable(uniformly distributed) } \\

Give a random variable  X($\omega$)=x where x in [0,1]\\
1)	\\
Define probability measure P \\
P(a \textless x \textless b) = b - a, the pdf is $p(x) = 1$. 
This is a uniform measure.\\

\noindent 2)\\
Define another probability measure P^{'}\\
P^{'}(a \textless x  \textless b) = $b^2$ - $a^2$, the pdf is $p(x) = 2x$.
So this is non-uniform measure\\


\noindent 3)	To justify they both are probability measure\\
Check  P[0,1] = 1; P(0)=0;\\
       P^{'}[0,1] =1; P^{'}(0)=0;\\


\noindent 4) we can define change of measure to connect P and $P^{'}$ \\
Consider the transformation \\
$P^{'}(a < X(\omega) = x < b) = \int _a ^b 2x dx = \int _a ^b 2x dx = \int_a ^b 2x dP(X(\omega))$\\
$dP{��}(X(\omega)) = Z(X(\omega)) dP(X(\omega))$ where $Z(X(\omega))= 2x$\\
This is the change of measure for a continuously random variable. \\


\subsection{Change of measure for normal distributed random variable}\\
We show an example of change of measure in normal distribution. If X is N(0,1), let Y = X+ u, so Y is N(u,1), so the random variable Y does not have mean 0. However, based
on the definition of expectation\\
\begin{align*}
	E(Y(\omega)) = \int Y(\omega) dP(\omega)
\end{align*}
we can change the probability measure $P(\omega)$, such that E(Y) becomes zero.
Define $Z(w) = exp(-uX(\omega) - \frac{1}{2}u^2)$
We are able to show two things\\ 
1 Z \textgreater 0 \\
2 E(Z) =1 i.e. $\int Z(w)dP(X(w)) = 1$ \\
Because 
\begin{align*}
E(Z)  & = \int exp(-ux-1/2u^2) \frac{1}{\sqrt{2\pi}} exp(-1/2x^2) dx\\
& = \frac{1}{\sqrt{2\pi}}  \int exp(-1/2(x+u)^2) dx \\
& = \frac{1}{\sqrt{2\pi}}  \int exp(-1/2(y)^2) dy \\
& = 1 \\
\end{align*}

So $P^{'}(w) = \int Z(w) dP(w)$ is a new probability measure

The pdf of Y under the new measure is

\begin{align*}
	P^{'}(Y(\omega) \leq b) & = \int_{Y(\omega) \leq b} d P^{'}(\omega) \\
&= \int_{Y(\omega)<=b} Z(\omega) dP(\omega) \\
&= \int 1_{X(\omega)<=b-u} exp(-uX- \frac{1}{2} u^2 )dP(\omega) \\
&=\int 1_{X(\omega)<=b-u} exp(-uX- \frac{1}{2} u^2 ) pdf(N(0,1)) dx \\
&=  {\sqrt{2\pi}}^{-1} \int_{-\infty}^{b-u} exp(-ux -\frac{1}{2}u^2 -1/2x^2) dx \\
&=  {\sqrt{2\pi}}^{-1}\int_{-\infty}^{b-u} exp(-\frac{1}{2} (x+u)^2) dx \\
&\textrm{(changing x back to y)}\\
&=  {\sqrt{2\pi}}^{-1}\int _{-\infty}^{b} exp(-\frac{1}{2} (y)^2) dy \\
&= \textrm{cdf of N(0,1)} \\
\end{align*}
This shows it is a standard normal distribution with mean 0.\\

\section{Change of Measure for a Filtration(Series of Events in Time)}
\subsection{Change of measure for Stock under binomial model - Risk neutral measure}\\
Suppose we have the following stock $S_0$ at t=0. At t=1, we can associate the value of $S_1$ to outcome of tossing a coin. When we toss a coin and if the coin is fair, we can get Head and Tail and each has 50\% probability. If we get a head, the stock moves to $S_1(H)$, and if we get a tail, the stock moves to $S_1(T)$. Clearly, the stock has 50\% to move up, and 50\% to move down.

\begin{align*}
S_1(H) & = (1+\alpha + \sigma) S_0\\  
S_1(T) & = (1+\alpha - \sigma) S_0\\     
\end{align*}

In the sense of risk neutral pricing, we would like to have the stock values grows as the same as a saving account with interest rate r.  Namely, we need
\begin{align*}
S_0 (1+r) = \frac{1}{2}S_1(H) + \frac{1}{2} S_1(T) 
\end{align*}

Simply plug in the definition of $S_1$, we easily see the equation does not hold except the special case when $\alpha = r$.

When $\alpha$ does not equal to r, we artificially create two probabilities p and q with p + q =1, 
define 
\begin{align*}
S_0 (1+r) = p S_1(H) + q S_1(T) 	
\end{align*}
Then solve for p and q, we have
\begin{align*}
p= \frac{r - \alpha + \sigma}{2 \sigma} \\
q= \frac{\alpha - r + \sigma}{2 \sigma} \\
\end{align*}

We call this risk-neutral measure. Under this measure, the expectation of the stock return is the same as the return of saving account. We define this as risk neutral measure. To understand this measure, we can see when $\alpha > r$ then $q(H)< q(T)$, so we lower the prob of stock moving up and raise the prob of the stock moving down such that the return is 1+r. The same argument holds for $r < \alpha$. 


\subsection{Girsanov's Theorem}
\noindent{\bf Define change of measure for continous variable}\\
For $(\Omega, F, P)$, given random variable Z with $E(Z) = 1$, define new probability measure
\begin{align}
	P^{'} = \int_A Z(\omega) dP(\omega)\\
\end{align}
We have two expectation defined, one is under P, the other under P^{'}, $E^{'}[X] = E[XZ]$, 
$dP^{'}(\omega) = Z(\omega) dP(\omega)$, and $Z(\omega) = \frac{d^{'}P(\omega)}{dP(\omega)}$\\
\noindent{\bf Define change of measure for filtration}\\
$E[Z] = 1$ and $Z(t) = E[Z|F(t)]$\\

\noindent{\bf Properties of Z(t)}\\
1) Martingale\\
Given $0 <= s <=t <= T$\\
$E[Z(t)|F(s)] = E[E[Z|F(t)]|F(s)] = E[Z|F(s)] = Z(s)$\\
2) $E^{'}[Y] = E[YZ(t)]$\\
$E^{'}[Y]=E[YZ] = E[E[YZ|F(t)]]=E[Y E[Z|F(t)]] = E[YZ(t)]$\\
3) Given $0<=s<=t<=T$, Y is F(t)- measurable, then\\
E^{'}[Y|F(s)] = \frac{1}{Z(s)} E[YZ(t)|F(s)]\\

\noindent{\bf Girsanov's Theorem}\\
Suppose w(t) is Brownian Motion given $\Omega, F, P$ F(t) is the filtration. Let $\Theta(t)$, $0<=t<=T$ is adapted process, define $Z(t) = exp(-\int_0^{t} \Theta(u) dW(u))$, and $W^{'}(t) = W(t) + \int_0^t \Theta(u) du$, s.t. $E[\int_0^T \Theta^2(u) Z^2(u) du < \infty]$. Then $E[Z]=1$, and under $P^{'}$, $W^{'}(t)$ is Brownian motion.\\

\subsection{Risk neutral measure in filtration for stock price}\\
We model the stock price using geometric Brownian motion\\
\begin{align*}
dS(t) = \alpha(t) S(t) dt + \sigma(t) S(t) d W(t)
\end{align*}
Its integrated form is\\
\begin{align*}
	S(t) = S(0) exp(\int _0 ^t \sigma(s) dW(s) + \int _0 ^t (\alpha(s) - \frac{1}{2} \sigma^2(s)) ds)\\
\end{align*}
define $D(t) = exp(-\int_0^t R(s) ds) $, then $dD(t) = R(t)D(t)dt$\\
Discounted stock price\\
\begin{align*}
	D(t)S(t) = S(0) exp(\int_0^t \sigma(s) dW(s) + \int_0^t (\alpha(s) - R(s) - \frac{1}{2} \sigma^2(s)) ds)
\end{align*}
\begin{align*}
	d(D(t) S(t)) = \sigma(t) D(t) S(t) [\Theta(t) dt + dW(t)]\\
\end{align*}
where $\Theta(t) = \frac{\alpha(t) -R(t)}{\sigma(t)}$
let $dW^{'}(t) = dW(t) + \theta(t) dt$, then $d(D(t)S(t)) = \sigma(t) D(t) S(t) dW^{'}(t)$, {\bf which implies martingale}, then $dS(t) = R(t) S(t) dt + \sigma(t) S(t) dW^{'}(t)$. The drift term now change from $\alpha$ to risk-free interest Rate R. Knowing $d(D(t)S(t))$ is a martingale, we can work out a little of math to show
\begin{align*}
	d(D(t)X(t)) = \Delta(t) d(D(t)S(t))
\end{align*}
This means $D(t)X(t)$ is a martingale under $P^{'}$. Suppose T is the option payoff time, then 
\begin{align*}
	D(t)X(t) = E^{'} [D(T) X(T) | F(t)]
\end{align*}
{\bf This is our risk-neutral pricing formula.}
\end{document}