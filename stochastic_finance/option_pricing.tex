\documentclass[a4paper]{article}
\usepackage{amssymb, amsmath}
\usepackage{graphicx}
\title{A Review of Option Pricing Models and Simulations}
\author{Shi Guo} 
\begin{document}
\maketitle
{\bf Abstract} In this article, we review the straties and a variety methods for pricing options. We first review the theory of the Black-Scholes-Merton model, the Heston model with stochastic volatility and the Bates model that contains diffusion with jumps. Then we discuss the Monte Carlo simulation algorithms of each model and how to treat the stopping time for American style options. Finally, we present the simulation result. \\
\line(1,0) {350}
\section{European Option Pricing Models}
\subsection{Black Scholes Merton Equation: Introduction}
We assume the stock prices following a geometric Brownian motion\\
1) Stock price:\\
\begin{align*}
	 dS(t) = \alpha S(t) dt + \sigma S(t) dW(t)\\
\end{align*}
 
\noindent 2) We have a portfolio X(t) which consists of $\Delta(t)$ share of stock  $\Delta(t) S(t)$, and $(X(t) - \Delta (t) S(t))$ money market account with interest rate r. \\  
\begin{align*}
	X(t) = \Delta(t) dS(t) + r(X(t) -\Delta(t) S(t)) dt\\
\end{align*}
\noindent 3) Change of the portfolio with respect to time
\begin{align*}
	  d X(t) & = \Delta(t) d S(t) + r(X(t) - \Delta(t) S(t)) dt \\
             & = r X(t) dt + \Delta(t) (\alpha - r )S(t) + \Delta (t) \sigma S(t) d W(t) \\
\end{align*}
                      
\noindent 4) Change of the present value of the stock with respect to time\\
\begin{align*}
	d(e^{-r t} S(t)) & = (\alpha - r) e^{- r t}S(t) dt + \sigma e^{- r t} S(t) dW(t) \\
\end{align*}

\noindent 5) With a few steps, we get change of the present value of the portfolio with respect to time\\
\begin{align*}
 & d(e^{- r t} X(t)) \\
 = & \Delta(t) (\alpha - r) e^{-rt} S(t) dt + \Delta(t) \sigma e^{-rt} S(t) dW(t)\\
\end{align*}

\noindent 6) Assume the option value is $c(t,S(t))$ and we apply Ito's formula\\
\begin{align*}
	& d(e^{-rt } c(t, S(t)) \\
= & e^{-rt} [- r c(t, S(t)) + c_t(t, S(t)) + \alpha S(t) \frac{\partial c(t, S(t))}{\partial S(t)}  +\frac{1}{2} \sigma^2 S^2(t) \frac{\partial^2 c(t, S(t))}{\partial S^2(t)} ] dt\\
  & +e^{-rt} \sigma S(t)  \frac{\partial c(t, S(t))}{\partial S(t)} dW(t)\\
\end{align*}
7)\noindent	Now equate Equation in 5) and 6), we get\\
dW(t) term:
\begin{align*}
	\Delta(t) = \frac{\partial c(t,S(t))}{\partial S(t)} \\	
\end{align*}
dt term:
\begin{align*}
	rc(t, S)  = c_t (t, S(t)) + r S(t) + \frac{1}{2} \sigma^2 S(t) \frac{\partial c(t, S(t))}{\partial S^2(t)}  \\
\end{align*}
which is known as Black-Scholes-Merton partial differential equation.\\
The terminal condition the equation satisfies for call option is
\begin{align*}
	c(T,S) = (S(T)-K)^+
\end{align*}
Similarly, for put option
\begin{align*}
	p(T,S) = (K-S(T))^+
\end{align*}
\subsection{Connection to Faynman-Kac formula}
In risk-neutral measure, we write the stock price as
\begin{align*}
	dS(t) = rS(t) dt + \sigma S(t) d \tilde W(t) 
\end{align*}
Where $\tilde W(t)$ is a standard Brownian motion under risk-neutral measure.\\
According to the risk-neutral pricing formula, the price of the derivative security at time t is
\begin{align}
	V(t) = \tilde E[e^{-r(T-t)}V(T)| F(t)] = \tilde E[e^{-r(T-t)}h(S(T))| F(t)] \label{rn}
\end{align}
Since the stock price is Markov and the payoff is a function fo the stock price alone, based on Faynman-Kac formula, there is a function $v(t,x)$ such that $V(t) = v(t, S(t))$, and $v(v, S(t))$ must satisfy discounted partial differential equation 
\begin{align*}
	v_t(t,x) + rx v_x(t,x) + \frac{1}{2} \sigma^2 x^2 v_{xx}(t,x) = rv(t,x)
\end{align*}
Now we have seen two ways of showing the Black-Scholes-Merton(BSM) equation. 
One way is to reproduce the payoff of the option using a portfolio that consists of a 
saving account. Another way is based on the risk-neutral pricing formula and Feynman-Kac formula. 
These two ways are equivalent. Because under risk-neutral measure, the payoff of a derivative is the same as a saving account, which imply we are able to reproduce the payoff using portfolio that consisting of a saving account.
\subsection{Black-Scholes-Merton Model: Analytic Solution for European Option}
{\bf1. European call option}\\
For European call option with payoff to be $V(T) = S(T) - K$, with K as strike price, let us assume constant volatility $\sigma$, and constant interest rate r. 
Then we can obtain the solution to the BSM equation with martingale property without bothering solving the complex parital
differential equation. The call option value satisfies
\begin{align*}
	c(t, S(t)) = \tilde E[e^{-r(T-t)}(S(T)-K)^+|F(t)]
\end{align*}
We write
\begin{align*}
	S(T) = S(t) exp\{\sigma(\tilde W(T) -\tilde W(t)) + (r - \frac{1}{2} \sigma^2) \tau\} \\
	     = S(t) exp\{-\sigma \sqrt{\tau} Y + (r - \frac{1}{2}\sigma^2)\tau\}
\end{align*}
Where Y is the stardard normal random variable and $\tau = T - t$ is the time to expiration. 
\begin{align*}
        Y = -\frac{\tilde W(T) - \tilde W(t)}{\sqrt{T - t}}
\end{align*}
So we see that $S(T)$ is the product of the $F(t)$ measurable random variable S(t) and random variable
\begin{align*}
        exp\{-\sigma \sqrt{\tau} Y +(r - \frac{1}{2} \sigma^2) \tau \}
\end{align*}
Which is independent of $F(t)$.
Therefore based on risk-neutral pricing formula[\ref{rn}]
\begin{align*}
        c(t,x) = \tilde E[e^{-r\tau}(x exp\{ -\sigma \sqrt{\tau} Y + (r - \frac{1}{2} \sigma^2)\tau\} -K)^+]\\
        = \frac{1}{\sqrt{2\pi}} \int_{-\infty}^{\infty} e^{-\tau r}(x exp\{-\sigma \sqrt{\tau} y
        +(r -\frac{1}{2}\sigma^2)\tau \} -K)^+ e^{-\frac{1}{2}y^2} dy
\end{align*}
After a little bit of math with integration, we have the solution to the Black-Scholes-Merton model for European call option
\begin{align*}
	c(\tau,x; K, r, \sigma) = xN(d_1(\tau,x)) - e^{-r\tau}KN(d_2(\tau,x))
\end{align*}
Where
\begin{align*}
	d_1 = \frac{1}{\sigma \sqrt{\tau}}[ln(\frac{S_t}{K}) + (r+ \frac{\sigma^2}{2})\tau] \\
	d_2 = d_1 - \sigma \sqrt{\tau}
\end{align*}
N() is the cumulative distribution function of the standard normal distribution
{\bf 2. European put option}\\
The payoff for the European put option is $V(T) = K - S(T)$, we can follow a similiar derivation and get the formula for put option
\begin{align*}
	p(\tau,x; K, r, \sigma) = Ke^{-r\tau}N(-d_2)-xN(-d_1)
\end{align*}
{\bf 3. Options based on dividend-paying stock}\\
The value of dividend-paying stock consists of two parts. One is the value of the stock, the other one is the value of dividend. Suppose the dividend rate is a, then the change of value of dividend-paying stock over time dt is $dS(t) + aS(t)dt$. In order to make the discounted value a martingale, we require 
\begin{align*}
	\frac{dS(t) + aS(t)dt}{S(t)} = rdt + \sigma \tilde dW(t)
\end{align*}
so
\begin{align*}
	dS(t) = (r-a) S(t)dt + \sigma S(t) \tilde dW(t)
\end{align*}
\begin{align*}
	S(T) = S(t) exp\{\sigma(\tilde W(T) -\tilde W(t)) + (r - a - \frac{1}{2} \sigma^2) \tau\} \\
\end{align*}
Following similar derivation, for call option we can get
\begin{align*}
	c(\tau,x; K, r, \sigma) = xe^{-a\tau} N(d_1(\tau,x)) - e^{-r\tau} K N(d_2(\tau,x))
\end{align*}
\begin{align*}
	p(\tau,x; K, r, \sigma) = Ke^{-r\tau} N(-d_2)-xe^{-a \tau}N(-d_1)
\end{align*}
{\bf 4. Boundary conditons}\\
Using the solution $c(t,x)$ and $p(t,x)$, we can easily check the boundary conditions when
time t approaches to expiration time T.\\
As we know
\begin{align*}
	d_1 = \frac{1}{\sigma \sqrt{\tau}}ln(\frac{S}{K}) 
	+ \frac{1}{\sigma}(r+\frac{\sigma^2}{2}\sqrt{\tau})
\end{align*}
When $\tau \to 0$, the second term decays much faster, so it vanishes. When $S>K$, $d_1$ goes to infinity, when $S<K$, $d_1$ goes to negative infinity.
Therefore, when $S>K$
\begin{align*}
	c(t,x) = S*N(+\infty) - K*N(+\infty) = S-K
\end{align*}
Therefore, when $S<K$
\begin{align*}
	c(t,x) = S*N(-\infty) - K*N(-\infty) = 0-0 = 0
\end{align*}
{\bf 5. Examples}\\
The following graphs show the change of option price with respect to different parameters.
\\
\\
\\
\\
\includegraphics[scale = 0.5]{option_price1.eps}\\
\includegraphics[scale = 0.5]{option_price2.eps}\\
{\bf 6. Alternative formulation}\\
If we introduce $F = e^{(r-a)\tau} S$, which is the forward price of the asset S. Then the equation pricing equation becomes
\begin{align*}
	C(F,\tau)= D[N(d_+)F - N(d_-)K] \\
	P(F,\tau)= D[N(-d_-)K - N(-d_+)F] \\
	d_{+/-} = \frac{1}{\sigma \sqrt{\tau}}[ln(\frac{F}{K})+/-\frac{1}{2}\sigma^2 \tau] \\
\end{align*}
The variables are:\\
$\tau = T - t$ is the time to expiry\\
$D = e^{-r\tau}$ is the discount factor \\
\subsection{Heston Stochasic Volatility Model}
The Black-Scholes equation assumes the volatility is constant, which is the ideal case and not practical in the real market. The Heston model assumes the volatility to follow a stochastic process. The following content refers to the paper: "A Closed-Form Solution for Options with Stochastic Volatility with Applications to Bond and Currency Options" by Steven Heston. Suppose a stock price under risk-neutral measure is governed by
\begin{align}
	dS(t) = rS(t)dt + \sqrt{V(t)} S(t) \tilde dW_1(t)
\end{align}
and the volatility itself is governed by the equation
\begin{align}
	dV(t) = (a -bV(t))dt + \sigma \sqrt{V(t)} \tilde dW_2(t)
\end{align}
Where 
\begin{align*}
	\tilde dW_1(t) \tilde dW_2(t) = \rho dt
\end{align*}
At time t, the risk-neutral price of a call expiring at time $T \geq t$ in this model is
\begin{align*}
	c(t, S(t), V(t)) = \tilde E[e^{-r(T-t)}(S(T)-K)^+|F(t)]
\end{align*}
If we move the term $c^{rt}$ to the left hand side, we see
\begin{align}
	e^{-rt}c(t, S(t), V(t)) = \tilde E[e^{-rT}(S(T)-K)^+|F(t)]
\end{align}
which satisfies the martingale property.
Then we take the differentiation of $e^{-rt}c(t, S(t), V(t))$. We get
\begin{align*}
	& d(e^{-rt}c(t, S(t), V(t))\\
	= & \frac{\partial e^{-rt}}{\partial t} c(t, S(t), V(t)) 
	+ e^{-rt}\frac{\partial c(t, S(t), V(t))}{\partial t} dt \\
	= & -re^{-rt}c(t, S(t),V(t))dt \hspace{2mm} (1)\\
	+ & e^{-rt} \frac{\partial c}{\partial t} dt \hspace{2mm} (2) \\
	+ & e^{-rt} \frac{\partial c}{\partial S} dS \hspace{2mm} (3) \\
	+ & e^{-rt} \frac{\partial^2 c}{\partial S^2} dS dS \hspace{2mm} (4)\\
	+ & e^{-rt} \frac{\partial c}{\partial V} dV \hspace{2mm} (5) \\
	+ & e^{-rt} \frac{\partial^2 c}{\partial V^2} dV dV\hspace{2mm} (6)\\
	+ & e^{-rt} \frac{\partial^2 c}{\partial V \partial S} dV dS\hspace{2mm} (7)
\end{align*}
As we are interested in only the dt terms, we find out the dt terms from (1) to (7)
the dt term in (1) is
\begin{align*}
	-rc(t, S(t), V(t))e^{-rt}dt
\end{align*}
the dt term in (2) is
\begin{align*}
	\frac{\partial c}{\partial t}e^{-rt}dt
\end{align*}
the dt term in (3) is
\begin{align*}
	\frac{\partial c}{\partial S}rSe^{-rt}dt
\end{align*}
the dt term in (4) is
\begin{align*}
	\frac{1}{2} \frac{\partial^2 c}{\partial S^2}VS^2e^{-rt}dt
\end{align*}
the dt term in (5) is
\begin{align*}
	\frac{\partial c}{\partial V}(a-bV(t))e^{-rt}dt
\end{align*}
the dt term in (6) is
\begin{align*}
	\frac{1}{2} \frac{\partial^2 c}{\partial V^2}V\sigma^2e^{-rt}dt
\end{align*}
the dt term in (7) is
\begin{align*}
	\frac{\partial^2 c}{\partial V \partial S} VS\sigma e^{-rt}dt
\end{align*}
Collect all the dt terms and let those terms equal to zero, we get
\begin{align}\label{c}
	c_t + rs c_s + (a-bv)c_v + \frac{1}{2} s^2 v c_{ss} + \rho \sigma svc_{sv}
	+\frac{1}{2} \sigma^2vc_{vv} = rc
\end{align}
The function $c(t, s, v)$ satisfies boundary condition
\begin{align*}
	c(T, s, v) = (s-K)^+ \\
	c(t, 0, v) = 0 \\
	c(t, s, 0) = (s - e^{-r(T-t)}K)^+\\
	lim_{s \to \infty} \frac{c(t, s, v)}{s-K} = 1 \\
	lim_{v \to \infty} c(t, s, v) = s\\  
\end{align*}
Based on the solution to the BSM equation, we can guess that the solution has the following form
\begin{align}
	c(t,s,v) = sf(t, logs, v) - e^{-r(T-t)}Kg(t,logs,v)
\end{align}	
Where f and g can be intepreted as a cumulative distribution function.
Then since c(t,s,v) satisfies the partial differential equation \ref{c}, we can show that $f$ and $g$ satisfy the following
\begin{align} \label{f}
	f_t + (r + \frac{1}{2}v)f_x + (a-bv+\rho \sigma v)f_v 
	+ \frac{1}{2}  v f_{xx} + \rho \sigma vf_{xv}
        +\frac{1}{2} \sigma^2 vf_{vv} = 0\\
\end{align}
\begin{align}
	g_t + (r - \frac{1}{2}v)g_x + (a-bv)g_v 
	+ \frac{1}{2}  v g_{xx} + \rho \sigma vg_{xv}
        +\frac{1}{2} \sigma^2 vg_{vv} = 0\\
\end{align}
The derivation is straightforward but one needs to keep in mind here we treat x and v as two independent variables. The above PDE for f and g satisfy boundary condition
\begin{align*}
	f(T, x, v) = 1_{x\geq logK}\\
	g(T, x, v) = 1_{x\geq logK}\\
\end{align*}
This implies that f and g can be intepreted as "Probabilities". We can define
\begin{align*}
	f(t, x, v) = E^{t,x,v}1_{x\geq logK}\\
\end{align*}
We suppose a pair of stochastic process X(t), V(t) given by the following expression
\begin{align*}
	dX(t) = (r+\frac{1}{2}V(t))dt + \sqrt{V(t)}dW_1(t)
	dV(t) = (a-bV(t)+ \rho \sigma V(t))dt + \sigma \sqrt{V(t)}dW_2(t)
\end{align*}
By F-K formula, we can show that f satisfies the PDE above. Similarly, we have
\begin{align*}
	g(t, x, v) = E^{t,x,v} 1_{x\geq logK}\\
\end{align*}
and the stochastic process of X(t) and V(t) are
\begin{align*}
	dX(t) = (r-\frac{1}{2}V(t))dt + \sqrt{V(t)}dW_1(t)\\
	dV(t) = (a-bV(t))dt + \sigma \sqrt{V(t)}dW_2(t)
\end{align*}
To find the analytical solution of f(t, x, v) and g(t, x, v) is not an easy task. Instead, we do Fourier transform of f and g. 
First we work out the function f(t, x, v. ). Let $\tau = T-t$
\begin{align*}
	\tilde f(k, v, \tau) = \int_{-\infty}^{\infty}dx e^{-ikx}f(x, v, \tau)
\end{align*}
The inverse Fourier transform is
\begin{align*}
	f(x, v, \tau) = \int_{-\infty}^{\infty}\frac{dk}{2\pi} e^{ikx}
	\tilde f(k, v, \tau)
\end{align*}
Substitute this into equation \ref{f}, then
\begin{align} \label{ff}
	-\frac{\partial \tilde f}{\partial \tau}
	+ (r + \frac{1}{2})ik \tilde f 
	+ (a-bv+\rho \sigma v)\frac{\partial \tilde f}{\partial v}
        -\frac{1}{2}vk^2 \tilde f 
	+ \rho \sigma v ik \frac{\partial \tilde f}{\partial v}
        +\frac{1}{2} \sigma^2 v \frac{\partial^2 \tilde f}{\partial v^2}  = 0\\
\end{align}
Now the problem is to solve for $\tilde f$. We note when $\tau=0$,
\begin{align*}
	\tilde f(k, v, 0) = \int_{-\infty}^{\infty}dx e^{-ikx}f(x, v, \tau=0)
			  = \int_{-\infty}^{\infty}dx e^{-ikx}1_{x \geq logK}
			     = \int_{0}^{\infty}dx e^{-ikx}
			     = \pi \delta(k) + \frac{1}{ik}
\end{align*}
when $\tau!=0$, we guess a general solution which has the following form
\begin{align*}
	\tilde f(k, v, \tau) = exp(C\tau +D\tau v) \tilde f(k,v,0)
\end{align*}
From above we easily see it match the terminal condition at $\tau \to 0$. With inverse Fourier transform
\begin{align} \label{fs}
	f(x, v, \tau, x) & = \int_{-\infty}^{\infty}\frac{dk}{2\pi} e^{ikx} \tilde f(k, v, \tau)\\
			 & = \int_{0}^{\infty}\frac{dk}{\pi} e^{ikx} exp(C\tau + D\tau v) \tilde f(k, v, 0)\\
			 & = \int_{0}^{\infty}\frac{dk}{\pi} e^{ikx} exp(C\tau + D\tau v) 
			 	(\pi \delta(k) + \frac{1}{ik}) \\
			 & = \frac{1}{2} + \int_{0}^{\infty}\frac{dk}{\pi} e^{ikx} exp(\frac{C\tau + D\tau v}{ik}) 
\end{align}
Now the only remaining task is to find C and D. If we substitute \ref{fs} into \ref{ff}, we can get the expression C and D. 
\subsection{Jump Diffusion Model}
The BSM model and the Heston model assumes the asset prices follows the stochastic process driven by Brownian motion. Under this assumption, the asset price is continuous in time. The jump diffusion model extends the BSM model and Heston model by assuming asset price has discrete jump in time. Therefore in jump diffusion model, the stochastic process contains a continuous Brownian motion process and a discrete jump process.\\

\noindent{\bf 1. Number of jumps during per unit time: Possion distribution}\\
The number of jump should be either zero or a positive integer. Let N be the number of jump per unit of time and k be a nonnegative integer. Then the probability of $N = k$ should satisfies
\begin{align*}
	\sum_{k = 0}^\infty P(N = k) = 1
\end{align*}
In order to find the distribution, we borrow the idea of Taylor's expansion. We consider a function $e^{\lambda}$, its Taylor's expansion is
\begin{align*}
	e^{\lambda} = \sum_{k = 0}^\infty \frac{\lambda^k}{k!}
\end{align*}
Dividing $e^{\lambda}$ on both sides, we can
\begin{align*}
	1 = \sum_{k = 0}^\infty \frac{\lambda^k}{k!}e^{-\lambda} 
\end{align*}
So we can assign the probability
\begin{align*}
P(N = k) = \frac{\lambda^k}{k!}e^{-\lambda}
\end{align*}
This is the well-known Possion distribution. It is easy to show Possion distribution has mean and variance equal to $\lambda$. So $\lambda$ can be intepreted as the average number of jumps per unit of time.\\

\noindent{\bf 2. Number of jumps during time interval $\Delta t$: Possion process}\\
Now we consider the number of jumps in a given time interval $\Delta t$, and we call it $\Delta N$. The average jump during $\Delta t$ is $\lambda \Delta t$. So we can get the probability distribution by substitude $\lambda \Delta t$ to $\lambda$
\begin{align*}
	P(\Delta N = k) = \frac{(\lambda \Delta t)^k}{k!}e^{-\lambda \Delta t}
\end{align*}
The mean of $\Delta N$ is $\lambda \Delta t$.\\

Let $\Delta t$ be really small time interval $dt$, and from above we see
\begin{align*}
	P(dN = 0) = e^{-\lambda dt} \approx 1 - \lambda dt \\
	P(dN = 1) = \lambda dt e^{-\lambda dt} \approx \lambda dt (1 - \lambda dt) \approx \lambda dt \\
	P(dN = 2) = (-\lambda dt)^2 e^{- \lambda dt} \approx 0 \\
\end{align*}
This makes perfectly sense, as during an infinitesimal time interval, the jump can either not happen or happen only once.\\

We define $S_k$ is the time when the kth jump occurs and we assume $S_0 = 0$. Then the Possion process is
\begin{align*}
	N(t) = k  \text{ if } S_k \leq t < S_{k+1}
\end{align*}
With $E[N(t)] = \lambda t$.
Note the Poisson process defined above is right-continuous.\\

\noindent{\bf 3. Compensated Poisson process}\\
The Poisson process has a mean that is a function of time. As time evolves, the process has a non-zero drift so it is not a martingale. We define a compensated Poission process $M(t)$ which obeys the martingale property
\begin{align*}
	M(t) = N(t) - \lambda t \\
	E[M(t)|F(s)] = M(s)
\end{align*}

\noindent{\bf 4. Compound Poisson process}\\
In the above definition, we assume the jump size in the Possion process is 1. Now we allow the jump size to be random. Let N(t) be a Poisson process with mean $\lambda$ and $Y_i$ be a sequence of random variable with mean $E[Y] = m$. We define compound Poisson process
\begin{align*}
	Q(t) = \sum_{i=1}^{N(t)} Y_i
\end{align*}
The number of jumps follows Poisson distribution, and at each jump, the size of the jump is $Y_i$. In other words, the first jump size is $Y_1$ and the second jump size is $Y_2$, etc.

The mean of the compound Poisson process is the product of the number of jumps and the size of the jumps, as these two are independent.
\begin{align*}
	EQ(t) = m \lambda t
\end{align*}

We can also define a compensated compound Poisson process
\begin{align*}
	\tilde M(t) = Q(t) - m \lambda t
\end{align*}
And this is a martingale.\\

\noindent{\bf 5. Geometric Possion process with constant jump size}\\
In the compound Possion process, let the jump size be a constant $y$. When jump occurs, we have 
\begin{align*}
	Q(t_i) = Q(t_{i-}) + y 
\end{align*}
The asset value in the financial market follows the geometric Possion process. Let S(t) be the asset price, let
\begin{align*}
	& S(t_i) = S(t_{i-}) y \\
	& \frac{S(t_i) - S(t_{i-})}{S(t_{i-})} =  y - 1 \\
	&  \Delta S(t_i) =  (y - 1) S(t_{i-}) \Delta N(t)  \\
\end{align*}
Where $\Delta N(t) = 1$ when jump occurs, $0$ otherwise. When $N(t) = 1$, there is one jump from time zero up to t, so $S(t) = yS(0)$. When $N(t) = 2$, $S(t) = y^2 S(0)$. So for any abitrary $N(t)$
\begin{align*}
	S(t) = S(0)y^{N(t)} 
\end{align*}
In order to fulfill the martingale condition, we write
\begin{align*}
	dS(t) = (y - 1)S(t) dN(t) - (y - 1) \lambda S(t) dt
\end{align*}
Its integral form is
\begin{align*}
	S(t) = S(0)y^{N(t)} exp(- (y - 1) \lambda t) 
\end{align*}
\noindent{\bf 6. Geometric Poisson process with log-normal jump size: Bates jump diffusion model}\\
If the jump size Y is random in the geometric Poisson process, then to model the asset price, we need to modify
the last equation by changing $y^{N(t)}$ to $\Pi_{i=1}^{N(t)} Y_i$, and $y - 1$ to E[Y] - 1. 
\begin{align*}
	S(t) = S(0)\Pi_{i=1}^{N(t)} Y_i exp(- (E[Y] - 1) \lambda t) 
\end{align*}
We need to model the jump size $Y_i$. $Y_i$ is a strictly positive random variable so a good candidate is the exponential of normal random variable which is log-nomal random variable. If we have a normal distributed random variable $J \sim N(\nu, \delta^2)$, and let $Y$ be $e^{J}$. Then $Y$ follows log-normal distribution with mean $e^{\nu + \frac{1}{2}\delta^2}$. We can write S(t) using $J$
\begin{align*}
	S(t) = S(0) exp{\sum_i^{N(t)} J_i} exp(-(e^{\nu + \frac{1}{2}\delta^2}-1) \lambda t) 
\end{align*}
Let $Z = \sum_i^{N(t)} J_i$, which is a sum of normal random variable. Then $Z \sim Normal(N(t)\nu, N(t)\delta^2)$. So
\begin{align*}
	S(t) = S(0) exp{(Z -(e^{\nu + \frac{1}{2}\delta^2}-1) \lambda t)} 
\end{align*}
Where $ Z \sim Normal(N(t)\nu, N(t)\delta^2)$ given $N(t) \sim Poisson(\lambda t)$.\\
\section{Monte Carlo Simulation Processes}
In simulation, we decompose the time evolvement of an asset price into three components: drift, diffusion, jump. The
drift term is dt term, the diffusion term has Brownian motion component, the jump term has Possion component. Let us look at the following examples. \\ 
1. BSM process\\
In BSM process, the asset price follows
\begin{align*}
	S(t) = S(0)exp((r - \frac{1}{2} \sigma^2 ) dt + \sigma dW(t)) \\
\end{align*}
The drift term is clearly $r - \frac{1}{2} \sigma^2$ for any given dt.  For the diffusion term, we need to generate W(t).
We know the W(t) is a Normal random variable with mean 0 and variance t. So suppose we generate a Gaussian random variable $\hat W$, it turns out $W = \sqrt{t} \hat W$ is normally distributed with variance t. 
\begin{align*}
	& \mu = r - \frac{1}{2} \sigma^2\\
	& S^{(t + dt)} = S^{(t)}exp(\mu dt + \sigma \sqrt{dt} \hat W) \\
\end{align*}
2. Heston process\\
The Heston model states
\begin{align*}
        dS(t) = rS(t)dt + \sqrt{V(t)} S(t) \tilde dW_1(t) \\
        dV(t) = (a -bV(t))dt + \sigma \sqrt{V(t)} \tilde dW_2(t) \\
        \tilde dW_1(t) \tilde dW_2(t) = \rho dt \\
\end{align*}
{\bf a. Generating two Brownian motions with correlation. }\\
To simulate Heston process needs to generate two Gaussian random variables that has correlation $\rho$, we do this by first generate two independent Gaussian random variables $W_1$, $W_2$, then
\begin{align*}
	W_1^{'} = \hat W_1 \\
	W_2^{'} = \rho \hat W_1 + \sqrt{1 - \rho^2} \hat W_2 \\
\end{align*}
We can easily check the mean of $W_2^{'}$
\begin{align*}
	&E[W_2^{'}] = \rho E[W_1] + \sqrt{1 - \rho^2} E[W_2] = 0\\
	&Var[W_2^{'}] = \rho^2 Var[W_1] + (1 - \rho^2) Var[W_2] = \rho^2 + 1 - \rho^2 = 1 \\
\end{align*}
\begin{align*}
	cor(W_1^{'}, W_2^{'}) & = \frac{E[W_1^{'}W_2^{'}]}{\sqrt{Var[W_1^{'}]}\sqrt{Var[W_2^{'}]}}\\
		      & = E[W_1 (\rho W_1 +\sqrt{1 - \rho^2})W_2] \\
		      & = \rho E[W_1^2] + \sqrt{1 - \rho^2} E[W_1 W_2] \\
		      & = \rho \\
\end{align*}
The last step uses the fact $W_1$ and $W_2$ are independent, so $E[W_1 W_2] = E[W_1] E[W_2] = 0$.\\
{\bf b. Coping with negative V(t). } \\
The Heston model setting provides a non-negative V(t). This can be seen when V(t) is positively moving close to zero, the drift term $adt$ pulls the V(t) up and avoid $V(t)$ to cross zero. However, in discretized version $V(t)$ may attain negative values. There exists several methods to deal with this issue.\\
(1) Absorption: Keep positive part of the previous $V(t)$ and use it to calculate next step $V(t+dt)$ and X(t+dt)\\
\begin{align*}
	V^{(t + dt)} = V^{+(t)} + (a - b V^{+(t)}) \sqrt{dt} + \sigma \sqrt{V^{+(t)}} dt (\rho \hat W_1
	+ \sqrt{1 - \rho^2} \hat W_2)
\end{align*}
(2) Relection: Keep the absolute value of the previous $V(t)$ and use it to calculate next step $V(t+dt)$ and X(t+dt)\\
\begin{align*}
	V^{(t + dt)} = |V^{(t)}| + (a - b |V^{(t)}|) \sqrt{dt} + \sigma \sqrt{|V^{(t)}|}  dt (\rho \hat W_1
	+ \sqrt{1 - \rho^2} \hat W_2)
\end{align*}
(3) Partial truncation: Using the posivive part in diffusion term of $V(t+dt)$ and in $X(t + dt)$\\
\begin{align*}
	V^{(t + dt)} = V^{(t)} + (a - b V^{(t)}) \sqrt{dt} + \sigma \sqrt{V^{+(t)}} dt (\rho \hat W_1
	+ \sqrt{1 - \rho^2} \hat W_2)
\end{align*}
(4) Full truncation: Using the positive part in drift and diffusion terms of $V(t + dt)$, and in $X(t + dt)$. 
This scheme is proved to have the lowest bias. Reference: A comparison of biased simulation schemes for stochastic volatility models by Roger Lord, Quantitative Finance,2010.
\begin{align*}
	V^{(t + dt)} = V^{(t)} + (a - b V^{+(t)}) \sqrt{dt} + \sigma \sqrt{V^{+(t)}} dt (\rho \hat W_1
	+ \sqrt{1 - \rho^2} \hat W_2)
\end{align*}
Together with formula for the drift and asset price, we write
\begin{align*}
	& \mu = r - \frac{1}{2} (V^{+(t)})\\
	& S^{(t + dt)} = S^{(t)}exp(\mu dt + \sqrt{V^{+(t)}} \sqrt{dt} W_1) \\
	& V^{(t + dt)} = V^{(t)} + (a - b V^{+(t)}) \sqrt{dt} + \sigma \sqrt{V^{+(t)}} dt (\rho \hat W_1
	+ \sqrt{1 - \rho^2} \hat W_2) 
\end{align*}        
3. Bates process\\
\begin{align*}
	& \mu = r - \frac{1}{2} (V^{+(t)}) - \lambda (e^{\nu + \frac{1}{2}\delta^2}  -1) \\
	& S^{(t + dt)} = S^{(t)}exp(\mu dt  + \sqrt{V^{+(t)}} \sqrt{dt} dW_1 
	   + (\nu_*N+\delta^2*\sqrt{N})*dW_3)\\
	& V^{(t + dt)} = V^{(t)} + (a - b V^{+(t)}) dt + \sigma \sqrt{V^{+(t)}} dt (\rho \hat W_1
	+ \sqrt{1 - \rho^2} \hat W_2 )
\end{align*}
\section{Simulation of Stopping Time for American Option}
American Option can be exercised at any time. So in Monte Carlo Simulation, for each path we need to determine the stopping time, in other words, time for the option to be exerciesed. LongStaff\cite{americanoption} proposed a algorithm to determine the stopping time and here we briefly outline the algorithm\\
Let us assume the option has a payoff function $h(S, K)$, where K is strike price.
Suppose we have M samples, the current asset price at time i is $S^{(m)}_i$, where $0 \geq i \leq T$. The discount factor $D_{i,j}$ means discount the asset price at time j to time t. For example $D_{13} = e^{-(3-1)r} = e^{-2r}$. 
For each sample, the asset price forms a price path from $i = 0$ to $T$, and we define a stopping time $\tau^{(m)}$ for each path. \\
(1) set $\tau^{(m)}$ to T for each sample path m.\\
(2) for t from T - 1 to 1:\\
\indent  for each sample:\\
\indent  (2b) calculate discount payoff value: $DV^{(m)} = D_{t, \tau} h(S^{(m)}_\tau, K) $\\
\indent  (2c) calculate the current payoff value: $h(S^{(m)}_t)$\\
\indent	 (2d) regress $DV$ on ${S_t}$ using least square and get estimated \\
\indent      discount payoff value $EDV = f(X_t)$\\
\indent	 (2e) if $h(S^{(m)}_t) > EDV^{(m)}(X^{(m)}_i)$, \\
\indent       this means we can exercise the option, then upate $\tau^{(m)} = t$.\\
(3) for each path calculate the present value of the payoff at stopping time $\tau$ \\
$PV^{(m)} = D_{t, \tau} h(S^{(m)}_\tau, K)$\\
(4) average $PV^{(m)}$ to get the price of the option\\
\section{Result}
{\bf Simulation of option value with BSM process using both analytical(ANAL) and Monte Carlo(MC)}\\
We calcuate value of  option of different type and style listed below. For MC method, we choose 100 timesteps and 100000 sample paths. For Eurpoean style, we also compare the result with the analytical one.
\begin{table}
	\begin{tabular} { l  c  c   c  c  c       c       c          }
		Style 	 & Type & S  & K & r & $\sigma$ & NPV(ANAL) & NPV(MC)\\
	\hline 
		European & Call & 36.0 & 40 & 0.06 & 0.2 & 4.28 & 4.27(3) \\
		American & Call & 36.0 & 40 & 0.06 & 0.2 & N/A  & 4.28(3) \\
		European & Put  & 36.0 & 40 & 0.06 & 0.2 & 3.76 & 3.76(2) \\
		American & Put  & 36.0 & 40 & 0.06 & 0.2 & N/A  & 4.81(3) \\
	\hline
	\end{tabular}\\
	{\raggedright S: Underlying Asset Price, K: strike price, r: interest rate, 
	$\sigma$: volatility}
\end{table}
\begin{thebibliography}{9}
\bibitem{americanoption} 
Francis Longstaff, Eduardo Schwartz 
\textit{The Review of Financial Studies}. 
Spring 2001 Vol. 14, No. 1, pp. 113-147.
\end{thebibliography}
\end{document}
