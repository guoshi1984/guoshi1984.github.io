\documentclass[a4paper]{article}
\usepackage{amssymb, amsmath}
\usepackage{graphicx}
\begin{document}
Author: Dr. Shi Guo  \hspace{30mm} Email: guoshi1984@hotmail.com\\
\line(1,0){350}
\section{Black Scholes Merton Equation: Introduction}
We assume the stock prices following a geometric Brownian motion\\
1) Stock price:\\
\begin{align*}
	 dS(t) = \alpha S(t) dt + \sigma S(t) dW(t)\\
\end{align*}
 
\noindent 2) We have a portfolio X(t) which consists of $\Delta(t)$ share of stock  $\Delta(t) S(t)$, and $(X(t) - \Delta (t) S(t))$ money market account with interest rate r. \\  
\begin{align*}
	X(t) = \Delta(t) dS(t) + r(X(t) -\Delta(t) S(t)) dt\\
\end{align*}
\noindent 3) Change of the portfolio with respect to time
\begin{align*}
	  d X(t) & = \Delta(t) d S(t) + r(X(t) - \Delta(t) S(t)) dt \\
             & = r X(t) dt + \Delta(t) (\alpha - r )S(t) + \Delta (t) \sigma S(t) d W(t) \\
\end{align*}
                      
\noindent 4) Change of the present value of the stock with respect to time\\
\begin{align*}
	d(e^{-r t} S(t)) & = (\alpha - r) e^{- r t}S(t) dt + \sigma e^{- r t} S(t) dW(t) \\
\end{align*}

\noindent 5) With a few steps, we get change of the present value of the portfolio with respect to time\\
\begin{align*}
 & d(e^{- r t} X(t)) \\
 = & \Delta(t) (\alpha - r) e^{-rt} S(t) dt + \Delta(t) \sigma e^{-rt} S(t) dW(t)\\
\end{align*}

\noindent 6) Assume the option value is $c(t,S(t))$ and we apply Ito's formula\\
\begin{align*}
	& d(e^{-rt } c(t, S(t)) \\
= & e^{-rt} [- r c(t, S(t)) + c_t(t, S(t)) + \alpha S(t) \frac{\partial c(t, S(t))}{\partial S(t)}  +\frac{1}{2} \sigma^2 S^2(t) \frac{\partial^2 c(t, S(t))}{\partial S^2(t)} ] dt\\
  & +e^{-rt} \sigma S(t)  \frac{\partial c(t, S(t))}{\partial S(t)} dW(t)\\
\end{align*}
7)\noindent	Now equate Equation in 5) and 6), we get\\
dW(t) term:
\begin{align*}
	\Delta(t) = \frac{\partial c(t,S(t))}{\partial S(t)} \\	
\end{align*}
dt term:
\begin{align*}
	rc(t, S)  = c_t (t, S(t)) + r S(t) + \frac{1}{2} \sigma^2 S(t) \frac{\partial c(t, S(t))}{\partial S^2(t)}  \\
\end{align*}
which is known as Black-Scholes-Merton partial differential equation.
\section{Connection to Faynman-Kac formula}
In risk-neutral measure, we write the stock price as
\begin{align*}
	dS(t) = rS(t) dt + \sigma S(t) d \tilde W(t) 
\end{align*}
Where $\tilde W(t)$ is a standard Brownian motion under risk-neutral measure.\\
According to the risk-neutral pricing formula, the price of the derivative security at time t is
\begin{align*}
	V(t) = \tilde E[e^{-r(T-t)}V(T)| F(t)] = \tilde E[e^{-r(T-t)}h(S(T))| F(t)]
\end{align*}
Since the stock price is Markov and the payoff is a function fo the stock price alone, based on Faynman-Kac formula, there is a function $v(t,x)$ such that $V(t) = v(t, S(t))$, and $v(v, S(t))$ must satisfy discounted partial differential equation 
\begin{align*}
	v_t(t,x) + rx v_x(t,x) + \frac{1}{2} \sigma^2 x^2 v_{xx}(t,x) = rv(t,x)
\end{align*}
Now we have seen two ways of showing the Black-Scholes-Merton(BSM) equation. 
One way is to reproduce the payoff of the option using a portfolio that consists of a 
saving account. Another way is based on the risk-neutral pricing formula and Feynman-Kac formula. 
These two ways are equivalent. Because under risk-neutral measure, the payoff of a derivative is the same as a saving account, which imply we are able to reproduce the payoff using portfolio that consisting of a saving account.
\section{Black-Scholes-Merton Model: Analytic Solution for European option}
{\bf1. European call option}\\
For European call option with payoff to be $V(T) = S(T) - K$, with K as strike price, let us assume constant volatility $\sigma$, and constant interest rate r. 
Then we can obtain the solution to the BSM equation with martingale property without bothering solving the complex parital
differential equation. The call option value satisfies
\begin{align*}
	c(t, S(t)) = \tilde E[e^{-r(T-t)}(S(T)-K)^+|F(t)]
\end{align*}
We write
\begin{align*}
	S(T) = S(t) exp\{\sigma(\tilde W(T) -\tilde W(t)) + (r - \frac{1}{2} \sigma^2) \tau\} \\
	     = S(t) exp\{-\sigma \sqrt{\tau} Y + (r - \frac{1}{2}\sigma^2)\tau\}
\end{align*}
Where Y is the stardard normal random variable and $\tau = T - t$ is the time to expiration. We see we write S(T) as a product of S(t), which is F(t) measurable, and a random variable independent of F(t).
Then sovling c(t, S(t)) becomes solving an expectation value of a random varible composed of a standard normal random variable. 
\begin{align*}
	c(t,x) = \tilde E[e^{-\tau r}(x exp\{-\sigma \sqrt{\tau} Y + (r -\frac{1}{2}\sigma^2)\tau\}-K)^+]
\end{align*}
Where Y is a standard normal distribution under $\hat P$.\\
After a little bit of math with integration, we have the solution to the Black-Scholes-Merton model for European call option
\begin{align*}
	c(\tau,x; K,r,\sigma) = xN(d_{+}(\tau,x)) - e^{-r\tau}KN(d_{-}(\tau,x))
\end{align*}
Where
\begin{align*}
	d_1 = \frac{1}{\sigma \sqrt{T-t}}[ln(\frac{S_t}{K}) + (r+ \frac{\sigma^2}{2})(T-t)] \\
	d_2 = d_1 - \sigma \sqrt{T-t}
\end{align*}
N() is the cumulative distribution function of the standard normail distribution
{\bf 2. European put option}
The payoff for the European put option is $V(T) = K - S(T)$, we can follow a similiar derivation and get the formula for put option
\begin{align*}
	p(t,x) = N(-d_2)Ke^{-r(T-t)} -N(-d_1)x
\end{align*}
{\bf 3. Alternative formulation}
\begin{align*}
	C(F,\tau)= D[N(d_+)F - N(d_-)K] \\
	P(F,\tau)= D[N(-d_-)K - N(-d_+)F] \\
	d_{+/-} = \frac{1}{\sigma \sqrt{\tau}}[ln(\frac{F}{K})+/-\frac{1}{2}\sigma^2 \tau] \\
\end{align*}
The variables are:\\
$\tau = T - t$ is the time to expiry\\
$D = e^{-r\tau}$ is the discount factor \\
$F = e^{r\tau} S$ is the forward price of the asset S.
\end{document}
