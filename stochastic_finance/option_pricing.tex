\documentclass[a4paper]{article}
\usepackage{amssymb, amsmath}
\usepackage{graphicx}
\begin{document}
Author: Dr. Shi Guo  \hspace{30mm} Email: guoshi1984@hotmail.com\\
\line(1,0){350}
\section{Black Scholes Merton Equation: Introduction}
We assume the stock prices following a geometric Brownian motion\\
1) Stock price:\\
\begin{align*}
	 dS(t) = \alpha S(t) dt + \sigma S(t) dW(t)\\
\end{align*}
 
\noindent 2) We have a portfolio X(t) which consists of $\Delta(t)$ share of stock  $\Delta(t) S(t)$, and $(X(t) - \Delta (t) S(t))$ money market account with interest rate r. \\  
\begin{align*}
	X(t) = \Delta(t) dS(t) + r(X(t) -\Delta(t) S(t)) dt\\
\end{align*}
\noindent 3) Change of the portfolio with respect to time
\begin{align*}
	  d X(t) & = \Delta(t) d S(t) + r(X(t) - \Delta(t) S(t)) dt \\
             & = r X(t) dt + \Delta(t) (\alpha - r )S(t) + \Delta (t) \sigma S(t) d W(t) \\
\end{align*}
                      
\noindent 4) Change of the present value of the stock with respect to time\\
\begin{align*}
	d(e^{-r t} S(t)) & = (\alpha - r) e^{- r t}S(t) dt + \sigma e^{- r t} S(t) dW(t) \\
\end{align*}

\noindent 5) With a few steps, we get change of the present value of the portfolio with respect to time\\
\begin{align*}
 & d(e^{- r t} X(t)) \\
 = & \Delta(t) (\alpha - r) e^{-rt} S(t) dt + \Delta(t) \sigma e^{-rt} S(t) dW(t)\\
\end{align*}

\noindent 6) Assume the option value is $c(t,S(t))$ and we apply Ito's formula\\
\begin{align*}
	& d(e^{-rt } c(t, S(t)) \\
= & e^{-rt} [- r c(t, S(t)) + c_t(t, S(t)) + \alpha S(t) \frac{\partial c(t, S(t))}{\partial S(t)}  +\frac{1}{2} \sigma^2 S^2(t) \frac{\partial^2 c(t, S(t))}{\partial S^2(t)} ] dt\\
  & +e^{-rt} \sigma S(t)  \frac{\partial c(t, S(t))}{\partial S(t)} dW(t)\\
\end{align*}
7)\noindent	Now equate Equation in 5) and 6), we get\\
dW(t) term:
\begin{align*}
	\Delta(t) = \frac{\partial c(t,S(t))}{\partial S(t)} \\	
\end{align*}
dt term:
\begin{align*}
	rc(t, S)  = c_t (t, S(t)) + r S(t) + \frac{1}{2} \sigma^2 S(t) \frac{\partial c(t, S(t))}{\partial S^2(t)}  \\
\end{align*}
which is known as Black-Scholes-Merton partial differential equation.\\
The terminal condition the equation satisfies for call option is
\begin{align*}
	c(T,S) = (S(T)-K)^+
\end{align*}
Similarly, for put option
\begin{align*}
	p(T,S) = (K-S(T))^+
\end{align*}
\section{Connection to Faynman-Kac formula}
In risk-neutral measure, we write the stock price as
\begin{align*}
	dS(t) = rS(t) dt + \sigma S(t) d \tilde W(t) 
\end{align*}
Where $\tilde W(t)$ is a standard Brownian motion under risk-neutral measure.\\
According to the risk-neutral pricing formula, the price of the derivative security at time t is
\begin{align}
	V(t) = \tilde E[e^{-r(T-t)}V(T)| F(t)] = \tilde E[e^{-r(T-t)}h(S(T))| F(t)] \label{rn}
\end{align}
Since the stock price is Markov and the payoff is a function fo the stock price alone, based on Faynman-Kac formula, there is a function $v(t,x)$ such that $V(t) = v(t, S(t))$, and $v(v, S(t))$ must satisfy discounted partial differential equation 
\begin{align*}
	v_t(t,x) + rx v_x(t,x) + \frac{1}{2} \sigma^2 x^2 v_{xx}(t,x) = rv(t,x)
\end{align*}
Now we have seen two ways of showing the Black-Scholes-Merton(BSM) equation. 
One way is to reproduce the payoff of the option using a portfolio that consists of a 
saving account. Another way is based on the risk-neutral pricing formula and Feynman-Kac formula. 
These two ways are equivalent. Because under risk-neutral measure, the payoff of a derivative is the same as a saving account, which imply we are able to reproduce the payoff using portfolio that consisting of a saving account.
\section{Black-Scholes-Merton Model: Analytic Solution for European Option}
{\bf1. European call option}\\
For European call option with payoff to be $V(T) = S(T) - K$, with K as strike price, let us assume constant volatility $\sigma$, and constant interest rate r. 
Then we can obtain the solution to the BSM equation with martingale property without bothering solving the complex parital
differential equation. The call option value satisfies
\begin{align*}
	c(t, S(t)) = \tilde E[e^{-r(T-t)}(S(T)-K)^+|F(t)]
\end{align*}
We write
\begin{align*}
	S(T) = S(t) exp\{\sigma(\tilde W(T) -\tilde W(t)) + (r - \frac{1}{2} \sigma^2) \tau\} \\
	     = S(t) exp\{-\sigma \sqrt{\tau} Y + (r - \frac{1}{2}\sigma^2)\tau\}
\end{align*}
Where Y is the stardard normal random variable and $\tau = T - t$ is the time to expiration. 
\begin{align*}
        Y = -\frac{\tilde W(T) - \tilde W(t)}{\sqrt{T - t}}
\end{align*}
So we see that $S(T)$ is the product of the $F(t)$ measurable random variable S(t) and random variable
\begin{align*}
        exp\{-\sigma \sqrt{\tau} Y +(r - \frac{1}{2} \sigma^2) \tau \}
\end{align*}
Which is independent of $F(t)$.
Therefore based on risk-neutral pricing formula[\ref{rn}]
\begin{align*}
        c(t,x) = \tilde E[e^{-r\tau}(x exp\{ -\sigma \sqrt{\tau} Y + (r - \frac{1}{2} \sigma^2)\tau\} -K)^+]\\
        = \frac{1}{\sqrt{2\pi}} \int_{-\infty}^{\infty} e^{-\tau r}(x exp\{-\sigma \sqrt{\tau} y
        +(r -\frac{1}{2}\sigma^2)\tau \} -K)^+ e^{-\frac{1}{2}y^2} dy
\end{align*}
After a little bit of math with integration, we have the solution to the Black-Scholes-Merton model for European call option
\begin{align*}
	c(\tau,x; K,r,\sigma) = xN(d_{+}(\tau,x)) - e^{-r\tau}KN(d_{-}(\tau,x))
\end{align*}
Where
\begin{align*}
	d_1 = \frac{1}{\sigma \sqrt{\tau}}[ln(\frac{S_t}{K}) + (r+ \frac{\sigma^2}{2})\tau] \\
	d_2 = d_1 - \sigma \sqrt{\tau}
\end{align*}
N() is the cumulative distribution function of the standard normail distribution
{\bf 2. European put option}\\
The payoff for the European put option is $V(T) = K - S(T)$, we can follow a similiar derivation and get the formula for put option
\begin{align*}
	p(t,x) = N(-d_2)Ke^{-r\tau} -N(-d_1)x
\end{align*}
{\bf 3. Boundary conditons}\\
Using the solution $c(t,x)$ and $p(t,x)$, we can easily check the boundary conditions when
time t approaches to expiration time T.\\
As we know
\begin{align*}
	d_1 = \frac{1}{\sigma \sqrt{\tau}}ln(\frac{S}{K}) 
	+ \frac{1}{\sigma}(r+\frac{\sigma^2}{2}\sqrt{\tau})
\end{align*}
When $\tau \to 0$, the second term decays much faster, so it vanishes. When $S>K$, $d_1$ goes to infinity, when $S<K$, $d_1$ goes to negative infinity.
Therefore, when $S>K$
\begin{align*}
	c(t,x) = S*N(+\infty) - K*N(+\infty) = S-K
\end{align*}
Therefore, when $S<K$
\begin{align*}
	c(t,x) = S*N(-\infty) - K*N(-\infty) = 0-0 = 0
\end{align*}
{\bf 4. Examples}\\
The following graphs show the change of option price with respect to different parameters.
\\
\\
\\
\\
\includegraphics[scale = 0.5]{option_price1.eps}\\
\includegraphics[scale = 0.5]{option_price2.eps}\\
{\bf 5. Alternative formulation}\\
If we introduce $F = e^{r\tau} S$, which is the forward price of the asset S. Then the equation pricing equation becomes
\begin{align*}
	C(F,\tau)= D[N(d_+)F - N(d_-)K] \\
	P(F,\tau)= D[N(-d_-)K - N(-d_+)F] \\
	d_{+/-} = \frac{1}{\sigma \sqrt{\tau}}[ln(\frac{F}{K})+/-\frac{1}{2}\sigma^2 \tau] \\
\end{align*}
The variables are:\\
$\tau = T - t$ is the time to expiry\\
$D = e^{-r\tau}$ is the discount factor \\
\section{Heston Stochasic Volatility Model}
\subsection{Introduction}
The Black-Scholes equation assumes the volatility is constant, which is the ideal case and not practical in the real market. The Heston model assumes the volatility to follow a stochastic process. Suppose a stock price under risk-neutral measure is governed by
\begin{align}
	dS(t) = rS(t)dt + \sqrt{V(t)} S(t) \tilde dW_1(t)
\end{align}
and the volatility itself is governed by the equation
\begin{align}
	dV(t) = (a -bV(t))dt + \sigma \sqrt{V(t)} \tilde dW_2(t)
\end{align}
Where 
\begin{align*}
	\tilde dW_1(t) \tilde dW_2(t) = \rho dt
\end{align*}
At time t, the risk-neutral price of a call expiring at time $T \geq t$ in this model is
\begin{align*}
	c(t, S(t), V(t)) = \tilde E[e^{-r(T-t)}(S(T)-K)^+|F(t)]
\end{align*}
If we move the term $c^{rt}$ to the left hand side, we see
\begin{align}
	e^{-rt}c(t, S(t), V(t)) = \tilde E[e^{-rT}(S(T)-K)^+|F(t)]
\end{align}
which satisfies the martingale property.
Then we take the differentiation of $e^{-rt}c(t, S(t), V(t))$. We get
\begin{align*}
	& d(e^{-rt}c(t, S(t), V(t))\\
	= & \frac{\partial e^{-rt}}{\partial t} c(t, S(t), V(t)) 
	+ e^{-rt}\frac{\partial c(t, S(t), V(t))}{\partial t} dt \\
	= & -re^{-rt}c(t, S(t),V(t))dt \hspace{2mm} (1)\\
	+ & e^{-rt} \frac{\partial c}{\partial t} dt \hspace{2mm} (2) \\
	+ & e^{-rt} \frac{\partial c}{\partial S} dS \hspace{2mm} (3) \\
	+ & e^{-rt} \frac{\partial^2 c}{\partial S^2} dS dS \hspace{2mm} (4)\\
	+ & e^{-rt} \frac{\partial c}{\partial V} dV \hspace{2mm} (5) \\
	+ & e^{-rt} \frac{\partial^2 c}{\partial V^2} dV dV\hspace{2mm} (6)\\
	+ & e^{-rt} \frac{\partial^2 c}{\partial V \partial S} dV dS\hspace{2mm} (7)
\end{align*}
As we are interested in only the dt terms, we find out the dt terms from (1) to (7)
the dt term in (1) is
\begin{align*}
	-rc(t, S(t), V(t))e^{-rt}dt
\end{align*}
the dt term in (2) is
\begin{align*}
	\frac{\partial c}{\partial t}e^{-rt}dt
\end{align*}
the dt term in (3) is
\begin{align*}
	\frac{\partial c}{\partial S}rSe^{-rt}dt
\end{align*}
the dt term in (4) is
\begin{align*}
	\frac{1}{2} \frac{\partial^2 c}{\partial S^2}VS^2e^{-rt}dt
\end{align*}
the dt term in (5) is
\begin{align*}
	\frac{\partial c}{\partial V}(a-bV(t))e^{-rt}dt
\end{align*}
the dt term in (6) is
\begin{align*}
	\frac{1}{2} \frac{\partial^2 c}{\partial V^2}V\sigma^2e^{-rt}dt
\end{align*}
the dt term in (7) is
\begin{align*}
	\frac{\partial^2 c}{\partial V \partial S} VS\sigma e^{-rt}dt
\end{align*}
Collect all the dt terms and let those terms equal to zero, we get
\begin{align}\label{c}
	c_t + rs c_s + (a-bv)c_v + \frac{1}{2} s^2 v c_{ss} + \rho \sigma svc_{sv}
	+\frac{1}{2} \sigma^2vc_{vv} = rc
\end{align}
The function $c(t, s, v)$ satisfies boundary condition
\begin{align*}
	c(T, s, v) = (s-K)^+ \\
	c(t, 0, v) = 0 \\
	c(t, s, 0) = (s - e^{-r(T-t)}K)^+\\
	lim_{s \to \infty} \frac{c(t, s, v)}{s-K} = 1 \\
	lim_{v \to \infty} c(t, s, v) = s\\  
\end{align*}
Based on the solution to the BSM equation, we can guess that the solution has the following form
\begin{align}
	c(t,s,v) = sf(t, logs, v) - e^{-r(T-t)}Kg(t,logs,v)
\end{align}	
Where f and g can be intepreted as a cumulative distribution function.
Then since c(t,s,v) satisfies the partial differential equation \ref{c}, we can show that $f$ and $g$ satisfy the following
\begin{align}\label{f}
	f_t + (r + \frac{1}{2}v)f_x + (a-bv+\rho \sigma v)f_v 
	+ \frac{1}{2}  v f_{xx} + \rho \sigma vf_{xv}
        +\frac{1}{2} \sigma^2 vf_{vv} = 0\\
\end{align}
\begin{align}
	g_t + (r - \frac{1}{2}v)g_x + (a-bv)g_v 
	+ \frac{1}{2}  v g_{xx} + \rho \sigma vg_{xv}
        +\frac{1}{2} \sigma^2 vg_{vv} = 0\\
\end{align}
The derivation is straightforward but one needs to keep in mind here we treat x and v as two independent variables. The above PDE for f and g satisfy boundary condition
\begin{align*}
	f(T, x, v) = 1_{x\geq logK}\\
	g(T, x, v) = 1_{x\geq logK}\\
\end{align*}
This implies that f and g can be intepreted as "Probabilities". We can define
\begin{align*}
	f(t, x, v) = E^{t,x,v}1_{x\geq logK}\\
\end{align*}
We suppose a pair of stochastic process X(t), V(t) given by the following expression
\begin{align*}
	dX(t) = (r+\frac{1}{2}V(t))dt + \sqrt{V(t)}dW_1(t)
	dV(t) = (a-bV(t)+ \rho \sigma V(t))dt + \sigma \sqrt{V(t)}dW_2(t)
\end{align*}
By F-K formula, we can show that f satisfies the PDE above. Similarly, we have
\begin{align*}
	g(t, x, v) = E^{t,x,v} 1_{x\geq logK}\\
\end{align*}
and the stochastic process of X(t) and V(t) are
\begin{align*}
	dX(t) = (r-\frac{1}{2}V(t))dt + \sqrt{V(t)}dW_1(t)\\
	dV(t) = (a-bV(t))dt + \sigma \sqrt{V(t)}dW_2(t)
\end{align*}
To find the analytical solution of f(t, x, v) and g(t, x, v) is not an easy task. Instead, we do Fourier transform of f and g. 
First we work out the function f(t, x, v. ). Let $\tau = T-t$
\begin{align*}
	\tilde f(k, v, \tau) = \int_{-\infty}^{\infty}dx e^{-ikx}f(x, v, \tau)
\end{align*}
The inverse Fourier transform is
\begin{align*}
	f(x, v, \tau) = \int_{-\infty}^{\infty}\frac{dk}{2\pi} e^{ikx}
	\tilde f(k, v, \tau)
\end{align*}
Substitute this into equation \ref{f}, then
\begin{align*}
	-\frac{\partial \tilde f}{\partial \tau}
	+ (r + \frac{1}{2})ik \tilde f 
	+ (a-bv+\rho \sigma v)\frac{\partial \tilde f}{\partial v}
        -\frac{1}{2}vk^2 \tilde f 
	+ \rho \sigma v ik \frac{\partial \tilde f}{\partial v}
        +\frac{1}{2} \sigma^2 v \frac{\partial^2 \tilde f}{\partial v^2}  = 0\\
\end{align*}
Now the problem is to solve for $\tilde f$. We note when $\tau=0$,
\begin{align*}
	\tilde f(k, v, 0) = \int_{-\infty}^{\infty}dx e^{-ikx}f(x, v, \tau=0)
			  = \int_{-\infty}^{\infty}dx e^{-ikx}1_{x \geq logK}
			     = \int_{0}^{\infty}dx e^{-ikx}
			     = \pi \delta(k) + \frac{1}{ik}
\end{align*}
when $\tau!=0$, we guest a general solution which has the following form
\begin{align*}
	\tilde f(k, v, \tau) = exp(C\tau +D\tau v) \tilde f(k,v,0)
\end{align*}
From above we easily see it match the terminal condition at $\tau to 0$. With inverse Fourier transform
\begin{align*}
	f(x, v, \tau, x) & = \int_{-\infty}^{\infty}\frac{dk}{2\pi} e^{ikx} \tilde f(k, v, \tau)\\
			 & = \int_{0}^{\infty}\frac{dk}{2\pi} e^{ikx} exp(C\tau + D\tau v) \tilde f(k, v, 0)\\
			 & = \int_{0}^{\infty}\frac{dk}{2\pi} e^{ikx} exp(C\tau + D\tau v) 
			 	(\pi \delta(k) + \frac{1}{ik}) \\
			 & = \frac{1}{2} + \int_{0}^{\infty}\frac{dk}{2\pi} e^{ikx} exp(\frac{C\tau + D\tau v}{ik}) 
\end{align*}
Now the only remaining task is to find C and D.
\end{document}
