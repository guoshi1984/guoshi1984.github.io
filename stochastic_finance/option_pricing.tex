\documentclass[a4paper]{article}
\usepackage{amssymb, amsmath}
\usepackage{graphicx}
\begin{document}
Author: Dr. Shi Guo  \hspace{30mm} Email: guoshi1984@hotmail.com\\
\line(1,0){350}
\section{Black Scholes Merton Equation}
We assume the stock prices following a geometric Brownian motion\\
1) Stock price:\\
\begin{align*}
	 dS(t) = \alpha S(t) dt + \sigma S(t) dW(t)\\
\end{align*}
 
\noindent 2) We have a portfolio X(t) which consists of $\Delta(t)$ share of stock  $\Delta(t) S(t)$, and $(X(t) - \Delta (t) S(t))$ money market account with interest rate r. \\  
\begin{align*}
	X(t) = \Delta(t) dS(t) + r(X(t) -\Delta(t) S(t)) dt\\
\end{align*}
\noindent 3) Change of the portfolio with respect to time
\begin{align*}
	  d X(t) & = \Delta(t) d S(t) + r(X(t) - \Delta(t) S(t)) dt \\
             & = r X(t) dt + \Delta(t) (\alpha - r )S(t) + \Delta (t) \sigma S(t) d W(t) \\
\end{align*}
                      
\noindent 4) Change of the present value of the stock with respect to time\\
\begin{align*}
	d(e^{-r t} S(t)) & = (\alpha - r) e^{- r t}S(t) dt + \sigma e^{- r t} S(t) dW(t) \\
\end{align*}

\noindent 5) With a few steps, we get change of the present value of the portfolio with respect to time\\
\begin{align*}
 & d(e^{- r t} X(t)) \\
 = & \Delta(t) (\alpha - r) e^{-rt} S(t) dt + \Delta(t) \sigma e^{-rt} S(t) dW(t)\\
\end{align*}

\noindent 6) Assume the option value is $c(t,S(t))$ and we apply Ito's formula\\
\begin{align*}
	& d(e^{-rt } c(t, S(t)) \\
= & e^{-rt} [- r c(t, S(t)) + c_t(t, S(t)) + \alpha S(t) \frac{\partial c(t, S(t))}{\partial S(t)}  +\frac{1}{2} \sigma^2 S^2(t) \frac{\partial^2 c(t, S(t))}{\partial S^2(t)} ] dt\\
  & +e^{-rt} \sigma S(t)  \frac{\partial c(t, S(t))}{\partial S(t)} dW(t)\\
\end{align*}
7)\noindent	Now equate Equation in 5) and 6), we get\\
dW(t) term:
\begin{align*}
	\Delta(t) = \frac{\partial c(t,S(t))}{\partial S(t)} \\	
\end{align*}
dt term:
\begin{align*}
	rc(t, S)  = c_t (t, S(t)) + r S(t) + \frac{1}{2} \sigma^2 S(t) \frac{\partial c(t, S(t))}{\partial S^2(t)}  \\
\end{align*}
which is known as Black-Scholes-Merton partial differential equation.
\section{Connection to Faynman-Kac formula}
In risk-neutral measure, we write the stock price as
\begin{align*}
	dS(t) = rS(t) dt + \sigma S(t) d \tilde W(t) 
\end{align*}
Where $W(t)$ is a standard Brownian motion under risk-neutral measure.\\
According to the risk-neutral pricing formula, the price of the derivative security at time t is
\begin{align*}
	V(t) = \tilde E[e^{-r(T-t)}V(T)| F(t)] = \tilde E[e^{-r(T-t)}h(S(T))| F(t)]
\end{align*}
Since the stock price is Markov and the payoff is a function fo the stock price alone, based on Faynman-Kac formula, there is a function $v(t,x)$ such that $V(t) = v(t, S(t))$, and $v(v, S(t))$ must satisfy discounted partial differential equation 
\begin{align*}
	v_t(t,x) + rx v_x(t,x) + \frac{1}{2} \sigma^2 x^2 v_{xx}(t,x) = rv(t,x)
\end{align*}
Now we have seen two ways of showing the Black-Scholes-Merton equation. One way is to reproduce the payoff of the option using a portfolio that consists of a saving account. Another way is based on the risk-neutral pricing formula and Feynman-Kac formula. These two ways are equivalent. Because under risk-neutral measure, the payoff of a derivative is the same as a saving account, which imply we are able to reproduce the payoff using portfolio that consisting of a saving account.
\end{document}