\documentclass[a4paper]{article}
\usepackage{amssymb, amsmath}
\usepackage{graphicx}
\usepackage{hyperref}
\begin{document}
Author: Dr. Shi Guo  \hspace{30mm} Email: guoshi1984@hotmail.com\\
\line(1,0){350}
\section{LU Decomposition}
{\bf a. Motivation}\\
When we solve a linear system equation $Ax=b$, imagine we are able to decompose A into L and U and $A=LU$, where L is a lower triangular matrix U is an upper triangular matrix. Then equation $Ax=b$ becomes $LUx=b$. Let $Ux=y$, as both L and U are triangular matrix, we can solve for y in $Ly=b$ first, then solve for x in $Ux=y$ second.\\
{\bf b. Formula}\\
Decomposition form, $LU=A$\\
$\left( \begin{array}{ccccc}
1      & 	0 	& 	0 	 & 	 0 		& 	0 	\\
l_{21} & 	1 	& 	0 	 &	 0 		& 	0	\\
l_{31} & l_{32} &   1  	 &   0		&	0	\\
l_{41} & l_{42} & l_{43} &   1 		& 	0	\\
l_{51} & l_{52} & l_{53} & l_{54} 	& 	1	\end{array} \right)$
$\left( \begin{array}{ccccc}
u_{11} 	& u_{12} & 	 u_{13}  & 	u_{14} & u_{15} 	\\
	0	& u_{22} & 	 u_{23}  &	u_{24} & u_{25}	\\
	0	& 	0	 &   u_{33}  &  u_{34} &	u_{35}	\\
	0 	& 	0	 & 	 0		 &  u_{44} & 	u_{45}	\\
	0	& 	0	 & 	 0 		 &  0    & 	u_{55}	\\\end{array} \right)
$
=$\left( \begin{array}{ccccc}
a_{11} & a_{12} & 	a_{13} 	 & 	 a_{14} 	& 	a_{15} 	\\
a_{21} & a_{22} & 	a_{23} 	 &	 a_{24} 	& 	a_{25}	\\
a_{31} & a_{32} &   a_{33}   &   a_{34}		&	a_{35}	\\
a_{41} & a_{42} & 	a_{43} 	 &   a_{44} 	& 	a_{45}	\\
a_{51} & a_{52} & 	a_{53}	 & 	 a_{54} 	& 	a_{55}	\\\end{array} \right)\\
$
we run the calculation starting from the 1st column to the last column, and for each column we update from
the top row to the bottom row. For each element $a_{ij}$, we write it as a product of l's row and u's column.
We start from $a_{11}$\\
\begin{align*}
	a_{11} = u_{11} \Rightarrow u_{11}\\
	a_{21} = l_{21}u_{11} \Rightarrow l_{21}\\
	a_{31} = l_{31}u_{11} \Rightarrow l_{31}\\
	a_{41} = l_{41}u_{11} \Rightarrow l_{41}\\
	a_{51} = l_{51}u_{11} \Rightarrow l_{51}\\
	a_{12} = u_{12} \Rightarrow u_{12}\\
	a_{22} = l_{21}u_{12} + u_{22} \Rightarrow u_{12}\\
	a_{32} = l_{31}u_{12} + l_{32} u_{22} \Rightarrow u_{22}\\
\end{align*}
We keep going in this way, we can get $l_{ij}$ and $u_{ij}$ one by one, and in the end we get the whole L and U matrices. Now let us try to summarize this routine by a few formula.\\
\noindent 1) for $i<j$, which are the upper triangular part, namely U part(except diagonal), for example, for i=2, j=3, the matrix multiplication takes the 2nd row of L and the 3rd column of U(in bold)\\
$\left( \begin{array}{ccccc}
1      & 	0 	& 	0 	 & 	 0 		& 	0 	\\
	{\bf l_{21}} & 	{\bf 1} 	& 	0 	 &	 0 		& 	0	\\
. & . &   1  	 &   0		&	0	\\
. & . & . &   1 		& 	0	\\
. & . & . & . 	& 	1	\\\end{array} \right)$
$\left( \begin{array}{ccccc}
	. 	&  . &  {\bf u_{13}}  & 	. & . 	\\
	0	&  . & 	{\bf u_{23}}  &	. & .	\\
	0	& 	0	 &   .  & . &	.	\\
	0 	& 	0	 & 	 0		 &  . & 	.	\\
	0	& 	0	 & 	 0 		 &  0    & .	\\\end{array} \right)\\
$\\
We get the following equation
\begin{align*}
	l_{21}u_{13} + u_{23} = a_{23}
\end{align*}
and we can calculate $u_{23}$ as
\begin{align*}
	u_{23} = a_{23} - l_{21}u_{13}
\end{align*}
as $l_{21}$ and $u_{13}$ are known.
Generally,\\
\begin{equation}
	\sum_{k=1}^{i-1} l_{ik} u_{kj} + u_{ij} = a_{ij}
\end{equation}
and
\begin{align*}
	u_{ij} = a_{ij} - \sum_{k=1}^{i-1} l_{ik} u_{kj} 
\end{align*}
\\
We use the above equation to calculate $u_{ij}$\\
\noindent 2) for $i=j$, which are diagonal part, for example, for i=3, j=3, the matrix multiplication form takes the 3rd row and 3rd column(in bold)\\
$\left( \begin{array}{ccccc}
1      & 	0 	& 	0 	 & 	 0 		& 	0 	\\
. & 	1 	& 	0 	 &	 0 		& 	0	\\
{\bf l_{31}} & {\bf l_{32}} &   1  	 &   0		&	0	\\
. & . & . &   1 		& 	0	\\
. & . & . & . 	& 	1	\\\end{array} \right)$
$\left( \begin{array}{ccccc}
	. 	&  	.	 &  {\bf u_{13}}  & 	. & . 	\\
	0	&  	. 	 & 	{\bf u_{23}}  &	. & .	\\
	0	& 	0	 &   {\bf u_{33}} & . &	.	\\
	0 	& 	0	 & 	 0		 &  . & 	.	\\
	0	& 	0	 & 	 0 		 &  0    & .	\\\end{array} \right)\\
$\\

\begin{align*}
	l_{31}u_{13} + l_{32}u_{23} + u_{33}= a_{33}
\end{align*}
and we can calculate $u_{33}$ as
\begin{align*}
	u_{33} = a_{33} - l_{31}u_{13} - l_{32}u_{23}
\end{align*}
as $l_{31}$ and $l_{32}$ are known.
Generally\\
\begin{equation}
	\sum_{k=1}^{j-1} l_{jk} u_{kj} + u_{jj} = a_{jj}	
\end{equation}
\\
We use the above equation to calculate $u_{jj}$\\
\begin{equation}
	u_{jj} = a_{jj} - \sum_{k=1}^{j-1} l_{jk} u_{kj}
\end{equation}
\noindent 3) for $i>j$, which are the lower triangular part, namely L part(except diagonal), for example, for i=3, j=2, the matrix multiplication form takes the 3rd row and 2nd column\\
$\left( \begin{array}{ccccc}
1      & 	0 	& 	0 	 & 	 0 		& 	0 	\\
. & 	1 	& 	0 	 &	 0 		& 	0	\\
{\bf l_{31}} & {\bf l_{32}} &   1  	 &   0		&	0	\\
. & . & . &   1 		& 	0	\\
. & . & . & . 	& 	1	\\\end{array} \right)$
$\left( \begin{array}{ccccc}
	. 	&  {\bf u_{12}}  & . & 	. & . 	\\
	0	&  {\bf u_{22}}  & .  &	. & .	\\
	0	& 	0	 &   .  & . &	.	\\
	0 	& 	0	 & 	 0		 &  . & 	.	\\
	0	& 	0	 & 	 0 		 &  0    & .	\\\end{array} \right)\\
$
\begin{align*}
	l_{31}u_{12} + l_{32}u_{22} = a_{32}
\end{align*}
we use the above the calculate $l_{ij}$
\begin{align*}
	l_{32} = (a_{32} - l_{31}u_{12})/u_{22}
\end{align*}
as $l_{31}$, $u_{12}$ and $u_{22}$ are known.
Generally,\\
\begin{equation}
	\sum_{k=1}^{j-1} l_{ik} u_{kj} + l_{ij}u_{jj} = a_{ij}	
\end{equation}
\begin{equation}
	  l_{ij}u_{jj} = a_{ij} -\sum_{k=1}^{j-1} l_{ik} u_{kj}	
\end{equation}
\begin{equation}
	  l_{ij} = \frac{1}{u_{jj}}(a_{ij} -\sum_{k=1}^{j-1} l_{ik} u_{kj})	
\end{equation}
We use the above equation to calculate $l_{ij}$\\
In summary, we use the ith row of L and jth column of U plus $a_{ij}$ to calculate either $l_{ij}$ and $u_{ij}$.
\noindent {\bf c. Inplace Update}\\
How can do in place update? In the update formula, each $a_{ij}$ is used only once, and then we get either $l_{ij}$ or $u_{ij}$, so we can store the $l_{ij}$ or $u_{ij}$  in the exact same place where $a_{ij}$ is stored. This is in-place update.
For example, for a certain step i=4, j=3, our in-place updated matrix A looks like\\
$\left( \begin{array}{ccccc}
u_{11} & u_{12} & 	u_{13} 	 & 	 a_{14} 	& 	a_{15} 	\\
l_{21} & u_{22} & 	u_{23} 	 &	 a_{24} 	& 	a_{25}	\\
l_{31} & l_{32} &   u_{33}   &   a_{34}		&	a_{35}	\\
l_{41} & l_{42} & 	l_{43} 	 &   a_{44} 	& 	a_{45}	\\
l_{51} & l_{52} & 	a_{53}	 & 	 a_{54} 	& 	a_{55}	\\\end{array} \right)\\
$
\noindent {\bf d. Partial Pivoting}\\
For each $l_{ij}$, we need to divided by $u_{jj}$, which is in the diagonal position, from equation 3 
\begin{equation}
	u_{jj} = a_{jj} - \sum_{k=1}^{j-1} l_{jk} u_{kj}
\end{equation}
we never like dividing a really small number. So we check the rows below $u_{jj}$(the diagonal) with the following question: do we get a bigger diagonal $u_{jj}$ if we exchange any of the following rows(the ith rows with $i>j$) with the jth row, are we able to get a bigger $u_{jj}$ to use as a divider? We do this by subsitute the row index j with i($i > j$)
\begin{align*}
	divider = a_{ij} - \sum_{k=1}^{j-1} l_{ik} u_{kj}
\end{align*}
The right hand side of the above equation is the same as the right hand side of equation 5. This means we can do this step without duplicating any calculation. So for all rows with$(i \geq j)$ we calculate the dividers based on the above equation(this is equivalent of calculating Eq.3 and Eq.5), then choose the largest divider. The next step is to interchange the jth row with the row that has the largest divider, divide all the rest rows by the largest divider to get $l_{ij}$(This is equivalent to Eq.6).\\

\noindent{\bf e. Scaled Partial Pivoting}\\
Notice the fact that if we multiply the row of the matrix by a very large number, the solution to the linear system equation does not change. So in order to compare the divider's value $u_{jj}$, we can not compare the true value, we rather need to compare the scaled value, which is the value divided by the largest element in the row.\\

\noindent{\bf f. Implementation Details}\\
When we use scaled partial pivoting, we basically change the order of rows. So the decomposed matrix is not the original matrix A, rather than a row permutation matrix of A. When we solve the equation $Ax=b$, since A has been change after LU decomposition, we need b to be changed in the same way of row permutation as A so that A and b are both consistent. This requires us to keep track of the order of the rows of matrix A. We need store the new row index into an array. We do not need to know which rows we exchange at every step, and knowing the order is enough. One more thing we need to take care of is regarding how to evaluate the determinant. When we interchange two rows of the matrix, the determinant gets a minus sign. So we need to store whether we exchange the rows even number of times or odd number of times, we call this parity, which takes two values only, +1 and -1.\\

\noindent{\bf g. Java Implementation}\\
The java code is on github repo\\ 
\url{https://github.com/guoshi1984/guoshi1984.github.io/tree/master/quant\_lib/src/math}
\end{document}
