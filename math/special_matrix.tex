\documentclass[a4paper]{article}
\usepackage{amssymb, amsmath}
\usepackage{graphicx}
\begin{document}
Author: Dr. Shi Guo  \hspace{30mm} Email: guoshi1984@hotmail.com\\
\line(1,0){350}
\section{Symmetric Matrix}
{\bf Definition}\\
\begin{align*}
	A = A^T
\end{align*}
{\bf Properties}\\
1)	We know the eigenvectors associated with distinct eigenvalues are linearly independent for all matrix. The eigenvectors associate with distinct eigenvalues ($v_1$, $v_2$, ��$v_n$ )of a symmetric matrix are not only linearly independent but also orthogonal. So let V be the matrix whose columns are the eigenvectors of A, then $VV^T =I$, $V^{-1} = V^T$. If with same eigenvalues, then the eigenvector may not be orthogonal,  we can do Gram-Schmit transformation to make it orthogonal.\\
2)	The diagonal factorization of an symmetric matrix is\\
\begin{align*}
	A = & VCV^T\\
      = & AI\\
	  = & A \sum_i v_i v_i^T\\
	  = &\sum_i c_i v_i v_i^T\\
\end{align*}

3)	Maximum value of A��s quadratic form\\
\begin{align*}
	&x^TAx\\
= &x^T\sum_i c_i v_i v_i^T x\\
= &\sum_i b^T V^T v_i v_i^TVbc_i\\
= &\sum_i b_i^2 c_i\\
<= &max (c_i) b^T b\\
= &max (c_i)  x^Tx\\
\end{align*}

\section{Hermitian Matrix}
A = A*(where A* is the complex conjugate of A)\\
1)	The eigenvalues are real.\\

\section{Orthogonal Matrix}
{\bf Definition}\\
$A{-1} = A^T$\\
{\bf Intuition}\\
Orthogonal matrix arise from dot product. Consider vector u , and a matrix Q. When we apply the matrix Q to v, we get $v^{'}= Q v$. We would like to have the dot product preserved, namely
\begin{align*}
	v^T v = v^{'T} v^{'} = (Qv)^T (Qv) = v^T Q^T Q v
\end{align*}
So $Q^T Q =1$, $Q^T = Q^{-1}$.\\
{\bf Properties}\\
Orthogonal matrices imply orthogonal transformations. Examples include rotations, reflections and combinations\\
{\bf Examples}\\
1) Rotation Matrix
\begin{align*}
	\left(  \begin{array} {cc}
		cos \theta & - sin \theta\\
		sin \theta & cos \theta \\
		\end{array}
		\right)
\end{align*}
2) Reflection Matrix
\begin{align*}
	\left(  \begin{array} {cc}
		1 & 0\\
		0 & -1 \\
		\end{array}
		\right)
\end{align*}

\section{Idempotent Matrix}
{\bf Definition}\\
$A^2 = A$\\
{\bf Properties}\\
1)	Its eigenvalues are either 0 or 1.
Because the eigenvalues of $A^2$ are the squares of the eigenvalues of A.
2)	Any vector in the columns space of an idempotent matrix A is an eigenvector of A
3)	The number of eigenvalues that are 1 is the rank of an idempotent matrix. $tr(A) = rank(A)$

\section{Symmetric Positive Definite}
{\bf Definition}\\
A symmetric positive definite matrix satisfies for any non-zero vector x, $x^T A x >0$\\
{\bf Properties}
1)	Positive definite matrix is non-singular.\\
Proof: If A is singular, it means there is a non-zero vector x so that Ax=0.  Therefore $x^T A x = 0$, which is a contradiction.\\
2)	All the eigenvalues are positive.\\
3)	Its leading principal minors are all positive.\\
4)	It has a unique Cholesky decomposition. \\

\end{document}